\documentclass[./main.tex]{subfiles}
\begin{document}
  
\begin{prop}
  
  Let $X$ be a smooth classical scheme of
  locally almost finite type.
  Then there is an equivalence $\CRYS^L(X) \simeq \DIFF_X\MOD$
  such that the following commutes up to natural isomorphism : 
  \begin{cd}
    {\mathrm{Crys}^L(X)} & {\mathrm{Diff}_X\mathrm{Mod}} \\
    {\mathrm{QCoh}\,X} & {\mathrm{QCoh}\,X}
    \arrow["{\mathrm{oblv}^L}"', from=1-1, to=2-1]
    \arrow[from=1-2, to=2-2]
    \arrow["{\id{}}"', from=2-1, to=2-2]
    \arrow["\sim", from=1-1, to=1-2]
  \end{cd}
  The right vertical morphism is the forgetful functor.
\end{prop}
\begin{proof}
  
  Outline : 
  \begin{enumerate}
    \item $\CRYS^L \rightleftarrows \QCOH\,X$ and 
    $\DIFF_X\MOD \rightleftarrows \QCOH\,X$ are both monadic.
    So they are determined by their monads.

    \item Under the integral transform equivalence
    $\QCOH (X\times X) \simeq \DGCAT_\CTS(\QCOH\,X , \QCOH\,X)$,
    algebra objects on the left correspond to 
    algebra objects on the right, i.e. monads.
    Therefore the monads $\OBLV^L\,\INDUCE^L$ and $\DIFF_X \otimes \_$
    correspond to quasi-coherent sheaves on $X \times X$.

    \item Of course, $\DIFF_X \otimes \_$ corresponds to 
    $\DIFF_X$ on $X \times X$.

    \item
    Let $\s{D}^L_X \in \QCOH(X\times X)$ correspond to $\OBLV^L\,\INDUCE^L$.
    GR Crystals 5.3.6 shows $\s{D}^L_X$ lies $\QCOH(X\times X)^\heartsuit$.

    \item
    GR Crystals 5.4.1 claims
    for any $\FF, \GG \in (\QCOH\,X)^\heartsuit$ and 
    $\QQ \in \QCOH(X \times X)^\heartsuit$ 
    with set-theoretically supported on the diagonal,
    \begin{align*}
      (p_2)_* (p_1^! \FF \otimes \QQ) \to \GG \,\,\,\,\,\,
      \leftrightsquigarrow\,\,\,\,\,\,\,\, & p_1^! \FF \otimes \QQ \to p_2^! \GG \\
      \leftrightsquigarrow\,\,\,\,\,\,\,\, & \QQ \to \underline{\HOM}(p_1^! \FF , p_2^!\GG) \\
      \leftrightsquigarrow\,\,\,\,\,\,\,\, & \QQ \to \DIFF_X(\FF , \GG)
    \end{align*}
    Applying to $\FF = \GG = \OO_X, \QQ = \s{D}^L_X$,
    we obtain a morphism $\s{D}^L_X \to \DIFF_X$.
    This is a morphism of algebra objects on $\QCOH(X\times X)$.

    \item GR Crystals 5.4.3 claims $\s{D}^L_X \map{\sim}{} \DIFF_X$ 
    is a classical computation.

  \end{enumerate}

  (1) $\DIFF_X\MOD$ is by definition the $\infty$-category of
  modules in $\QCOH\,X$ over the monad corresponding to 
  $\DIFF_X \in \QCOH(X\times X)$,
  so $\DIFF_X\MOD \rightleftarrows \QCOH\,X$ is monadic.
  
  For $\CRYS^L\,X$,
  by Lurie HA 4.7.0.3, 
  it suffices to show that 
  \begin{itemize}
    \item $\OBLV^L$ is conservative
    \item $\CRYS^L\,X$ has geometric realisations and $\OBLV^L$ preserves them.
  \end{itemize}
  % Recall we have the following commuting diagram : 
  % \begin{cd}
  %   {\mathrm{Crys}^L\,X} & {\mathrm{Crys}^R \, X} \\
  %   {\mathrm{QCoh}\,X} & {\mathrm{IndCoh}\,X}
  %   \arrow["{\Upsilon_X}", from=2-2, to=2-1]
  %   \arrow["{\mathrm{oblv}^L}"', from=1-1, to=2-1]
  %   \arrow["{\mathrm{oblv}^R}", from=1-2, to=2-2]
  %   \arrow["{\Upsilon_{X_\mathrm{dR}}}"', from=1-2, to=1-1]
  % \end{cd}
  % where the bottom morphism is an equivalence since
  % $X$ is smooth and classical,
  % the top morphism is the equivalence between left and right crystals.
  % So it suffices to show that $\CRYS^R\,X$ and $\OBLV^R$ has 
  % the deserved properties.

  % By definition, $\INDCOH^! : \PSTK_\LAFT \to (\DGCAT_\CTS)^\OP$ so 
  % $\CRYS^R\,X$ has all small colimits and $\OBLV^R$ preserves them.
  % This applies in particular to geometric realisations.
  % Now for conservativity of $\OBLV^R$.
  % Since $X$ is a smooth classical scheme of locally almost finite type,
  % it is classically formally smooth and hence
  % we have the equivalence 
  % \begin{cd}
  %   {\mathrm{Crys}^R\,X} & {\varprojlim \mathrm{IndCoh}(\check{C}(X/X_\mathrm{dR}))}
  %   \arrow["\sim", from=1-1, to=1-2]
  % \end{cd}

  Since $X$ is a laft smooth classical scheme,
  \linkto{crys.sm_implies_cfsm}{it is classically formally smooth}.
  This \linkto{crys.cfsm}{implies} the equivalence 
  between left crystals and infinitesimally equivariant quasi-coherent sheaves :
  $\CRYS^L\,X \map{\sim}{} \LIM \QCOH(\check{C}(X / X_\DR))$.
  Now take $\FF \in \CRYS^L\,X$ and assume $\OBLV^L\,\FF \simeq 0$.
  The ``forgetful functor'' $\OBLV^L$ is the morphism from
  $\CRYS^L\,X$ to the zero-th part of the Cech nerve.
  So $\OBLV^L\,\FF \simeq 0$ implies $\FF$ is zero in 
  $\LIM \QCOH(\check{C}(X / X_\DR))$ and hence in $\CRYS^L\,X$ as desired. 

  (2) 

  GR2 Ch2 1.6.11 - definition of ind-proper morphism.

  (3)

  (4)

  (5)

  (6)

\end{proof}

\end{document}