\documentclass[./main.tex]{subfiles}
\begin{document}
  
\begin{prop}[Main Result]
  
  Let $X$ be a smooth proper classical scheme of
  locally almost finite type.
  Then there is an equivalence $\CRYS^L(X) \simeq \DIFF_X\MOD$
  such that the following commutes up to natural isomorphism : 
  \begin{cd}
    {\mathrm{Crys}^L(X)} & {\mathrm{Diff}_X\mathrm{Mod}} \\
    {\mathrm{QCoh}\,X} & {\mathrm{QCoh}\,X}
    \arrow["{\mathrm{oblv}^L}"', from=1-1, to=2-1]
    \arrow[from=1-2, to=2-2]
    \arrow["{\id{}}"', from=2-1, to=2-2]
    \arrow["\sim", from=1-1, to=1-2]
  \end{cd}
  The right vertical morphism is the forgetful functor.
  \cite[Section 5.5.5]{Crys}
\end{prop}
\begin{proof}
  \begin{enumerate}
    \item $\CRYS^L\,X \rightleftarrows \QCOH\,X$ and 
    $\DIFF_X\MOD \rightleftarrows \QCOH\,X$ are both monadic.
    
    $\DIFF_X\MOD$ is by definition the $\infty$-category of
    modules in $\QCOH\,X$ over the monad corresponding to 
    $\DIFF_X \in \QCOH(X\times X)$,
    so $\DIFF_X\MOD \rightleftarrows \QCOH\,X$ is monadic.
    
    For $\CRYS^L\,X$, by \cite[Prop 4.7.0.3]{Lurie-HA}, 
    it suffices to show that 
    \begin{itemize}
      \item $\OBLV^L$ is conservative
      \item $\CRYS^L\,X$ has geometric realisations and $\OBLV^L$ preserves them.
    \end{itemize}
    The second point is taken care of by 
    the fact that $\QCOH^* : \PSTK \to (\DGCAT_\CTS)^\OP$.
    For the first point,
    since $X$ is a laft smooth classical scheme,
    \linkto{crys.sm_implies_cfsm}{it is classically formally smooth}.
    This \linkto{crys.cfsm}{implies} the equivalence 
    between left crystals and infinitesimally equivariant quasi-coherent sheaves,
    as well as the equivalence between right crsytals and 
    infinitesimally equivariant ind-coherent sheaves :
    \begin{cd}
      {\mathrm{Crys}^L\,X} & {\varprojlim \mathrm{QCoh}(\check{C}(X/X_\mathrm{dR}))} \\
      {\mathrm{Crys}^R\,X} & {\varprojlim \mathrm{IndCoh}(\check{C}(X/X_\mathrm{dR}))}
      \arrow["\sim", from=1-1, to=1-2]
      \arrow["\sim", from=2-1, to=2-2]
    \end{cd}
    Under the \linkto{crys.leftRight}{equivalence of left and right crystals}
    and the equivalence $\Upsilon_X : \QCOH\,X \simeq \INDCOH\,X$ due to
    smoothness of $X$,
    we obtain $\OBLV^L \simeq \Upsilon_X^{-1} \OBLV^R \Upsilon_{X_\DR}$.
    Since $\OBLV^R$ is conservative, we obtain the same for $\OBLV^L$.

    So the two $\infty$-categories $\CRYS^L\,X$ and 
    $\DIFF_X\MOD$ are determined by their monads.

    \item Under the \linkto{fm.qcoh}{integral transform equivalence}
    $\QCOH (X\times X) \simeq \DGCAT_\CTS(\QCOH\,X , \QCOH\,X)$,
    algebra objects on the left correspond to 
    algebra objects on the right, i.e. monads.
    Therefore the monads $\OBLV^L\,\INDUCE^L$ and $\DIFF_X \otimes \_$
    correspond to quasi-coherent sheaves on $X \times X$.

    \item Of course, $\DIFF_X \otimes \_$ corresponds to 
    $\DIFF_X$ on $X \times X$.

    \item
    Let $\s{D}^L_X \in \QCOH(X\times X)$ correspond to $\OBLV^L\,\INDUCE^L$.
    \cite[Prop 5.3.6]{Crys} shows 
    \[ 
      \s{D}^L_X \simeq (\om_X \boxtimes \OO_X) \otimes 
      \mathrm{Fiber}(\OO_{X\times X} \to j_* j^* \OO_{X\times X})
    \]
    lies $\QCOH(X\times X)^\heartsuit$,
    with $j : X \times X \minus \De_X \to X \times X$.

    \item
    In \cite[Section 5.4.1]{Crys}, we have
    for any $\FF, \GG \in (\QCOH\,X)^\heartsuit$ and 
    $\QQ \in \QCOH(X \times X)^\heartsuit$ 
    with set-theoretically supported on the diagonal,
    % \begin{align*}
    %   (p_2)_* (p_1^! \FF \otimes \QQ) \to \GG \,\,\,\,\,\,
    %   \leftrightsquigarrow\,\,\,\,\,\,\,\, & p_1^! \FF \otimes \QQ \to p_2^! \GG \\
    %   \leftrightsquigarrow\,\,\,\,\,\,\,\, & \QQ \to \underline{\HOM}(p_1^! \FF , p_2^!\GG) \\
    %   \leftrightsquigarrow\,\,\,\,\,\,\,\, & \QQ \to \DIFF_X(\FF , \GG)
    % \end{align*}
    \begin{align*}
      (p_2)_* (p_1^* \FF \otimes \QQ) \to \GG \,\,\,\,\,\,
      \leftrightsquigarrow \,\,\,\,\,\, \QQ \to \DIFF_X(\FF , \GG)
    \end{align*}
    Applying to $\FF = \GG = \OO_X, \QQ = \s{D}^L_X$,
    we obtain a morphism $\s{D}^L_X \to \DIFF_X$.
    This is a morphism of algebra objects on $\QCOH(X\times X)$.

    \item 
    As in \cite[Section 5.4.3]{Crys}, 
    $\s{D}^L_X \map{\sim}{} \DIFF_X$ 
    is a classical computation by using the fact that
    when $X$ is smooth proper Noetherian classical over $k$,
    $\om_X$ has an explicit description as shifted top forms.
    \cite[Lem 48.15.7]{stacks}

  \end{enumerate}

  % (1) 
  % % Recall we have the following commuting diagram : 
  % % \begin{cd}
  % %   {\mathrm{Crys}^L\,X} & {\mathrm{Crys}^R \, X} \\
  % %   {\mathrm{QCoh}\,X} & {\mathrm{IndCoh}\,X}
  % %   \arrow["{\Upsilon_X}", from=2-2, to=2-1]
  % %   \arrow["{\mathrm{oblv}^L}"', from=1-1, to=2-1]
  % %   \arrow["{\mathrm{oblv}^R}", from=1-2, to=2-2]
  % %   \arrow["{\Upsilon_{X_\mathrm{dR}}}"', from=1-2, to=1-1]
  % % \end{cd}
  % % where the bottom morphism is an equivalence since
  % % $X$ is smooth and classical,
  % % the top morphism is the equivalence between left and right crystals.
  % % So it suffices to show that $\CRYS^R\,X$ and $\OBLV^R$ has 
  % % the deserved properties.

  % % By definition, $\INDCOH^! : \PSTK_\LAFT \to (\DGCAT_\CTS)^\OP$ so 
  % % $\CRYS^R\,X$ has all small colimits and $\OBLV^R$ preserves them.
  % % This applies in particular to geometric realisations.
  % % Now for conservativity of $\OBLV^R$.
  % % Since $X$ is a smooth classical scheme of locally almost finite type,
  % % it is classically formally smooth and hence
  % % we have the equivalence 
  % % \begin{cd}
  % %   {\mathrm{Crys}^R\,X} & {\varprojlim \mathrm{IndCoh}(\check{C}(X/X_\mathrm{dR}))}
  % %   \arrow["\sim", from=1-1, to=1-2]
  % % \end{cd}

  % (2) 

  % GR2 Ch2 1.6.11 - definition of ind-proper morphism.

  % (3)

  % (4)

  % (5)

  % (6)

\end{proof}

\end{document}