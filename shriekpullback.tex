\documentclass[./main.tex]{subfiles}
\begin{document}

We describe what $\INDCOH$ does to morphisms in $\DSCH_\AFT$.
Pushforward for $\INDCOH$ is easy to describe.
It follows from the definition of the t-structure of $\INDCOH$.

\begin{prop}[Pushforward for $\INDCOH$ on $\DSCH_\AFT$]

  Let $f : X \to Y$ be in $\DSCH_\AFT$.
  Then there exists a unique 
  $f_*^\INDCOH \in \DGCAT_\CTS(\INDCOH\,X , \INDCOH\,Y)$
  that is compatible with $\Psi$ meaning the following commutes :
  \begin{cd}
    {\INDCOH\,X} & {\QCOH\,X} \\
    {\INDCOH\,Y} & {\QCOH\,Y}
    \arrow["{\Psi_X}", from=1-1, to=1-2]
    \arrow["{\Psi_Y}", from=2-1, to=2-2]
    \arrow["{f_*^\INDCOH}"', from=1-1, to=2-1]
    \arrow["{f_*^\QCOH}", from=1-2, to=2-2]
  \end{cd}
\end{prop}
\begin{proof} This is a sketch of \cite[Ch 4 , 2.1.2]{GR1}.
  
  This crucially uses the fact that $\Psi$ gives an equivalence
  $(\INDCOH\,X)^+ \map{\sim}{} (\QCOH\,X)^+$.
  Using the t-structure of $\INDCOH\,X$ and the
  \linkto{dgcat.ind.up}{universal property of ind-completions}
  in order to give $f_*^{\INDCOH}$
  it suffices to show that $f_*^{\QCOH} \Psi_X$ maps
  $(\INDCOH\,X)^+$ into $(\QCOH\,Y)^+$.
  This comes down to the fact that $f_*^{\QCOH}$ is a right derived functor
  and hence sends $(\QCOH\,X)^+$ to $(\QCOH\,Y)^+$.
\end{proof}

In \cite[Ch 4, 2.2]{GR1}, functoriality of pushforward is shown.
\begin{prop}
  
  There exists a functor \[
    \INDCOH_* : \DSCH_\AFT \to \DGCAT_\CTS  
  \]
  equipped with a natural transformation \[
    \Psi : \INDCOH_* \to \QCOH_*  
  \]
  such that on objects and morphisms it yields \[
    \INDCOH^*(X) = \INDCOH(X) \,\,\,\,\, 
    f : X \to Y \rightsquigarrow  f_* : \INDCOH\,X \to \INDCOH\,Y
  \]
  \cite[Ch 4, 2.2.3]{GR1}
\end{prop} 

We now describe pullback for $\INDCOH$.
This is more delicate because it needs to achieve two things : 
\begin{itemize}
  \item for proper $f : X \to Y$ in $\DSCH_\AFT$,
  we want to have an adjunction \[
    f_* \dashv f^! : \INDCOH\,X \rightleftarrows \INDCOH\,Y
  \]
  inside $\DGCAT_\CTS$.
  \item for an open embedding $j : U \to X$ in $\DSCH_\AFT$,
  we want to have an adjunction \[
    j^! \dashv j_* : \INDCOH\,U \rightleftarrows \INDCOH\,X  
  \]
\end{itemize}
Let us first show how this pullback functors
exist in the individual cases.

\begin{prop}[Pullback for $\INDCOH$ along Proper Morphisms]
  \link{indcoh.proper}
  
  Let $f : X \to Y$ be a proper morphism in $\DSCH_\AFT$.
  Then $f_* : \INDCOH\,X \to \INDCOH\,Y$ maps
  $\COH\,X$ into $\COH\,Y$.
  Hence we have an adjunction \[
    f_* \dashv f^! : \INDCOH\,X \rightleftarrows \INDCOH\,Y  
  \]
  where $f^! \in \DGCAT_\CTS(\INDCOH\,Y , \INDCOH\,X)$.

\end{prop}
\begin{proof}
  This is a sketch of \cite[Ch 4, 5.1.4]{GR1}.
  By construction of $f_*$ for $\INDCOH$,
  it suffice to show the result for $f_* : \QCOH\,X \to \QCOH\,Y$.
  By a standard argument with t-structures,
  it suffices to show $f_* (\COH\,X)^\heartsuit \subs \COH\,Y$.
  We try to the case of $X, Y$ both classical.
  First, notice that the inclusion $i : X^\CL \to X$ induces an equivalence
  $(\COH\,X)^\heartsuit \simeq (\COH\,X)^\heartsuit$.
  Combining this with the fact that the inclusion
  $Y^\CL \to Y$ maps $\COH\,Y^\CL \to \COH\,Y$,
  it suffices to prove the result for $X^\CL \to Y^\CL$.
  We are now in the classical case \cite[Prop 30.19.1]{stacks}.
  This proves $f^* \COH\,X \subs \COH\,Y$.

  The fact that we have an adjunction $f_* \dashv f^!$
  is simply an application of the \linkto{dgcat.adjoint}{adjoint functor theorem}
  since $f_*$ is by definition in $\DGCAT_\CTS$.
  The content of this proposition is that
  since \linkto{indcoh.compact}{$\COH\,X \sim (\INDCOH\,X)^c$}
  and $f_*$ preserves compact objects,
  \linkto{dgcat.radj.cts}{it follows} that $f^!$ is continuous.
\end{proof}

\begin{prop}[Proper Base Change for $\INDCOH$]
  \link{indcoh.proper_bc}

  Suppose we have the following cartesian square in $\DSCH_\AFT$ : 
  \begin{cd}
    W & Y \\
    X & Z
    \arrow["v"{description}, from=1-1, to=2-1]
    \arrow["g"{description}, from=1-2, to=2-2]
    \arrow["f"{description}, from=2-1, to=2-2]
    \arrow["u"{description}, from=1-1, to=1-2]
    \arrow["\lrcorner"{anchor=center, pos=0.125}, draw=none, from=1-1, to=2-2]
  \end{cd}
  where $g$ and $v$ are proper.
  Then the dashed morphism below obtained from $g_* u_* \simeq f_* v_*$
  is an equivalence : 
  \begin{cd}
    {u_* v^!} & {g^! f_*} \\
    {g^!g_*u_*v^!} & {g^!f_*v_*u^!}
    \arrow[from=1-1, to=2-1]
    \arrow["\sim"', from=2-1, to=2-2]
    \arrow[from=2-2, to=1-2]
    \arrow[dashed, from=1-1, to=1-2]
  \end{cd}
\end{prop}
\begin{proof}
  This is \cite[Ch 4, 5.2.2]{GR1}.
  The proof follows basically from
  the compatibility of the continuous right adjoint $f^!$ for $\INDCOH$
  and the \emph{not necessarily continuous} right adjoint $f^!$ for $\QCOH$
  under \link{indcoh.t}{the equivalence} 
  $\Psi : (\INDCOH\,\_)^+ \simeq (\INDCOH\,,\_)^+$,
  the adjunction $f_* \dashv f^! : \QCOH\,X \rightleftarrows \QCOH\,Y$
  from classical algebraic geometry,
  and finally base change for $\QCOH$ along quasi-compact schematic morphisms.
  \cite[Ch 3, 2.2.2]{GR1}
\end{proof}

\begin{prop}[Pullback for $\INDCOH$ across Open Embeddings]
  \link{indcoh.open_emb}
  
  Let $j : U \to X$ be an open embedding.
  Then we have an adjunction \[
    j^! \dashv j_* : \INDCOH\,U \rightleftarrows \INDCOH\,X  
  \]
  where $j_*$ is fully faithful.
\end{prop}
\begin{proof}
  
  This is a shortened version of \cite[Ch 4, 3]{GR1} applied to
  the special case of open embeddings.
  First note that $j^* : \QCOH\,X \to \QCOH\,U$
  maps $\COH\,X$ into $(\QCOH\,U)^+ \simeq (\INDCOH\,U)^+ \subs \INDCOH\,U$.
  So by the \linkto{dgcat.ind.up}{universal property of ind-completions},
  we obtain $j^* \in \DGCAT_\CTS(\INDCOH\,X , \INDCOH\,U)$.
  We claim that $j^! := j^*$ works.

  To show $j^! \dashv j_*$ on $\INDCOH$,
  since both are continuous,
  it suffices to check on the full subcategory of compact objects.
  \linkto{indcoh.compact}{This is $\COH\,X$, respectively $\COH\,U$}.
  The adjunction now follows from 
  $j^* \dashv j_* : \QCOH\,X \rightleftarrows \QCOH\,U$.

  The fully faithfulness of $j_*$,
  equivalently the fact that $j^*j_* \to \id{}$ is an equivalence
  is checked in the same way : on compact objects.
\end{proof}

Now, the idea for $f^!$ for a general $f : X \to Y$ in $\DSCH_\AFT$
is to use Nagata's compactification theorem
\cite[Thm 38.33.8]{stacks} to factor $f = p j$ where
$p$ is proper and $j$ is an open embedding and define \[
  f^! := p^! j^!  
\]
In \cite[Ch 5 , 2.1]{GR1},
it is then shown that the $\infty$-category $\FACTOR(f)$ of
factorings of $f$ into an open embedding followed by a proper morphism
is contractible and that
this gives functoriality for !-pullback.
Unfortunately, the proof is much too complicated so
we will omit the proof.

\begin{prop}[Functoriality of !-Pullback]
  
  There exists a functor $\INDCOH^! : \DSCH_\AFT \to (\DGCAT_\CTS)^\OP$
  such that for proper $f$ in $\DSCH_\AFT$,
  the image of $f$ recovers $f^!$ and for open embeddings $j$ in $\DSCH_\AFT$,
  the image of $j$ recovers $j^!$.
  \cite[Ch 5 , 3.1.4]{GR1}
\end{prop}

Finally, we obtain ind-coherent sheaves on all laft prestacks.

\begin{dfn}[Ind-coherent sheaves on laft prestacks]
  
  We define $\INDCOH^! : \PSTK_\LAFT \to (\DGCAT_\CTS)^\OP$
  as the left Kan extension of 
  $\INDCOH^! : \DAFF^{<\infty}_\FT \to (\DGCAT_\CTS)^\OP$
  via the \linkto{dgcat.psh.up}{
    universal property of presheaf $\infty$-categories
  }.
  \cite[Ch 5 , 3.4.1]{GR1}
\end{dfn}

\end{document}