\documentclass[./main.tex]{subfiles}
\begin{document}
  
We will use ``infinity category'' to refer only to 
$(\infty,1)$-categories.
The word ``category'' will exclusively refer to $1$-categories.

$(\infty,1)\dash\CAT$ denotes the infinity category of 
small infinity categories. 
$(\infty,1)\dash\CAT$ has all small limits \cite[Cor 7.4.1.11]{Kerodon}
and small colimits \cite[Cor 7.4.3.13]{Kerodon} and
is cartesian closed.
We use $\FUN(C,D)$ to denote the infinity category of functors from
$C$ to $D$.

There is an adjunction 
$h \dashv N : (\infty,1)\dash\CAT \rightleftarrows 1\dash\CAT$.
$N$ is called the \emph{nerve functor} and it is fully faithful,
allowing us to see 1-categories as $\infty$-categories.
In particular, we use $\De^n$ to denote the \emph{$n$-simplex},
the $\infty$-category obtained from the linear order
$[n] = \set{0 \leq \cdots \leq n}$.
Given an infinity category $C$,
objects of $C$ are the same as functors $\De^0 \to C$
and morphisms in $C$ are the same as functors $\De^1 \to C$.

There is a full subcategory $\SPC$ of $(\infty,1)\dash\CAT$ consisting
of $\infty$-categories $X$ where all morphisms are isomorphisms.
These are called \emph{$\infty$-groupoids} but also \emph{spaces}
by homotopy theorists.
The infinity category $\SPC$ plays the role of $\SET$ in 1-category theory
in the sense that given an infinity category $C$,
the fiber of $\FUN(\De^1 , C) \to \FUN(\partial\De^1 , C)$
over $(X,Y)$, denoted $C(X,Y)$, is in fact a space.
The nerve functor $N$ lands $\SET$ inside $\SPC$,
where given a set $S$ and two points $x, y \in S$,
we have $NS(x,y) \simeq \nothing$ the empty space.
In other words, sets are ``discrete spaces''.

For any infinity category $C$, 
we define $\PSH\,C := \FUN(C\op,\SPC)$ and refer to its objects as
\emph{presheaves in $C$}.
Infinity categories of presheaves are significant since
we will be working with derived algebraic geometry functorially.
We will use the following universal property of $\PSH\,C$ many times.

\link{dgcat.psh.up}
\begin{prop}[Universal Property of Presheaf $\infty$-Categories]

  Let $S$ be a small $\infty$-category.
  \begin{itemize}
    \item There is a fully faithful functor $S \to \PSH\,S$
    which takes each object $x$ in $S$ to the functor 
    $S(\_ , x) : C\op \to \SPC$ taking points $y$ to $S(y,x)$.
    This is called the \emph{Yoneda embedding}.
    \item \cite[Prop 5.1.2.3]{Lurie-HTT}
    $\PSH\,S$ has small colimits and small limits.
    In fact, they are computed pointwise.
    \item \cite[Prop 5.1.5.6, 5.2.6.5]{Lurie-HTT}
    For $C$ be an $\infty$-category with small colimits, 
    let $\mathrm{Fun}^L(\PSH\, S , C)$ denote the
    full subcategory of $\mathrm{Fun}(\PSH\, S, C)$ consisting of
    functors preserving small colimits.
    Then restricting along the Yoneda embedding $S \to \PSH\, S$
    gives an equivalence of $\infty$-categories : 
    \[
      \mathrm{Fun}^L(\PSH\, S , C) \map{\sim}{} \mathrm{Fun}(S , C)  
    \]
    An inverse functor is given by left Kan extension.
    In particular,
    for $u_! \in \mathrm{Fun}^L(\PSH\, S , C)$ corresponding to 
    $u \in \mathrm{Fun}(S , C)$,
    we have for every $X \in \PSH\, S$ that
    $u_!$ exhibits $u_!(X)$ as the colimit of the diagram 
    $S_{/ X} \to S \to C$.
    Furthermore, if we assume $C$ is locally small,
    then we have an adjunction 
    \[
      u_! \dashv u^* : \PSH\,S \rightleftarrows C
    \]
    where $u^*$ is given by the composition 
    \[
      C \map{\text{Yoneda}}{} \PSH\,C = \FUN(C,\SPC^\OP) 
      \map{\text{restrict along $u$}}{} \FUN(S,\SPC^\OP) = \PSH\,S
    \]
  \end{itemize}
\end{prop}

An immediate consequence of the above is the following,
which is key for defining de Rham spaces.

\link{dgcat.psh.triple}
\begin{lem}[Left and Right Kan Extensions of Presheaves]
  
  Let $u : S \to T$ be a functor between small $\infty$-categories.
  Then we have a triple of adjoints : 
  \begin{cd}
    {\mathrm{PSh}\,S} & {} & {\mathrm{PSh}\,T}
    \arrow["{u^*}"{description}, from=1-3, to=1-1]
    \arrow["{\mathrm{u_!}}", shift left=5, from=1-1, to=1-3]
    \arrow["{u_*}"', shift right=5, from=1-1, to=1-3]
    \arrow["\bot"{description}, shift left=3, draw=none, from=1-1, to=1-3]
    \arrow["\bot"{description}, shift right=3, draw=none, from=1-1, to=1-3]
  \end{cd}
  where 
  \begin{itemize}
    \item $u_!$ is the left Kan extension of $S \to T \to \PSH\,T$
    \item $u^*$ is the composition $\PSH\,T \to \PSH\,\PSH\,T \to \PSH\,S$
    \item $u_*$ is the composition $\PSH\,S \to \PSH\,\PSH\,S \to \PSH\,T$
  \end{itemize}
  In particular, for $X \in \PSH\,S$,
  $u_!(X)$ is the left Kan extension of $X$ along $S \to T$
  and $u_*(X)$ is the right Kan extension of $X$ along $S \to T$.
  \begin{proof1}
    For $u_! \dashv u^*$, we apply the universal property of
    presheaf $\infty$-categories to $S \to T \to \PSH\,T$.
    Now for $u^* \dashv u_*$, note that
    since colimits in $\PSH\,T$ are computed pointwise,
    we have that $u^*$ preserves small colimits.
    This means we can apply the universal property of 
    presheaf $\infty$-categories again,
    but this time to the composition $T \to \PSH\,T \to \PSH\,S$.
    This gives $u^* \dashv u_*$. 
  \end{proof1}
\end{lem}

\end{document}