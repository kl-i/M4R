\documentclass[./main.tex]{subfiles}
\begin{document}
  
\begin{dfn}[Covariant $\QCOH$]

  There is a functor $\QCOH^* : \DAFF \to \DGCAT_{cts}^\OP$
  that assigns to each $A \in \DAFF$ the 
  \emph{derived category of $A$-modules}, denoted $A\dash\MOD$.

  For $A$ discrete (i.e. a commutative ring),
  there is the following description of $A\dash\MOD$
  under the quasi-category model of $\infty$-categories
  summarised in a single diagram : 
  \begin{cd}
    {N(\mathrm{Ch}\,A)} \\
    {N_{dg}(\mathrm{Ch}\,A)} & 
      {N_{dg}((\mathrm{Ch}\,A)_f)} & {A\text{-}\mathrm{Mod}} \\
    & {D(A)}
    \arrow["\subseteq"', from=1-1, to=2-1]
    \arrow["L", shift left=2, from=2-1, to=2-2]
    \arrow["\supseteq", shift left=2, from=2-2, to=2-1]
    \arrow["\bot"{description}, draw=none, from=2-1, to=2-2]
    \arrow["{W^{-1}}", shift left=3, from=1-1, to=2-2]
    \arrow["{=:}"{description}, draw=none, from=2-2, to=2-3]
    \arrow["h", from=2-2, to=3-2]
  \end{cd}
  Details : 
  \begin{itemize}
    \item $\CH\,A$ is the honest-to-god dg-category of chain complexes of
    honest-to-god $A$-modules
    and $D(A)$ is the category of complexes of injectives 
    \item $\CH\,A$ has a model structure such that 
    cofibrations are degree-wise injections and 
    weak equivalences are quasi-isomorphisms (Lurie HA 1.3.5.3).
    Although the class of fibrations are defined abstractly as
    those satisfying right lifting with respect to acyclic cofibrations,
    it turns out that any fibrant complex must be degree-wise injective,
    and partially conversely,
    any bounded above complex of injectives is fibrant 
    (Lurie HA 1.3.5.6).
    $(\CH\,A)_f$ denotes the full subcategory of fibrant complexes.
    \item $N$ denotes the nerve functor which converts
    1-categories to simplicial sets, which have the property of being
    $\infty$-categories.
    $N_{dg}$ denotes the dg-nerve functor which achieves the same thing for
    honest-to-god dg-catgeory categories. 
    (See Kerodon 2.5.3 for a construction.)
    \item $h$ is the truncation of an infinity category to a 1-category
    by taking its homotopy catgeory.
    It is the left adjoint to $N$.
    (See Kerodon 1.2.5 for a construction.)

    We have that the homotopy category of $A\dash\MOD$ gives the
    usual derived category of $A$-modules, as in classical algebraic geometry.
    \item $L$ is a left adjoint to the inclusion 
    $N_{dg}((\CH\,A)_f) \subs N_{dg}(\CH\,A)$.
    Intuitively, for every complex $M_\bullet$, 
    there exists a acyclic cofibration $M_\bullet \to I_\bullet$ 
    to fibrant $I_\bullet$
    and this is initial in the category of arrows from 
    $M_\bullet$ into $N_{dg}((\CH\,A)_f)$
    (Lurie HA 1.3.5.12).
    This means for each $M_\bullet$, 
    such a morphism $M_\bullet \to I_\bullet$ is unique up to equivalence
    and assembles to the desired functor $L$.
    Practically speaking, $L(M_\bullet) \simeq I_\bullet$.
    \item The composition $N(\CH\, A) \to N_{dg}((\CH\, A)_f)$
    exhibits the latter as the $\infty$-categorical localisation 
    of the former at quasi-isomorphisms (Lurie HA 1.3.5.15).
    This matches the standard treatment in classical algebraic geometry : 
    the localisation functor from $\CH\, A$ to $D(A)$
    takes a complex and resolves it by injecting it
    quasi-isomorphically into a complex of injectives.
  \end{itemize}

\end{dfn}

\begin{itemize}
  \item $\DAFF^\OP$ is localisation of commutative dg-algebras w.r.t.
  suitable model structure. 
  (Alternatively, localisation of simplicial commutative rings w.r.t. 
  suitable model structure. However, then need to show
  is equivalent to commutative algebra objects in $\VEC$.)
  \item for $\SPEC\,A \in \DAFF$, $A\dash\MOD$ is
  $\infty$-category of left modules over $A$ in $\VEC$.
  Can be realised as localisation of left modules over
  $A$ as a commutative dg-algebra w.r.t. some model structure.

  \item $\QCOH^* : \DAFF \to \DGCAT_\CTS^\OP$ works
  and for each $f : \SPEC\,B \to \SPEC\,A$,
  $f^* : \QCOH\,A \to \QCOH\,B$ can be realised as
  $A \otimes^L_k \_$.

  \item for $\SPEC\,A \in \DAFF$, $\QCOH\,A$ has obvious t-structure.
  
\end{itemize}

\begin{dfn}[Quasi-coherent Sheaves on Prestacks]
  
  We define $\QCOH^* : \PSTK \to \DGCAT_\CTS^\OP$
  as the left Kan extension of $\QCOH^* : \DAFF \to \DGCAT_\CTS^\OP$.
\end{dfn}

\begin{rmk}
  
\end{rmk}


\end{document}