\documentclass[./main.tex]{subfiles}
\begin{document}

The theory of ind-coherent sheaves does not stand alone from
the theory of quasi-coherent sheaves.
So before starting with ind-coherent sheaves,
we briefly gather some basic facts about quasi-coherent sheaves.
Specifically, how they are set up in derived algebraic geometry.
  
\begin{prop}[Contravariant $\QCOH$]

  There is a functor $\QCOH^* : \DAFF \to (\infty,1)\dash\CAT^\OP$
  that assigns to each $A \in \DAFF$ the 
  \emph{derived category of $A$-modules}, denoted $A\dash\MOD$.

  For $\SPEC\,A \in \DAFF$, $A\dash\MOD$ is
  $\infty$-category of left modules over $A$ in $\VEC$.
  It can be realised as localisation of left modules over
  $A$ as a commutative dg-algebra w.r.t. some model structure.
  $A\dash\MOD = \QCOH\,A$ is a cocomplete and complete stable
  infinity category.

  For $f : \SPEC\,B \to \SPEC\,A$,
  $f^* : \QCOH\,A \to \QCOH\,B$ can be realised as
  derived tensoring $A \lotimes_k \_$.
\end{prop}

\begin{rmk}[Concrete description of $A\dash\MOD$]

  For $A$ discrete (i.e. a commutative ring),
  there is the following description of $A\dash\MOD$
  under the quasi-category model of $\infty$-categories
  summarised in a single diagram : 
  \begin{cd}
    {N(\mathrm{Ch}\,A)} \\
    {N_{dg}(\mathrm{Ch}\,A)} & 
      {N_{dg}((\mathrm{Ch}\,A)_f)} & {A\text{-}\mathrm{Mod}} \\
    & {D(A)}
    \arrow["\subseteq"', from=1-1, to=2-1]
    \arrow["L", shift left=2, from=2-1, to=2-2]
    \arrow["\supseteq", shift left=2, from=2-2, to=2-1]
    \arrow["\bot"{description}, draw=none, from=2-1, to=2-2]
    \arrow["{W^{-1}}", shift left=3, from=1-1, to=2-2]
    \arrow["{=:}"{description}, draw=none, from=2-2, to=2-3]
    \arrow["h", from=2-2, to=3-2]
  \end{cd}
  Explanation : 
  \begin{itemize}
    \item $\CH\,A$ is the honest-to-god dg-category of chain complexes of
    honest-to-god $A$-modules
    and $D(A)$ is the category of complexes of injectives 
    \item $\CH\,A$ has a model structure such that 
    cofibrations are degree-wise injections and 
    weak equivalences are quasi-isomorphisms.
    \cite[Prop 1.3.5.3]{Lurie-HA}
    Although the class of fibrations are defined abstractly as
    those satisfying right lifting with respect to acyclic cofibrations,
    it turns out that any fibrant complex must be degree-wise injective,
    and partially conversely,
    any bounded above complex of injectives is fibrant.
    \cite[Prop 1.3.5.6]{Lurie-HA}
    $(\CH\,A)_f$ denotes the full subcategory of fibrant complexes.
    \item $N$ denotes the nerve functor which converts
    1-categories to simplicial sets, which have the property of being
    $\infty$-categories.
    $N_{dg}$ denotes the dg-nerve functor which achieves the same thing for
    honest-to-god dg-catgeory categories. 
    (See \cite[Subsection 2.5.3]{Kerodon} for a construction.)

    We have that the homotopy category of $A\dash\MOD$ gives the
    usual derived category of $A$-modules, as in classical algebraic geometry.
    \item $L$ is a left adjoint to the inclusion 
    $N_{dg}((\CH\,A)_f) \subs N_{dg}(\CH\,A)$.
    Intuitively, for every complex $M_\bullet$, 
    there exists a acyclic cofibration $M_\bullet \to I_\bullet$ 
    to fibrant $I_\bullet$
    and this is initial in the category of arrows from 
    $M_\bullet$ into $N_{dg}((\CH\,A)_f)$.
    \cite[Prop 1.3.5.12]{Lurie-HA}
    This means for each $M_\bullet$, 
    such a morphism $M_\bullet \to I_\bullet$ is unique up to equivalence
    and assembles to the desired functor $L$.
    Practically speaking, $L(M_\bullet) \simeq I_\bullet$.
    \item The composition $N(\CH\, A) \to N_{dg}((\CH\, A)_f)$
    exhibits the latter as the $\infty$-categorical localisation 
    of the former at quasi-isomorphisms.
    \cite[Prop 1.3.5.15]{Lurie-HA}
    This matches the standard treatment in classical algebraic geometry : 
    the localisation functor from $\CH\, A$ to $D(A)$
    takes a complex and resolves it by injecting it
    quasi-isomorphically into a complex of injectives.
  \end{itemize}
\end{rmk}

\begin{rmk}
  For $\SPEC\,A \in \DAFF$, $\QCOH\,A$ has obvious t-structure.
\end{rmk}

\begin{prop}[Pullback is Symmetric Monoidal]
  \link{qcoh.symm_mon}
  
  The functor $\QCOH^* : \DAFF \to (\infty,1)\dash\CAT^\OP$
  admits a lifts to a functor $\QCOH^* : \DAFF \to 
  \brkt{\mathrm{cMon}\brkt{(\infty,1)\dash\CAT}}^\OP$,
  where the latter is the opposite of 
  the $\infty$-category of commutative monoid objects
  in $(\infty,1)\dash\CAT$, A.K.A. symmetric monoidal $\infty$-categories
  with symmetric monoidal functors.

  \cite[Prop 4.5.3.1]{Lurie-HA}
\end{prop}

\begin{rmk}
  
  We take for granted that $\QCOH^* : \DAFF \to (\infty,1)\dash\CAT^\OP$ 
  as mentioned so far
  can be lifted to $\QCOH^* : \DAFF \to \DGCAT_\CTS^\OP$.
  For individual $\QCOH\,A$,
  this is not hard to believe since the hom complexes of $\QCOH\,A$
  are left $A$-modules, and hence in particular,
  complexes of $k$-vector spaces.

  With that, 
  we are ready for quasi-coherent sheaves on all prestacks.
\end{rmk}

\begin{dfn}[Quasi-coherent Sheaves on Prestacks]
  
  We define $\QCOH^* : \PSTK \to \DGCAT_\CTS^\OP$
  as the left Kan extension of $\QCOH^* : \DAFF \to \DGCAT_\CTS^\OP$.
\end{dfn}

\begin{rmk}
  For $X \in \PSTK$, $\QCOH\,X$ has obvious t-structure
  that is checked pointwise. 
  We refer the reader to \cite[Ch 3, 1.5]{GR1}.

  For $X \in \DSCH$, 
  the heart of $\QCOH\,X$ gives the abelian 1-category
  of quasi-coherent sheaves.
\end{rmk}

There is also a covariant $\QCOH$.

\begin{prop}

  There exists a functor $\QCOH_* : \PSTK \to \DGCAT$
  which at the level of objects does the same as $\QCOH^*$
  but at the level of morphisms $f : X \to Y$,
  its image $f_*$ is obtained as the right adjoint to
  $f^*$ via the \linkto{dgcat.adjoint}{adjoint functor theorem}.
  \cite[Ch 3, 2.1.1]{GR1}
\end{prop}

\begin{rmk}
  It is important to note that $f_*$ is \emph{not} in general
  continuous.
\end{rmk}


\end{document}