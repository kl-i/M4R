\documentclass[./main.tex]{subfiles}

\begin{document}

\subsection{Why crystals?}

The goal is to provide a setting where 
representation theory can be interpreted in algebraic geometry.
Here, 
\begin{itemize}
  \item ``representation theory'' refers to the study of Lie algebras
  and their representations
  \item ``algebraic geometry'' means being able to pushforward and pullback 
  sheaves on spaces where 
  computations ultimately come down to commutative algebra.
\end{itemize}

One key example of something from representation theory we would like to 
interpret algebro-geometrically is Beilinson-Bernstein localisation.
This result played a central role in solving the Kazhdan-Lusztig conjectures
and is one of the founding results of modern representation theory.
Roughly speaking, 
this says that given a semi-simple Lie algebra $\f{g}$,
there is an equivalence \[
  L : \f{g}\MOD_0 \map{\sim}{} \DIFF_X\MOD
\]
where \begin{itemize}
  \item $\f{g}\MOD_0$ is the category of left modules over 
  the universal enveloping algebra $U(\f{g})$ of $\f{g}$ where
  the center of $U(\f{g})$ acts trivially.
  \item $\DIFF_X\MOD$ is the category of left $D$-modules on $X$ with
  $X$ being the flag variety of $\f{g}$.
\end{itemize}
We refer the reader to \cite*{Keller} for an introduction to 
Beilinson-Bernstein localisation.
A more detailed treatment with relations to Kazhdan-Lusztig conjectures
can be found in \cite[Ch 11, Ch 12]{H.T.T.}.

Let us describe what the Beilinson-Bernstein localisation functor looks like
in this new algebro-geometric framework.
Let $k$ be a field of characteristic zero
and $\bullet := \SPEC\,k$.
Firstly, we would have a correspondence between
Lie algebras over $k$ and ``formal groups'' over $k$.
\cite[Ch 7, 3.1.4]{GR2}
\[
  \LIE : \GRP(\FMOD / \bullet) \simeq \LIE\ALG(\QCOH\,k) : \exp
\]
\begin{itemize}
  \item $\FMOD / \bullet$ refers to \emph{formal moduli problems over $k$}
  \cite[Ch 2]{Lurie-DAGX}\cite[Ch 5, 1.1.]{GR2} and 
  group objects in there are what GR calls
  \emph{formal groups over $k$}.
  $\GRP(\FMOD / \bullet)$ denotes the category of these.
  \item For the full correspondence, we need to pass from Lie algebras to
  \emph{differential graded Lie algebras}.
  So $\QCOH\,k$ refers to the derived category of chain complexes of
  $k$-vector spaces and $\LIE(\QCOH\,k)$ refers to
  Lie algebra objects in there.
  \cite[2.1.5]{Lurie-DAGX}
\end{itemize}
Furthermore, the action of a $\f{g} \in \LIE\ALG(\QCOH\,k)$
on a scheme $X$ via vector fields should be equivalent to
an action of the formal group $\exp\,\f{g}$ on $X$.

Secondly, given a scheme $X$ equipped with an action from a formal group $G$,
we should be able to \emph{form the quotient $X$ by $G$},
which should be a morphism $X \to B_XG$.
In general, we should be able to \emph{quotient by formal groupoids $\GG$},
giving under object $X \to B_X \GG$ under $X$ such that
$\GG$ is recovered as $X \times_{B_X \GG} X$.
\cite[Ch 5 , 2.3.2]{GR2}
\[
  B_X : \FGRPD(X) \simeq \FMOD_{X /} : \text{ fiber product }
\]
\begin{itemize}
  \item $\FGRPD(X)$ refers to the category of formal groupoids over $X$.
  This includes $G \times X$ coming from the action of
  a formal group $G$ on $X$. \cite[Ch 5, 2.2.1]{GR2}
  \item $\FMOD_{X /}$ denotes the category of 
  \emph{formal moduli problems under $X$}. \cite[Ch 5, 1.3]{GR2}
\end{itemize}
As a corollary in the situation of $X = \bullet$,
we obtain an equivalence : 
\[
  B : \GRP(\FMOD / \bullet) \simeq \PT(\FMOD / \bullet) : \Om
\]
where $B$ takes a formal group $G$ to its \emph{classifying space $BG$}
and $\Om$ takes $s : \bullet \to Y$ to $\bullet \times_Y \bullet$.
Furthermore,
in the special case of $G = \exp\,\f{g} \in \GRP(\FMOD / \bullet)$
the formal group corresponding to a $\f{g} \in \LIE\ALG(\QCOH\,k)$,
we should have an equivalence : 
\begin{cd}
  {\QCOH(BG)} && {\mathfrak{g}\MOD(\QCOH\,k)} \\
	& {\QCOH\,k}
	\arrow["\sim", from=1-1, to=1-3]
	\arrow["{s^!}", shift left=3, from=1-1, to=2-2, shorten <=9pt]
	\arrow["{s_*}", shift left=2, from=2-2, to=1-1]
	\arrow["\top"{marking}, draw=none, from=2-2, to=1-1]
	\arrow["{\text{ind}}"{pos=0.3}, shift left=2, shorten >=10pt, from=2-2, to=1-3]
	\arrow["{\text{forget}}", shift left=2, shorten >=3pt, from=1-3, to=2-2]
	\arrow["\bot"{marking}, draw=none, from=2-2, to=1-3]
\end{cd}
which realises pullback $s^!$ along $s$ as the forgetful functor
$\f{g}\MOD \to \QCOH\,k$ and pushforward $s_*$ inducing representations
(from the trivial Lie algebra).

Thirdly, within the category of formal moduli problem under a scheme $X$,
there is a space called the \emph{de Rham space} $p_\DR : X \to X_\DR$.
The formal groupoid $X \times_{X_\DR} X$ over $X$ corresponding to
the de Rham space is called the \emph{infinitesimal groupoid}.
Intuitively, $p_\DR$ is the quotient of $X$ by all points that are 
``infinitesimally close'',
and thus sheaves on $X_\DR$ are supposed to be sheaves on $X$ equipped with
``infinitesimal equivariance''.
This idea leads to an equivalence under the case of smooth $X$ : 
\begin{cd}
  {\QCOH(X_\DR)} && {\DIFF_X\MOD} \\
	& {\QCOH\,X}
	\arrow["\sim", from=1-1, to=1-3]
	\arrow["{p_\DR^!}", shift left=3, from=1-1, to=2-2, shorten <=9pt]
	\arrow["{(p_\DR)_*}", shift left=2, from=2-2, to=1-1]
	\arrow["\top"{marking}, draw=none, from=2-2, to=1-1]
	\arrow["{\mathrm{ind}}"{pos=0.3}, shift left=2, shorten >=9pt, from=2-2, to=1-3]
	\arrow["{\text{oblv}}", shift left=2, shorten >=3pt, from=1-3, to=2-2]
	\arrow["\bot"{marking}, draw=none, from=2-2, to=1-3]
\end{cd}
which realises sheaves on $X_\DR$ as D-modules on $X$,
induction $\mathrm{ind}$ of D-modules as pushforward $(p_\DR)_*$ and
the forgetful functor $\mathrm{oblv}$ of D-modules as pullback $p_\DR^!$.
The category of sheaves on $X_\DR$ is called the category of 
\emph{(left) crystals on $X$}.

To put everything together, let $\f{g} \in \LIE\ALG(\QCOH\,k)$
and $X$ a scheme be equipped with an action from the formal group
$\exp\f{g}$ corresponding to $\f{g}$.
Then we obtain a \emph{correspondence of spaces} :
\begin{cd}
  {\exp\mathfrak{g}} & {(\exp\mathfrak{g}) \times X} & \rightsquigarrow & {B(\exp\mathfrak{g})} & {B_X(\exp\mathfrak{g})} \\
	\bullet & X & {X \times_{X_\DR} X} & {} & {X_\DR} \\
	&&&& {}
	\arrow[from=2-2, to=2-1]
	\arrow[from=1-1, to=2-1]
	\arrow[from=1-2, to=1-1]
	\arrow["{\text{action}}"', from=1-2, to=2-2]
	\arrow["p", from=1-5, to=2-5]
	\arrow["q"', from=1-5, to=1-4]
	\arrow[from=2-3, to=2-2]
	\arrow[from=1-2, to=2-3]
\end{cd}
Intuitively, $q$ comes from factoring 
$X \to \bullet \to B(\exp\f{g})$ through $X \to B_X(\exp\f{g})$
and $p$ comes from the morphism of formal groupoids
$(\exp\f{g}) \times X \to X \times_{X_\DR} X$ over $X$.
Finally, in the case of $\f{g}$ being a Lie algebra concentrated in degree zero
that is semi-simple,
and $X$ its flag variety,
we recover the Beilinson-Bernstein localisation functor as
pullback then pushforward across the above correspondence.
\[
  L \simeq p_* q^! : \f{g}\MOD \simeq \QCOH(B\exp\f{g}) \to 
  \QCOH(X_\DR) \simeq \DIFF_X\MOD
\]

The end goal of this paper is to understand just one part of this story : 
the equivalence between (left) crystals and (left) D-modules on
a smooth scheme $X$.

\subsection{Why stable infinity categories?}

Let us be clear that all of the categories of sheaves and categories
of modules mentioned in the previous section are 
\emph{stable infinity categories}.

We know from classical algebraic geometry that
when considering derived categories of sheaves,
we want to consider them as at least triangulated categories.
However, triangulated categories suffer some drawbacks : 
\begin{enumerate}
  \item cones are not functorial
  \item limits of triangulated categories are hard to deal with
\end{enumerate}
By passing to stable infinity categories, the situation is improved : 
\begin{itemize}
  \item Cones are colimits and thus can be made functorial,
  and easier to work with abstractly.
  \item Stable infinity categories are closed under limits.
  This is what will allow us to define sheaves on arbitrary prestacks
  rather simply and be able to argue with them abstractly.
  \item Analogous to the 
  theory of compactly generated triangulated categories and 
  colimit-preserving functors,
  there is a theory of compactly generated stable infinity categories
  and colimit-preserving functors which makes computations possible.
  This is used, for example, 
  in giving a working theory of \linkto{fm.qcoh.daff}{integral transforms}
  \item due to the existence of a \linkto{dgcat.dgcat.tensor}{tensor product},
  duality statements can be phrased nicely in terms of
  derived categories of sheaves being \emph{self-dual}.
  Examples include 
  \linkto{duality.naive}{self duality of $\QCOH\,X$} for quasi-compact schemes
  $X$ with affine diagonal and \linkto{duality.serre}{Serre duality}.
\end{itemize}

\subsection{Why derived algebraic geometry?}

There is another direction in which infinity categories are getting involved :
we require the use of \emph{derived algebraic geometry},
meaning that the basic building blocks of schemes are not
commutative rings but \emph{commutative differential graded algebras}.

A simple reason is that base change theorems break 
without flatness assumptions.
Consider the following example.
Let $A := k[t]/(t)^2$ where $k$ is a field and let
$\SPEC\,k \to \SPEC\,A$ be the closed embedding of $t = 0$.
Then the \emph{classical} fiber product gives the following.
\begin{cd}
  {\SPEC\,k} & {\SPEC\,k} & \rightsquigarrow & {\QCOH\,k} & {\QCOH\,k} \\
  {\SPEC\,k} & {\SPEC\,A} && {\QCOH\,k} & {\QCOH\,A}
  \arrow["i"', from=2-1, to=2-2]
  \arrow["{\id{}}"', from=1-1, to=2-1]
  \arrow["{\id{}}", from=1-1, to=1-2]
  \arrow["i", from=1-2, to=2-2]
  \arrow["\lrcorner"{anchor=center, pos=0.125}, draw=none, from=1-1, to=2-2]
  \arrow["{\id{}}"', from=1-4, to=2-4]
  \arrow["{\id{}}"', from=1-5, to=1-4]
  \arrow["{Li^*}", from=2-5, to=2-4]
  \arrow["{R i_*}", from=1-5, to=2-5]
  \arrow["\not\simeq"{description}, Rightarrow, from=1-4, to=2-5]
\end{cd}
Indeed the diagram on the right hand side does not commute up to isomorphism
since we have : 
\begin{cd}
  &&& \vdots & \vdots \\
  &&& {A \otimes_A k} & k \\
  {Li^*(Ri_* \,k)} & {Li^*(k)} & {k \otimes^L_A k} & {A \otimes_A k} & k & k
  \arrow["\simeq"{description}, draw=none, from=3-1, to=3-2]
  \arrow["\simeq"{description}, draw=none, from=3-2, to=3-3]
  \arrow["{t \otimes 1}"', from=2-4, to=3-4]
  \arrow["{t \otimes 1}"', from=1-4, to=2-4]
  \arrow["\simeq"{description}, draw=none, from=3-3, to=3-4]
  \arrow["0"', from=1-5, to=2-5]
  \arrow["0"', from=2-5, to=3-5]
  \arrow["\simeq"{description}, draw=none, from=3-4, to=3-5]
  \arrow["\not\simeq"{description}, draw=none, from=3-5, to=3-6]
\end{cd}
Going derived fixes this because we have the isomorphism 
\[
  \_ \otimes^L_A k \simeq \_ \otimes^L_k (k \otimes^L_A k)		
\]

Another reason is that the theory of formal moduli problems
mentioned in the first section also involves derived affine schemes,
rather than just affine schemes.
However, we will not get this in this paper.

\subsection{Why ind-coherent sheaves?}\label{why.indcoh}

Let us also be clear that it will not be enough to consider
quasi-coherent sheaves alone.
There are a few reasons for this : 
\begin{itemize}
  \item In the first section,
  we saw that many of adjunctions of representation theory
  under the interpreted using sheaves requires the pullback of sheaves to be
  the \emph{right adjoint} to pushforward,
  rather than the left adjoint as it usual is for quasi-coherent sheaves
  on schemes.

  \item As mentioned, 
  for computations with derived categories of sheaves
  as stable infinity categories, 
  it is desirable to stay in the world of
  compactly generated stable infinity categories and colimit-preserving functors.

\end{itemize}
The example of the closed embedding 
$i : \SPEC\,k \to \SPEC\,A = \SPEC\,k[t]/(t)^2$
again demonstrates both of the above points.
Let us explain.

We will show later that
\linkto{duality.daff}{$\QCOH\,k$ and $\QCOH\,A$ are both compactly generated},
meaning \linkto{dgcat.cg.out}
{they are the ind-completions of their full subcategory of 
compact objects}.
Now let us impose the above two points on
the derived pushforward $i_* : \QCOH\,k \to \QCOH\,A$,
i.e. let's assume that $i_*$ has a right adjoint $i^!$ that 
preserves small colimits.
Then \[
  \brkt{\QCOH\,A}(i_* V , M) \simeq \brkt{\QCOH\,k}(V , i^! M)
\]
for any $V \in \QCOH\,k , M \in \QCOH\,A$.
It follows that $i_*$ preserves compact objects
if and only if $i^!$ preserves small coproducts.
By the assumption,
we thus obtain that $i_* k$ must be compact as an object in $\QCOH\,A$.
However, it is a classical result that
for $A$ Noetherian, a module $M$ over $A$ is compact in $\QCOH\,A$
if and only if it is finitely generated and admits a
finite projective resolution.
This in particular implies that $\TOR_A^n(i_*k , i_*k) = 0$ for large enough
$n$.
But this is a contradiction since for all $n \geq 0$, \[
  \TOR_A^n(i_*k , i_*k) \simeq H^n(k \lotimes_A k) \simeq k \not\simeq 0
\]

The issue above can be seen as the failure of
$i_*$ to preserve compact objects.
The idea behind \emph{ind-coherent sheaves} is this : 
since $i$ is proper, $i_*$ sends $\COH\,k$ to $\COH\,A$ where
$\COH\,\_$ denotes the full subcategory $\QCOH\,\_$ of $\FF$
with finitely many non-zero cohomologies, all of which are coherent sheaves.
So we replace \begin{cd}
	{\QCOH\,k \simeq \IND\,\PERF\,k} & \rightsquigarrow & {\INDCOH\,k } & {\IND(\COH\,k)} \\
	{\QCOH\,A \simeq \IND\,\PERF\,A} && {\INDCOH\,A } & {\IND(\COH\,A)}
	\arrow["{i_*}"', from=1-1, to=2-1]
	\arrow["{:=}"{description}, draw=none, from=1-3, to=1-4]
	\arrow["{:=}"{description}, draw=none, from=2-3, to=2-4]
	\arrow["{i_*}"', from=1-3, to=2-3]
\end{cd}
where $\IND$ denotes \emph{ind-completion} and $i_*$ comes from the 
\linkto{dgcat.ind.up}{universal property of ind-completions}
and by definition preserves small colimits,
and hence has a right adjoint $i^!$
by \linkto{dgcat.adjoint}{adjoint functor theorem}.
(We will cover this later.)
Furthermore, this right adjoint also preserves small coproducts
because it can be shown that the full subcategory of 
compact objects of $\INDCOH\,k, \INDCOH\,A$ are $\COH\,k, \COH\,A$,
which are preserved by $i_*$ by construction.

Thus, a large part of the theory of crystals is
in developing the theory of ind-coherent sheaves together with 
pushforward, pullback, tensor, proper base change, etc for a 
large enough class of spaces including de Rham spaces, 
formal groups, quotients by formal groupoids.
Crystals on a scheme $X$ are then defined as 
\[
  \CRYS\,X := \INDCOH\,X_\DR
\]

\subsection{How crystals compare with usual D-modules?}

The classical theory of D-modules has its advantage in how explicit it is
and hence how indispensible it is for calculations in examples.
However, it is known that for singular varieties $Z$, 
even in the affine case the ring of differential operators can be
rather complicated.
\cite{BGG} gives an example where Noetherianness of 
the ring of differential operators fails.
One approach is via Kashiwara's theorem : 
find a closed embedding $Z \to X$ where $X$ is smooth
then define $D$-modules on $Z$ to be $D$-modules on $X$ which are
supported on $Z$.
One then has to do work to show that this is independent of
the closed embedding $Z \to X$.
One advantage of crystals is that 
they are intrinsic to the space, and 
Kashiwara's theorem can be deduced \cite[Ch 2, 2.5.6]{Crys}
rather than taken as a definition.

The main advantage of crystals over D-modules is 
in the perspective it brings : 
one can now think of D-modules as quasi-coherent / ind-coherent sheaves,
and hence operations to do with D-modules as usual operations
to do with quasi-coherent / ind-coherent sheaves.
This includes interpretation as Verdier duality for D-modules as 
self-duality of $\CRYS\,X := \INDCOH\,X_\DR$,
which places it on the same footing as \linkto{duality.serre}{Serre duality}.
Of course, it is also made with higher stacks in mind,
which makes it indispensible for the Geometric Langlands programme.

\subsection{Guide to Reading}

The structure of the paper is as follows : 
\begin{itemize}
  \item In section \ref{setting}
  we first explain how to work with ``dg categories'' in the sense of
  Gaitsgory--Rozenblyum, gathering all results needed to
  do reduces computations to the level of triangulated categories.
  Then we describe the types of spaces $\INDCOH$ will be made from,
  namely, \linkto{dsch.laft.char}{prestacks of locally almost finite type}.
  \item In section \ref{indcoh},
  we describe how ind-coherent sheaves work for derived schemes.
  We first take a look at the definition, 
  pushforward, pullback for open immersions,
  pullback for proper morphisms, and proper base change.
  Then we describe the theory of integral transforms
  for both quasi-coherent and ind-coherent sheaves,
  which is central to Gaitsgory--Rozenblyum's proof of
  the equivalence of (left) crystals and left D-modules
  for smooth classical schemes.
  On the way, we describe self-dualities of $\QCOH\,X$ and $\INDCOH\,X$
  on derived schemes of almost finite type
  and how they relate to the classical procedure of 
  taking duals of perfect complexes and Serre-duals of coherent sheaves.
  Finally, we describe a comparison $\Upsilon$ from
  $\QCOH$ to $\INDCOH$ which exist not only for derived schemes of
  almost finite type but all prestacks of locally almost finite type.
  \item In section \ref{crys},
  we first introduce de Rham spaces, left and right crystals.
  Then induction of crystals is obtained from
  Lurie's very general theory of descent and 
  the fact that $\INDCOH$ satisfies base change for
  \emph{ind-proper morphisms}.
  Finally, we prove the equivalence of left crystals and left D-modules.
\end{itemize}

\begin{rmk}
  
  The proofs of all of the results in this paper
  are well-known to the experts.
  However, I have tried to make proofs more explicit
  wherever possible by adding more details, 
  or slightly altering the argument so that things are 
  hopefully more digestable.
  I always indicate when I do this.

\end{rmk}

\end{document}