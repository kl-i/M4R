\documentclass[./main.tex]{subfiles}
\begin{document}
  
One of the main points of Gaitsgory-Rozenblyum is that
the assignment $X \rightsquigarrow \INDCOH\,X$ should be viewed
not as two functors $\INDCOH_* : \DSCH_\AFT \to \DGCAT_\CTS$ and
$\INDCOH^! : \DSCH_\AFT \to \DGCAT_\CTS$ but rather as a single
functor between \emph{$(\infty,2)$-categories} : 
\begin{cd}
  {\DSCH_\AFT} & {\CORR(\DSCH_\AFT)^\mathrm{proper}_\mathrm{all , all}} & {\DSCH_\AFT^\OP} \\
	& {\DGCAT_\CTS^{(\infty,2)}}
	\arrow["\INDCOH"{description}, from=1-2, to=2-2]
	\arrow["{\mathrm{vert}}", from=1-1, to=1-2]
	\arrow["{\INDCOH_*}"', from=1-1, to=2-2]
	\arrow["{\mathrm{hori}}"', from=1-3, to=1-2]
	\arrow["{\INDCOH^!}", from=1-3, to=2-2]
\end{cd}
In this section, 
we will explain informally how viewing $\INDCOH$ in this way
efficiently encodes many of the properties we wish an 
``$\infty$-category of sheaves'' to have.
We will then in the subsequent sections show how
each of the things we want actually computes and be content that
it can all be fit together into $\INDCOH$ 
with all higher-categorical coherencies accounted for.
The interested reader can find the formal construction of $\INDCOH$ in this way
in GR1 Ch5.

\begin{itemize}
  \item (Correspondences)
  $\CORR(\DSCH_\AFT)^\mathrm{proper}$ is an
  \emph{$(\infty,2)$-category}.
  The objects are the same as $\DSCH_\AFT$.
  Given $X , Y$ derived schemes of almost finite type, 
  there is an $(\infty,1)$-category of morphisms 
  \[
    \CORR^\mathrm{proper}(X,Y) \simeq 
    \brkt{\DSCH_\AFT / X \times Y}^\mathrm{proper}
  \]
  where the latter is the subcategory of $\DSCH_\AFT / X \times Y$
  consisting of all objects but the only morphisms are the ones
  that are proper, i.e.
  \begin{cd}
    Z & Y \\
    X & {Z_1}
    \arrow[from=1-1, to=2-1]
    \arrow[from=1-1, to=1-2]
    \arrow[from=2-2, to=1-2]
    \arrow[from=2-2, to=2-1]
    \arrow["{\mathrm{proper}}"{description}, from=1-1, to=2-2]
  \end{cd} 
  for $Z , Z_1 \in \DSCH_\AFT / X \times Y$.
  Given $Z \to X \times Y$, 
  we will refer to the projection into $X$ as the \emph{horizontal component}
  and the projection into $Y$ as the \emph{vertical component}.

  \item (Composition of Correspondences)
  Given two correspondences $X \mapfrom{}{} U \map{}{} Y$ and
  $Y \mapfrom{}{} V \map{}{} Z$,
  the composition is computed by using fiber products : 
  \begin{cd}
    X & U & W \\
    & Y & V \\
    && Z
    \arrow[from=1-2, to=1-1]
    \arrow[from=1-2, to=2-2]
    \arrow[from=2-3, to=2-2]
    \arrow[from=2-3, to=3-3]
    \arrow[from=1-3, to=1-2]
    \arrow[from=1-3, to=2-3]
    \arrow["\lrcorner"{anchor=center, pos=0.125, rotate=-90}, 
      draw=none, from=1-3, to=2-2]
  \end{cd}
  \item (Horizontal and Vertical Embeddings)
  The functor $\mathrm{vert} : \DSCH_\AFT \to \CORR(\DSCH_\AFT)^\PROPER$
  is identity on objects, and for $X , Y \in \DSCH_\AFT$,
  the induced functor \[
    \DSCH_\AFT(X , Y) \to 
    \CORR^\PROPER(X , Y) \simeq (\DSCH_\AFT / X \times Y)^\PROPER 
  \]
  puts $\id{X}$ in the horizontal component.

  The functor $\mathrm{hori}$ is similar, 
  except it puts $\id{Y}$ in the vertical component,
  hence the contravariance.

  \item $\DGCAT_\CTS^{(\infty,2)}$ is obtained from the $(\infty,1)$-category 
  $\DGCAT_\CTS$ by seeing it as an $(\infty,2)$-category whose only
  2-morphisms are equivalences.
  So $\INDCOH$ is a functor between $(\infty,2)$-categories.

  \item ($*$-Pushforward and $!$-Pullback)
  By restricting along $\mathrm{vert}$ respectively $\mathrm{hori}$,
  we obtain $\INDCOH_*$ respectively $\INDCOH^!$,
  which can be seen as functors between $(\infty,1)$-categories.
  For a 1-morphism $f$ in $\DSCH_\AFT$,
  we use $f_*$ and $f^!$ to denote the image of $f$ under
  $\INDCOH_*$ and $\INDCOH^!$ respectively.

  Let $(f , g) : Z \to X \times Y$ be a correspondence.
  We can compute its image under $\INDCOH$ as follows : 
  \begin{cd}
    X & Z & Z && {\INDCOH\,X} & {\INDCOH\,Z} \\
    & Z & Z & \rightsquigarrow && {\INDCOH\,Y} \\
    && Y
    \arrow["f", from=1-2, to=1-1]
    \arrow[from=1-2, to=2-2]
    \arrow[from=2-3, to=2-2]
    \arrow["g"', from=2-3, to=3-3]
    \arrow[from=1-3, to=1-2]
    \arrow[from=1-3, to=2-3]
    \arrow["\lrcorner"{anchor=center, pos=0.125, rotate=-90}, draw=none, from=1-3, to=2-2]
    \arrow["{\text{image of }(f,g)}"', from=1-5, to=2-6]
    \arrow["{f^!}", from=1-5, to=1-6]
    \arrow["{g_*}", from=1-6, to=2-6]
  \end{cd}
  Thus the image of $(f,g)$ is pull-push across the correspondence.

  \item (Proper Adjunction) 
  
  \item open adjunction
  \item proper base change
  \item symmetric monoidal structure on $\INDCOH^!$,
  hence internal hom
  \item $\INDCOH\,X$ self-dual
\end{itemize}


For intuition, it helps to think of the following linear algebra analogy : 
\begin{align*}
  &k \text{ base field} 
    & \VEC \\
  &V \text{ abelian group}
    & C \in (\infty,1)\dash\CAT^\EX_\CTS \\
  & \text{1-category of $k$-modules } k\MOD
    & \DGCAT_\CTS = \VEC\MOD \\
  & \HOM_k(V , W)
    & \DGCAT_\CTS(C , D) \\
  & V \otimes W 
    & C \otimes D \\
  & \bullet \text{ 1-category with single point, only identity}
    & \DAFF \\
  & \SET
    & \PSTK \\
  & k^\_ = \SET(\_,k) : \SET \to (k\MOD)^\OP
    & \QCOH^* : \PSTK_\LAFT \to (\DGCAT_\CTS)^\OP \\
\end{align*}



\end{document}