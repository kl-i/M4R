\documentclass{article}
\input{preamble.tex}

\addbibresource{refs.bib}

\begin{document}

\title{A Study in `A Study in Derived Algebraic Geometry' : 
Ind-Coherent Sheaves , Duality and Crystals }

\author{Ken Lee}
\date{2022}
\maketitle

\tableofcontents

Acknowledgements : 
\begin{itemize}
  \item Thanks Mum for these 22 years.
  \item Thanks Dad for getting into maths.
  \item Thanks to Ayano, my girlfriend for staying with me through hell
  and back.
  \item Thanks to my sister for pointing out when I'm being an idiot.
  \item Thanks to my supervisor Dr. Travis Schedler for 
  giving me the opportunity for this project.
  \item Thanks Haiping for 10 months-worth of diet coke.
\end{itemize}

Declaration : The following is my own work except otherwise stated.

\section{Introduction}

  \subfile{why.tex}

\section{Setting}\label{setting}

  \subsection{How to work with ``DG Categories''}\subfile{dgcat-intro.tex}

  \subsection{All you need to know about infinity categories}
    \subfile{inftycat.tex}

  \subsection{All you need know about stable infinity categories}
    \subfile{stable.tex}

  \subsection{
    All you need to know about
    compactly generated stable infinity categories
  }
    \subfile{cg.tex}

  \subsection{``DG Categories'' Finally}
    \subfile{dgcat.tex}

  \subsection{Derived Schemes}\subfile{dsch.tex}
% 1. DMod(X) = IndCoh(X_dR)
%    [Classically, nilpotent extensions were enough to consider.
%     What do we get from using all of dAff?]
% 2. LieAlg(IndCoh X) = Grp(FormMod / X)
%    [Q : this is where DAG is hiding?]
% 3. BB localization as B(exp g) <- B_X(exp g) -> X_dR

\section{IndCoherent Sheaves on Derived Schemes}\label{indcoh}

  \subsection{Quasi-Coherent Sheaves}\subfile{qcoh.tex}

  % \subsection{Why Ind-Coherent Sheaves?}\subfile{indcoh.tex}

  % \subsection{Correspondences and Six Functor Formalisms}\subfile{corr.tex}

  \subsection{Ind-Coherent Sheaves on a single Derived Scheme}
    \subfile{indCohdSch.tex}

  \subsection{Pushforward and Pullback across Various Morphisms}
    \subfile{shriekpullback.tex}

  % \subsection{Duality}
  %   \subfile{duality.tex}

  \subsection{Integral Transforms and Duality for Quasi-Coherent Sheaves}
    \subfile{fm-qcoh.tex}

  \subsection{Integral Transforms and Duality for Ind-Coherent Sheaves}
    \subfile{fm-indcoh.tex}

  \subsection{Relation between Quasi-Coherent Sheaves and Ind-Coherent Sheaves}
    \subfile{qcohindcoh.tex}
% Desired computations : 
% - !-pullback for a non-proper non-open morphism <- this is for abstract
% - example of proper basechange 
% - integral transform formula, at least in dAff case <- yey
% - the unit of !-tensor is really dualizing sheaf <- quote
% - the abstract duality functors match the classical computation <- naive ok.

\section{Crystals and D-modules}\label{crys}

  % \subsection{A note on inf-schemes}
  % Say something about making $\INDCOH$ base change
  % for larger class of spaces and morphisms
  % that includes $X \to X_\DR$.

  \subsection{Left and Right Crystals}\subfile{crys.tex}

  \subsection{Equivalence of Left and Right Crystals}\subfile{leftRightCrys.tex}

  \subsection{Induction of Left and Right Crystals}\subfile{induction.tex}

  \subsection{Equivalence of Left Crystals and D-modules}\subfile{dmodules.tex}

  % \subsection{Kashiwara's Lemma}\subfile{kashiwara.tex}
% Desired computations : 
% - the crystals monad actually corresponds sheaf of diff operators
%   (fill in details omitted in GR)

% \section{Lie Algebras and ``Formal'' Groups}

% \section{The Localization Functor}
% example of sl2-mod iso QCoh(P^1) as a correspondence

\section{Bibliography}
\printbibliography

\end{document}