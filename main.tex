\documentclass{article}
\input{preamble.tex}

\begin{document}

\title{M4R : Crystals }

\author{Ken Lee}
\date{2022}
\maketitle

\tableofcontents

``IP'' stands for ``indefinitely postponed'', 
which means the details are at the bottom of the priority list,
and I will only fill them in if I have time at the end.

\section{Introduction}

  \subsection{Why?}\subfile{why.tex}

\section{Setting}

  \subsection{How to work with ``DG Categories''}\subfile{dgcat.tex}

  \subsection{Derived Schemes}\subfile{dsch.tex}
% 1. DMod(X) = IndCoh(X_dR)
%    [Classically, nilpotent extensions were enough to consider.
%     What do we get from using all of dAff?]
% 2. LieAlg(IndCoh X) = Grp(FormMod / X)
%    [Q : this is where DAG is hiding?]
% 3. BB localization as B(exp g) <- B_X(exp g) -> X_dR

\section{IndCoherent Sheaves on Derived Schemes}

  \subsection{Quasi-Coherent Sheaves}\subfile{qcoh.tex}

  \subsection{Why Ind-Coherent Sheaves?}\subfile{indcoh.tex}

  \subsection{Correspondences and Six Functor Formalisms}\subfile{corr.tex}

  \subsection{Ind-Coherent Sheaves on a single Derived Scheme}
    \subfile{indCohdSch.tex}

  \subsection{Relation between Quasi-Coherent Sheaves and Ind-Coherent Sheaves}
    \subfile{qcohindcoh.tex}

  \subsection{Pushward and Pullback across Various Morphisms}

  \subsection{Integral Transforms}

  \subsection{Serre Duality}
% Desired computations : 
% - !-pullback for a non-proper non-open morphism
% - example of proper basechange
% - integral transform formula, at least in dAff case
% - the unit of !-tensor is really dualizing sheaf
% - the abstract duality functors match the classical computation

\section{Crystals and D-modules}

  \subsection{A note on inf-schemes}
  Say something about making $\INDCOH$ base change
  for larger class of spaces and morphisms
  that includes $X \to X_\DR$.

  \subsection{Left and Right Crystals}\subfile{crys.tex}

  \subsection{Induction of Left and Right Crystals}\subfile{induction.tex}

  \subsection{Equivalence of Left Crystals and D-modules}\subfile{dmodules.tex}

  \subsection{Kashiwara's Lemma}\subfile{kashiwara.tex}
% Desired computations : 
% - the crystals monad actually corresponds sheaf of diff operators
%   (fill in details omitted in GR)

% \section{Lie Algebras and ``Formal'' Groups}

% \section{The Localization Functor}
% example of sl2-mod iso QCoh(P^1) as a correspondence

%\printbibliography

\end{document}