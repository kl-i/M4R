\documentclass[./main.tex]{subfiles}
\begin{document}

TODO : 
\begin{itemize}
  \item describe how GR's Serre duality concretely
  recovers classical Serre duality
  \item IP : proof of $\om_X = \Om^d_X[d]$ in classical smooth case.
\end{itemize}

From the perspective of GR, 
both Serre duality and duality of D-modules are instances of
general duality of (left) modules over a commutative algebra object in
a symmetric monoidal $\infty$-category.
Extending the analogy with linear algebra : 
this is about getting an isomorphism $V \simeq V^\vee$ of vector spaces
from a non-degenerate bilinear form $\<\_,\_\> : V \otimes V \to k$.

There are many definitions of dualisability in a 
symmetric monoidal $\infty$-category.
We will take the following to be equivalent for granted.
\begin{prop}[Dualisable Objects (Lurie HA 4.6.1.6)]
  \link{duality.dualisable}
  
  Let $C$ be a symmetric monoidal $\infty$-category.
  Let $e : x^\vee \otimes x \to 1_C$ where $1_C$ is the unit.
  Then the following are equivalent : 
  \begin{enumerate}
    \item (duality datum) there exists $c : 1_C \to x \otimes x^\vee$
    such that we have the following commuting triangles in $hC$ : 
    \begin{cd}
      x & {x \otimes x^\vee \otimes x} & x^\vee \\
      & x & {x^\vee \otimes x \otimes x^\vee} & x^\vee
      \arrow["\id{}"', from=1-1, to=2-2]
      \arrow["{c \otimes \id{}}", from=1-1, to=1-2]
      \arrow["{\id{} \otimes e}", from=1-2, to=2-2]
      \arrow["{\id{} \otimes c}"', from=1-3, to=2-3]
      \arrow["{e \otimes \id{}}"', from=2-3, to=2-4]
      \arrow["{\id{}}", from=1-3, to=2-4]
    \end{cd}
    \item (Adjunction) for all $a \in C$, 
    the morphism $\id{} \otimes e : a \otimes x^\vee \otimes x \to a$
    induces an equivalence 
    \begin{cd}
      {C(b , a \otimes x^\vee)} & {C(b \otimes x , a \otimes x^\vee \otimes x)} \\
      & {C(b \otimes x , a)}
      \arrow["{\_ \otimes x}", from=1-1, to=1-2]
      \arrow["{(\id{} \otimes e)\,\_}", from=1-2, to=2-2]
      \arrow["\sim"', from=1-1, to=2-2]
    \end{cd}
    \item (Internal Hom) 
    Under the extra condition that $C$ has internal hom $\underline{\HOM}$,
    for any object $a \in C$,
    the morphism $e \otimes \id{} : a \otimes x^\vee \otimes x \to a$
    induces an equivalence \[
      a \otimes x^\vee \to \underline{\HOM}(x , a)
    \]

  \end{enumerate}
  An object $x$ is called \emph{dualisable} when 
  there exists another object $x^\vee$ together with
  a morphisms $e : x^\vee \otimes x \to 1_C$ satisfying any
  (and thus all) of the above.

\end{prop}

\begin{rmk}
  In the definition of dualisable objects,
  we assumed $C$ is symmetric monoidal.
  This implies given a dualisable pair $e : x^\vee \otimes x \to 1_C$,
  $e_\mathrm{transposed} : x \otimes x^\vee \to 1_C$ is also a dualisable pair.
  In particular, $(x^\vee)^\vee \simeq x$.
\end{rmk}

\begin{dfn}
  
  Let $C \in \DGCAT_\CTS$.
  Then $C$ is called \emph{dualisable} when
  it is dualisable as an object in $\DGCAT_\CTS = \VEC\MOD$.
\end{dfn}

\begin{rmk}
  
  Let's unpack the above a bit.
  To say $C \in \DGCAT_\CTS$ is dualisable means
  there exists $C^\vee \in \DGCAT_\CTS$ and 
  $e \in \DGCAT_\CTS(C^\vee \otimes C, \VEC)$
  such that for all $D \in \DGCAT_\CTS$,
  we have an equivalence
  \begin{cd}
    {\DGCAT_\CTS(E , D \otimes C^\vee)} & {\DGCAT_\CTS(E \otimes C , D \otimes C^\vee \otimes C)} \\
    & {\DGCAT_\CTS(E \otimes C , D)}
    \arrow["{\_ \otimes C}", from=1-1, to=1-2]
    \arrow["{(\id{} \otimes e)\,\_}", from=1-2, to=2-2]
    \arrow["\sim"', from=1-1, to=2-2]
  \end{cd}
  In particular, applying $D = \VEC$, 
  gives $C^\vee \simeq \DGCAT_\CTS(C , \VEC)$.
  On the other hand, applying $E = \VEC$ and a transposition yields
  $C^\vee \otimes D \simeq \DGCAT_\CTS(C , D)$.

  The following shows that compact generation of $C$ is sufficient for
  being dualisable and in fact, the dual $C^\vee$ has a explicit description.
\end{rmk}

\begin{prop}[Duality for Compactly Generated DG $\infty$-Categories
  (GR1 Ch1 7.3.2)]
  \link{duality.cg}
  
  Let $C \in \DGCAT_\CTS$ be compactly generated.
  Consider the following : 
  \begin{cd}
    {(C^c)^\OP \times C^c} 
      & {\IND((C^c)^\OP) \times C} 
        & {\IND((C^c)^\OP) \otimes C} \\
    & \VEC
    \arrow[from=1-1, to=1-2]
    \arrow["{\HOM_C}"', from=1-1, to=2-2]
    \arrow[dashed, from=1-2, to=2-2]
    \arrow[from=1-2, to=1-3]
    \arrow["{e}", dashed, from=1-3, to=2-2]
  \end{cd}
  The enriched hom bi-functor $\HOM_C$ on compact objects
  is $\VEC$-linear and exact in both variables 
  hence the
  middle dashed morphism 
  \linkto{dgcat.ind.up}{by left Kan extension in each variable}/
  This is $\VEC$-linear and continuous in both variables,
  hence the right dashed morphism in $\DGCAT_\CTS$ by
  the \linkto{dgcat.dgcat.tensor}{universal property of tensors}. 
  Then $e$ exhibits $\IND((C^c)^\OP)$ as the dual of $C$.

\end{prop}
\begin{proof}
  For any $E \in \DGCAT_\CTS$, we have the following equivalences
  \begin{align*}
    \DGCAT_\CTS(\DGCAT_\CTS(C , D) , E)
    &\simeq \DGCAT_\CTS(\DGCAT(C^c , D) , E) \\
    &\simeq \DGCAT_\CTS(E^\OP , \DGCAT(C^c , D)^\OP) \\
    &\simeq \DGCAT_\CTS(E^\OP , \DGCAT((C^c)^\OP , D^\OP)) \\
    &\simeq \DGCAT((C^c)^\OP , \DGCAT_\CTS(E^\OP , D^\OP)) \\
    &\simeq \DGCAT((C^c)^\OP , \DGCAT_\CTS(D , E)) \\
    &\simeq \DGCAT_\CTS(\IND((C^c)^\OP) , \DGCAT_\CTS(D , E)) \\
    &\simeq \DGCAT_\CTS(\IND((C^c)^\OP) \otimes D , E)
  \end{align*}
  where we have used the following equivalence without proof : 
  \[
    \DGCAT_\CTS(\_ , \star) \map{\sim}{} \DGCAT_\CTS(\star^\OP , \_^\OP)
  \]
  where the forward direction is obtained by taking right adjoints via the
  \linkto{dgcat.adjoint}{adjoint functor theorem} and the right side is
  well-defined because \linkto{dgcat.presentable.has_lim}{
    presentable $\infty$-categories have small limits
  }.
\end{proof}

\begin{prop}[$\QCOH\,A$ Compactly Generated (Derived Affine Case)]
  \link{duality.daff}

  Let $\SPEC\,A \in \DAFF$ and $C := \QCOH\,A$.
  Then $C \simeq \IND\,C(\om)$ where
  $C(\om)$ is the smallest stable full subcategory of $C$ containing $A$.
  In particular, $C$ is compactly generated.
  
  Furthermore, for $M \in C$, the following are equivalent : 
  \begin{itemize}
    \item $M$ is a retract of objects in $C(\om)$.
    \item $M$ is compact
    \item $M$ is dualisable
  \end{itemize}

\end{prop}
\begin{proof}

  Let $C := \QCOH\,A$.
  One can check that $A$ is a compact object of $\QCOH\,A$ by
  using the \linkto{dgcat.stable.cg}{1-categorical characterisation of
  compact objects} and the fact that
  homotopy coproducts in $h\QCOH\,A$ coincide with
  taking degree-wise coproducts.
  Similarly, one can check that $A$ is a generator
  at the level of the 1-categorical derived category $h\QCOH\,A$.
  \linkto{dgcat.stable.cg}{Hence} $C \simeq \IND\,C(\om)$.
  We also deduce that \linkto{dgcat.idem}{$C(\om) \to C^c$ exhibits $C^c$
  as the idempotent completion of $C(\om)$}.
  Therefore compact objects coincide with retracts of objects in $C(\om)$.

  We now show an $M \in C$ is compact if and only if it is dualisable.
  This is a slightly different proof to BZFN 3.4.

  Suppose $M$ is compact.
  Our ansatz for $M^\vee$ is $\underline{\HOM}(M,1)$,
  where $\underline{\HOM} := \underline{\HOM}_S$ is the internal hom of
  $\QCOH\,S$.
  For $e : M \otimes \underline{\HOM}(M,1) \to 1$,
  we choose it to correspond to the identity morphism 
  $\underline{\HOM}(M,1) \to \underline{\HOM}(M,1)$ under the adjunction
  $M\otimes \_ \dashv \underline{\HOM}(M , \_)$.
  \linkto{duality.dualisable}{It remains to show}
  that the natural transformation 
  $\_ \otimes \underline{\HOM}(M , 1) \to \underline{\HOM}(M , \_)$
  coming from 
  $\id{} \otimes e : \_ \otimes M \otimes \underline{\HOM}(M , 1) \to \_$
  is an equivalence.
  By assumption, $\underline{\HOM}(M , \_)$ preserves small coproducts.
  \linkto{dgcat.stable.colimits}
  {Hence}, both $\underline{\HOM}(M , \_)$ and 
  $\_ \otimes \underline{\HOM}(M , 1)$ preserves small colimits.
  Since $\QCOH\,S =: C \simeq \IND\,C(\om)$,
  \linkto{dgcat.ind.up}{it suffices} to show that
  $\_ \otimes \underline{\HOM}(M , 1) \to \underline{\HOM}(M , \_)$
  is an equivalence on $C(\om)$.
  Since $C(\om)$ is obtained from $1 = \OO(S)$ by 
  \link{dgcat.stable.cg}{iteratively adding it cofibers} and 
  both functors preserve small colimits, it suffices to show that
  $1 \otimes \underline{\HOM}(M , 1) \to \underline{\HOM}(M , 1)$
  is an equivalence.
  This is clear.

  Now assume $M$ is dualisable.
  Then $\underline{\HOM}(M , \_) \simeq \_ \otimes M^\vee$
  and hence preserves small coproducts.
  Therefore $M$ is compact.

\end{proof}

\begin{prop}[$\QCOH\,X$ Compactly Generated (Derived Scheme Case)]

  Let $X \in \DSCH_\mathrm{qc}$. 
  Then $\QCOH\,X \simeq \IND\,\PERF\,X$ where
  $\FF$ is in the full subcategory $\PERF\,X$ if and only if
  any of the following equivalent conditions are true : 
  \begin{itemize}
    \item $\FF$ is dualisable
    \item $\FF$ is compact
    \item for all $x : S \to X$ where $S$ derived affine,
    $\FF_x \in \QCOH\,S$ is compact.
  \end{itemize}
\end{prop}
\begin{proof}
  
  We omit the proof of $\QCOH\,X \simeq \IND\,\PERF\,X$ for brevity.
  See BZFN 3.19. 
  For $\FF \in \QCOH\,X$,
  the fact that $\FF$ is dualisable if and only if
  it is fiberwise dualisable is proved in BZFN 3.6.
  The equivalence of dualisability and compactness
  is shown in BZFN 3.9.
\end{proof}

We now show that $\QCOH\,X$ has a \emph{self-duality} when 
$X$ is a quasi-compact derived scheme.

\begin{prop}[Self Duality of $\QCOH\,X$]
  \link{duality.naive}

  Let $X \in \DSCH_\mathrm{qc}$.
  Define \[
    \<\_,\_\> : 
    \QCOH\,X \otimes \QCOH\,X \map{\otimes}{} \QCOH\,X \map{(p_X)_*}{} \VEC
  \]
  Then $\<\_,\_\>$ exhibits $\QCOH\,X$ as its own dual.
  Let $\D : \QCOH\,X \simeq (\QCOH\,X)^\vee$ correspond to $\<\_,\_\>$.
  We refer to $\D$ as the \emph{naive dualisation functor}.

\end{prop}
\begin{proof}
  This proved in GR1 Ch1 9.2 for general 
  stable rigid monoidal $\infty$-categories.
  What follows is an application of the general proof to this example.
  
  We know that the dual of $\QCOH\,X \simeq \IND\,\PERF\,X$
  is $\IND((\PERF\,X)^\OP)$.
  We thus compute the naive dualisation functor as : 
  \begin{align*}
    \QCOH\,X \map{}{} 
    \DGCAT_\CTS(\QCOH\,X , \VEC) \mapfrom{\sim}{}
    (\QCOH\,X)^\vee \\
    \FF \mapsto 
    \<\_ , \FF \> \simeq \HOM_X(\D \FF , \_)
    \mapsfrom \D \FF
  \end{align*}
  It suffices to show $\D$ is an equivalence.
  Since $\QCOH\,X$ is the ind-completion of $\PERF\,X$,
  \linkto{dgcat.cg.out}{it suffices} to show that
  $\D$ maps $\PERF\,X$ fully faithfully into 
  $(\PERF\,X)^\OP$ inside $(\QCOH\,X)^\vee$.

  Let $\FF \in \PERF\,X$.
  Then \[
    \HOM_X(\D \FF , \_) \simeq 
    \<\_ , \FF\> \simeq
    \HOM_X(\OO_X , \_ \otimes \FF) \simeq
    \HOM_X(\FF^\vee , \_) \simeq
    \HOM_X(\underline{\HOM}_X(\FF , \OO_X) , \_)
  \]
  Therefore on $\PERF\,X$, 
  the dualisation functor 
  $\D$ lands in the image under Yoneda of $(\PERF\,X)^\OP$.
  Identifying $(\PERF\,X)^\OP$ as a full subcategory of $(\QCOH\,X)^\vee$,
  we see that $\D(\_) \simeq \_^\vee \simeq \underline{\HOM}_X(\_ , \OO_X)$
  which is an equivalence $\PERF\,X \simeq (\PERF\,X)^\OP$
  because this is simply taking dual objects.

\end{proof}

We describe duality for $\INDCOH\,X$ when $X \in \DSCH_\AFT$.
Note that by definition, $\INDCOH\,X$ is compactly generated
and hence dualisable.
What's interesting is that $\INDCOH\,X$ is also self-dual
and this recovers the classical Serre duality for coherent sheaves.
First, we need to give $\INDCOH\,X$ a symmetric monoidal structure.

\begin{prop}[GR1 Ch4 6.3.2]

  There exists a unique functor
  \[
    \boxtimes \in \DGCAT_\CTS\brkt{
      \INDCOH\,X \otimes \INDCOH\,Y , \INDCOH(Y \times X)
    }
  \]
  that preserves compact objects and makes the following diagram commute : 
  \begin{cd}
    {\INDCOH\,X \otimes \INDCOH\,Y} & {\INDCOH(Y \times X)} \\
    {\QCOH\,X \otimes \QCOH\,Y} & {\QCOH(Y\times X)}
    \arrow["{\Psi_X \otimes \Psi_Y}"', from=1-1, to=2-1]
    \arrow["\boxtimes"', from=2-1, to=2-2]
    \arrow["{\Psi_{Y \times X}}", from=1-2, to=2-2]
    \arrow["\boxtimes", from=1-1, to=1-2]
  \end{cd}
\end{prop}

In GR1 Ch5 4.1, functoriality of $\boxtimes$ is shown.
\begin{prop}
 
  There is a symmetric monoidal structure on 
  $\INDCOH^! : \DSCH_\AFT^\OP \to \DGCAT_\CTS$ 
  such that for $X , Y \in \DSCH_\AFT$,
  the morphism \[
    \INDCOH\,X \otimes \INDCOH\,Y \map{}{} \INDCOH(Y \times X)
  \]
  from the symmetric monoidal structure of $\INDCOH^!$
  recovers $\boxtimes$. 
\end{prop}
\begin{proof}
  Omitted. See GR1 Ch5 4.1.
\end{proof}

\begin{rmk}
  
  Assuming the above result,
  we can give each $\INDCOH\,X$ a symmetric monoidal structure
  by the following trick from GR.
  A symmetric monoidal dg-$\infty$-category is by GR's definition
  a commutative algebra object in $\DGCAT_\CTS$.
  The idea is that the source of commutative algebra structure on 
  $\INDCOH\,X$ actually comes a commutative algebra structure on
  \emph{$X$ itself}.
  This means the following.
  
\end{rmk}

\begin{lem}[Symmetric Monoidal Structure on $\PSTK$]
  
  There is a symmetric monoidal structure $\otimes$ on $\PSTK^\OP$
  that gives for each $X , Y \in \PSTK^\OP$, \[
    X \otimes Y \simeq X \times Y \in \PSTK^\OP
  \]

  Furthermore, we have a functor
  $\PSTK^\OP \to \CALG(\PSTK^\OP)$
  which for every $X \in \PSTK^\OP$,
  gives it a commutative algebra structure such that
  the multiplication
  \[
    X \times X \to X  
  \]
  is given by (the opposite of) the diagonal $\De^\OP$.
  \begin{proof1}
    The proof is not so important for our purposes.
    We refer the reader to Lurie HA 2.4.1.5
    where the result is proved for any $\infty$-category
    admitting all finite products.
  \end{proof1}
\end{lem}

\begin{dfn}[GR1 Ch5 4.1.3]
  
  The symmetric monoidal functor $\INDCOH^! : \DSCH_\AFT^\OP \to \DGCAT_\CTS$
  sends commutative algebra objects to commutative algebra objects.
  So we have the following :
  \begin{cd}
    {\DSCH_\AFT^\OP} & {\DGCAT_\CTS} \\
    {\CALG(\DSCH_\AFT^\OP)} & {\CALG(\DGCAT_\CTS)}
    \arrow["{\INDCOH^!}", from=1-1, to=1-2]
    \arrow["{\text{use diagonal}}"', shift right=2, from=1-1, to=2-1]
    \arrow["{\INDCOH^!}"', from=2-1, to=2-2]
    \arrow["{\text{``forget''}}"', from=2-2, to=1-2]
    \arrow["{\text{``forget''}}"', shift right=2, from=2-1, to=1-1]
  \end{cd}
  Hence we obtain a factoring of $\INDCOH^!$ through the forgetful functor 
  $\CALG(\DGCAT_\CTS) \to \DGCAT_\CTS$.

  For $X \in \DSCH_\AFT$,
  we use $\OTIMES$ to denote the symmetric monoidal operation
  on $\INDCOH\,X$.
  
\end{dfn}

\begin{rmk}
  
  Tracing through the above functor
  $\INDCOH^! : \DSCH_\AFT^\OP \to \CALG(\DSCH_\AFT^\OP) \to \CALG(\DGCAT_\CTS)$,
  we see that for $X \in \DSCH_\AFT$,
  the symmetric monoidal operation for $\INDCOH\,X$ is the following : 
  \begin{cd}
    {X \times X} & \rightsquigarrow & {\INDCOH\,X \otimes \INDCOH\,X} & {\INDCOH(X \times X)} \\
    X &&& {\INDCOH\,X}
    \arrow["\boxtimes", from=1-3, to=1-4]
    \arrow["{\De_X^!}", from=1-4, to=2-4]
    \arrow["\OTIMES"', from=1-3, to=2-4]
    \arrow[from=1-1, to=2-1]
  \end{cd}
\end{rmk}

\begin{prop}[``Serre Duality'']
  \link{duality.serre}
  
  Let $X \in \DSCH_\AFT$.
  Define \[
    \<\_,\_\> : 
    \INDCOH\,X \otimes \INDCOH\,X \map{\OTIMES}{} 
    \INDCOH\,X \map{(p_X)_*}{} \VEC
  \]
  Then $\<\_,\_\>$ exhibits $\INDCOH\,X$ as its own dual.
  Let $\D_\SERRE : \INDCOH\,X \simeq (\INDCOH\,X)^\vee$
  correspond to $\<\_,\_\>$. 
  We refer to $\D_\SERRE$ as the \emph{Serre duality functor}.
\end{prop}
\begin{proof}
  
  Let $e := \<\_ , \_\>$.
  Following GR1 Ch5 4.2.8, 
  we will explicitly give a morphism 
  $c : \VEC \to \INDCOH\,X \otimes \INDCOH\,X$ in $\DGCAT_\CTS$
  making $(c , e)$ into duality datum.
  We want to map into $\INDCOH\,X\otimes\INDCOH\,X$.
  The only way we can do that by definition is 
  via $\otimes : \INDCOH\,X \times \INDCOH\,X \to \INDCOH\,X\otimes\INDCOH\,X$.
  However, ...
  should i just defer to reference? GR1 Ch4 6.3.4.
  This is why GR decided to do it first.

\end{proof}

  
\end{document}