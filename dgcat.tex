\documentclass[./main.tex]{subfiles}
\begin{document}
  
An important example of a compactly generated stable $\infty$-category
is $\VEC$.

\begin{prop}
  
  Let $k$ be a field.
  Then there exists an $\infty$-category $\VEC$ called
  the \emph{right derived $\infty$-category of $k$-vector spaces} 
  with the following properties : 
  \begin{itemize}
    \item \cite[Prop 1.3.2.18]{Lurie-HA} $\VEC$ is stable. 
    \item (Universal Property as Localisation \cite[Prop 1.3.4.4]{Lurie-HA}) 
    There is a functor $l : \CH^-(k) \to \VEC$
    with the property that for all $\infty$-categories $E$,
    restricting along $l$ yields a fully faithful functor
    $\FUN(\VEC , E) \to \FUN(\CH^-(k) , E)$
    with essential image consisting of functors $\CH^-(k) \to E$
    which invert quasi-isomorphisms.
    \item \cite[Prop 1.3.2.9]{Lurie-HA}
    $h\VEC$ gives the usual 1-category right derived category
    of $k$-vector spaces.
    
    Consequently, for $X, Y \in \CH^-(k)$,
    we have \[
      \pi_n \VEC(X,Y) = \pi_0 \Om^n \VEC(X,Y)
      \simeq \pi_0 \VEC(X , Y[n]) =: \EXT^n(X,Y)
    \]
    \item Derived tensor $\otimes$ makes $\VEC$ into a symmetric monoidal
    infinity category.

    \item $\VEC$ is compactly generated. See \linkto{duality.daff}{later}.
  \end{itemize}
\end{prop}
Practically speaking, 
computations tensor product are done by using the projective model structure 
on the category of complexes of $k$-vectors spaces.

% $1\dash\CAT^\mathrm{ex}_\mathrm{cts}$ 
% has symmetric monoidal structure $\otimes$
% via the \emph{Lurie tensor product}. 

% We won't really need to know anything about the tensor product
% beacuse we will specialise to

% (IP : Explain how practically speaking, 
% it suffices to know the universal property
% because we will work with compactly generated dg-categories,
% meaning computation will come down to
% compact generators and their homs.)

% (IP : Give impression of symmetric monoidal infinity categories via
% Lawvere theory perspective.)

% \textbf{
% Unanswered Q : is this only used because
% the theory of commutative dg-algebras compromises in positive characteristic?
% }
% \textbf{
%   A : No. Characteristic zero is also used later in equivalence of
%   Lie algebras and formal moduli problems.
%   But I have no time to look into this.
% }

% $(\VEC, \otimes)$ 
% is the stable symmetric monoidal (right bounded) derived infinity category of 
% complexes of $k$-vector spaces.
% Its homotopy category $h(\VEC)$ is 1-categorical localisation of 
% the category of complexes of $k$-vectors spaces at quasi-isomorphisms,
% and has the usual $t$-structure.
% The heart of $\VEC$ is the usual abelian category of $k$-vector spaces.
% We use cohomological degree,
% where negative cohomological degree refers to homological degree.
% The 

$(\VEC,\otimes)$ can be seen as an commutative algebra object in 
the symmetric monoidal infinity category $(\infty,1)\dash\CAT^\EX_\CTS$.
\begin{dfn}
  $\DGCAT_\CTS$ denotes the infinity category of 
  left modules over $\VEC$ inside $(\infty,1)\dash\CAT^\EX_\CTS$.

  \cite[Ch 1, 3.4, 10.3.3,]{GR1}
\end{dfn} 
It is beyond the scope of this paper to give a precise definition of this.
We will however describe some properties sufficient for our purposes.

Let $C \in \DGCAT_\CTS$.
Then there will be a functor $\VEC \otimes C \to C$ in 
$(\infty,1)\dash\CAT^\EX$.
Since the unit for the symmetric monoidal structure of $\VEC$ is $k$,
the functor $k \otimes \_ \in \FUN^\EX_\CTS(C , C)$
will be isomorphic to the the identity functor of $C$.
Fixing $x \in C$, we obtain a functor 
$\_ \otimes x \in (\infty,1)\dash\CAT^\EX_\CTS(\VEC , C)$.
By the \linkto{dgcat.adjoint}{adjoint functor theorem},
we have an adjunction \[
  \_ \otimes x \dashv \HOM_C(x , \_) : 
  \VEC \rightleftarrows C
\]
in $(\infty,1)\dash\CAT^\EX$.
This way for every pair of objects $x , y \in C$,
we have a complex of $k$-vector spaces $\HOM_C(x,y) \in \VEC$.

There is also a notion of \emph{tensor product of dg-categories over $\VEC$} 
with the expected universal property.
\begin{prop}
  \link{dgcat.dgcat.tensor}
  
  Let $C, D \in \DGCAT_\CTS$.
  Then there exists $C \otimes_\VEC D \in \DGCAT_\CTS$ together with
  a functor $\boxtimes : C \times D \to C \otimes_\VEC D$
  that is $\VEC$-linear and continuous in each component and
  for any $E \in \DGCAT_\CTS$, we have an equivalence
  \[
    \DGCAT_\CTS(C \otimes_\VEC D , E) \map{\sim}{}
    \DGCAT_{\mathrm{Bi-Cts}}(C \times D , E)
  \]
  given by restriction along $\boxtimes$ and the latter is
  the full subcategory of $\DGCAT(C \times D, E)$
  which is $\VEC$-linear and continuous in each component.
  \cite[Ch 1, 10.4]{GR1}
\end{prop}



% IP : derived rings stuff - Lurie HTT 5.5.9.3

% IP :  modules over derived rings as 
% symmetric monoidal $\infty$-cat - Lurie HA 7.1.2.13.
% Subtlety about different model structures is Lurie HA 7.1.2.9.

\end{document}