\documentclass[./main.tex]{subfiles}
\begin{document}

\begin{itemize}

  

  \item $\DAFF$ to be the opposite of the 
  infinity category of coconnective commutative dg-algebras over $k$.

  (IP : Explain how this is the infinity category of 
  commutative algebra objects in $\VEC$ and how practically
  it is the infinity category underlying the 
  model category of usual commutative dg-algebras over $k$,
  meaning all computations can be done 1-categorically
  using the model structure.)

  \item $\SPC$ denote the infinity category of infinity groupoids.
  For any infinity category $C$, let $\PSH\, C := (\SPC^{op})^C$
  be the infinity category of contravariant functors from $C$ to $\SPC$.
  Objects of $\PSH\, C$ are called \emph{presheaves over $C$}.
  
  \item $\PRE{\STK} := \PSH\,\DAFF$ denotes
  the infinity category of \emph{prestacks},
  which are contravariant functors from $\DAFF$ to $\SPC$.
  We will use $\SPEC : \DAFF \to \PRE{\STK}$ to denote
  the Yoneda embedding so that for a commutative dg-algebra $A$ over $k$,
  $\SPEC\, A$ denotes the corresponding prestack.
  The objects in the essential image of $\SPEC$ are called 
  \emph{derived affine schemes}.
\end{itemize}

\begin{itemize}
  \item
  $A \in \DAFF^{\leq n}$ means $H^{< -n} A = 0$.
  $\AFF := \DAFF^{\leq 0}$ consists of classical affine schemes and 
  $\AFF_\red$ is the full subcategory of $\AFF$
  consisting of reduced classical affine schemes.
  $A \in \DAFF^{<\infty} := \bigcup_{n \geq 0} \DAFF^{\leq n}$
  are called \emph{eventually coconnective}.

  $\DAFF^{<\infty}_{\ft}$ denotes the full subcategory of $\DAFF^{<\infty}$
  consisting \emph{finite type} $A$,
  which means $H^0 A$ finite type over $k$ and 
  $H^{<0} A$ finitely generated over $H^0 A$.

  \item 
  The fully faithful embedding $\AFF_\red \to \DAFF$
  induces an adjunction $\PSH \AFF_\red \leftrightarrow \PRE{\STK}$
  with right adjoint restriction and left adjoint left Kan extension,
  which is also full and faithful.
  Objects in the essential image will be called
  \emph{classical prestacks}.
  
  $\PRE{\STK}_{\laft} := \PSH\,\DAFF^{<\infty}_{\ft}$ denotes 
  the infinity category of \emph{prestacks of locally almost finite type}.
  
  We have $\DAFF^{<\infty}_{\ft} \subs \DAFF^{<\infty} \subs \DAFF$.
  Then by first left Kan extending then right Kan extending,
  we get that $\PRE{\STK}_{\laft}$ fully faithfully embeds into $\PRE{\STK}$.
  We henceforth treat $\PRE{\STK}_{\laft}$ as a full subcategory of $\PRE{\STK}$.

  \textbf{
    Unanswered Q : where is the $\laft$ condition used in the theory?
  }

  \item Let $f : A \to B$ be a morphism of commutative dg-algebras.
  
  $f$ flat := $H^0 A \to H^0 B$ flat morphism of $k$-algebras and
  $H^0 B \otimes_{H^0 A} H^{< 0} A \map{\sim}{} H^{< 0} B$.

  $f$ smooth, étale, open immersion, Zariski-cover := 
  $f$ flat and on nil-(smooth, étale, open immersion, Zariski-cover).

\end{itemize}

\subsection{How to work with ``DG Categories''}

Gaitsgory-Rozenblyum has a highly abstract but clean definition of
dg-categories, which can be summarised in the following diagram : 
\begin{cd}
	& {(\infty,1)\text{-}\mathrm{Cat}} & {(\infty,1)\text{-}\mathrm{Cat}^\mathrm{ex}} & {\mathrm{Vec}^\mathrm{f.d.}\text{-}\mathrm{Mod}} \\
	{} & {\bigcup_\kappa (\infty,1)\text{-}\mathrm{Cat}^\mathrm{ex}_{\kappa\text{-c.g.}}} & {(\infty,1)\text{-}\mathrm{Cat}^\mathrm{ex}_\mathrm{cts}} & {\mathrm{Vec}\text{-}\mathrm{Mod}} \\
	& {} & {} & {}
	\arrow[from=1-3, to=1-2]
	\arrow[from=1-4, to=1-3]
	\arrow["{=:}"{description}, draw=none, from=2-3, to=2-2]
	\arrow[from=2-3, to=1-3]
	\arrow[from=2-4, to=1-4]
	\arrow[from=2-4, to=2-3]
\end{cd}
% https://q.uiver.app/?q=WzAsOCxbMSwwLCIoXFxpbmZ0eSwxKVxcdGV4dHstfVxcbWF0aHJte0NhdH0iXSxbMiwwLCIoXFxpbmZ0eSwxKVxcdGV4dHstfVxcbWF0aHJte0NhdH1eXFxtYXRocm17ZXh9Il0sWzMsMCwiXFxtYXRocm17VmVjfV5cXG1hdGhybXtmLmQufVxcdGV4dHstfVxcbWF0aHJte01vZH0iXSxbMSwyLCIoXFxpbmZ0eSwxKVxcdGV4dHstfVxcbWF0aHJte0NhdH1eXFxtYXRocm17UHJ9X0wiXSxbMiwyLCIoXFxpbmZ0eSwxKVxcdGV4dHstfVxcbWF0aHJte0NhdH1eXFxtYXRocm17ZXh9X1xcbWF0aHJte2N0c30iXSxbMywyLCJcXG1hdGhybXtWZWN9XFx0ZXh0ey19XFxtYXRocm17TW9kfSJdLFsxLDEsIlxcbWF0aHJte0FjY30iXSxbMCwxLCJcXG1hdGhybXtBY2N9X1xca2FwcGEiXSxbMSwwXSxbMiwxXSxbNiwwXSxbMyw2XSxbNCwzXSxbNSw0XSxbNCwxXSxbNSwyXSxbNyw2XSxbNywwLCIiLDEseyJvZmZzZXQiOjF9XSxbMCw3LCJcXG1hdGhybXtJbmR9X1xca2FwcGEiLDIseyJsYWJlbF9wb3NpdGlvbiI6NzAsIm9mZnNldCI6NSwic2hvcnRlbiI6eyJzb3VyY2UiOjIwfX1dLFs3LDAsIlxcYm90IiwxLHsib2Zmc2V0IjotMiwic3R5bGUiOnsiYm9keSI6eyJuYW1lIjoibm9uZSJ9LCJoZWFkIjp7Im5hbWUiOiJub25lIn19fV1d
Some explanations are due :  
\begin{itemize}
  \item We will use ``infinity category'' to refer only to 
  $(\infty,1)$-categories.
  The word ``category'' will exclusively refer to $1$-categories.

  $(\infty,1)\dash\CAT$ denotes the infinity category of 
  small infinity categories. 
  $(\infty,1)\dash\CAT$ has all small limits (Kerodon 7.4.1.11)
  and small colimits (Kerodon 7.4.3.13).

  Just like one can learn how to compute with integrals without first knowing
  the definition of Lebesgue integration,
  we won't need to know the precise definition of $\infty$-categories
  because we will soon specialise to the ones we can \emph{compute} with,
  namely \emph{compactly generated $\infty$-categories}.
  Before this, we first explain
  the $\infty$-categories with homological significance : 
  stable $\infty$-categories.

  % (IP : Explain how $1\dash\CAT$ classifies cocartesian fibrations.)

  \item $(\infty,1)\dash\CAT^\mathrm{ex}$ denotes subcategory of $1\dash\CAT$
  consisting of \emph{stable infinity categories} and \emph{exact functors}.  
  It contains all small limits and
  the ``inclusion'' $1\dash\CAT^{ex} \to 1\dash\CAT$ preserves 
  small limits (Lurie HA 1.1.4.4).
  We note also that the functor $\infty$-category $\FUN(C,D)$
  is stable given $D$ is stable (Lurie HA 1.1.3.1),
  and hence $1\dash\CAT^\mathrm{ex}$ is cartesian closed.

  Stable infinity categories are basically
  triangulated categories where exact triangles are determined by
  an infinity-categorical universal property.
  Here is the definition.
  \begin{dfn}
    
    Let $C$ be an infinity category. 
    We say $C$ has a \emph{zero object} when
    it has an object that is both initial and final. 
    (Lurie HA 1.1.1.1.)

    Now assume $C$ have a zero object.
    Then a \emph{triangle} in $C$ is defined as a diagram in $C$ of the form : 
    \begin{cd}
      X & Y \\
      0 & Z
      \arrow[from=1-1,to=1-2]
      \arrow[from=1-1,to=2-1]
      \arrow[from=1-2,to=2-2]
      \arrow[from=2-1,to=2-2]
    \end{cd}
    A triangle is called a \emph{fiber sequence} when it is a cartesian
    and a \emph{cofiber sequence} when it is cocartesian.
    (Lurie HA 1.1.1.4.)
    In the first case,
    we say \emph{$Y \to Z$ admits a kernel} and refer to $X$ as the kernel,
    and in the other case
    we say \emph{$X \to Y$ admits a cokernel} and refer to $Z$ as the cokernel.
    \footnote{
      In Lurie HA,
      kernels are called fibers and cokernels are called cofibers.
    }

    $C$ is called \emph{stable} when the following are true : 
    \begin{itemize}
      \item every morphism has both a kernel and a cokernel.
      \item A triangle is fiber sequence iff it is a cofiber sequence.
      Such triangles are called \emph{exact triangles}.
    \end{itemize}
    (Lurie HA 1.1.1.9.)

    An exact functor $F : C \to D$ between stable infinity categories
    is one which satisfy any of the following equivalent conditions : 
    \begin{itemize}
      \item $F$ preserves exact triangles
      \item $F$ preserves finite limits
      \item $F$ preserves finite colimits.
    \end{itemize}
    (Lurie HA 1.1.4.1)
  \end{dfn}
  To help build intuition of ``stable infinity categories as 
  fixed triangulated categories'',
  we record here the important parts of the procedure
  of extracting a triangulated category from a stable infinity category.
  \begin{prop}[Lurie 1.1.2.14]
    
    Let $C$ be a stable infinity category.
    Then the following defines a triangulated structure on 
    the 1-category $hC$ : 
    \begin{itemize}
      \item Define the \emph{suspension functor} $\Sigma : C \to C$ by
      pushout against zeros : 
      \begin{cd}
        X & 0 \\
        0 & \Sigma X
        \arrow[from=1-1,to=1-2]
        \arrow[from=1-1,to=2-1]
        \arrow[from=1-2,to=2-2]
        \arrow[from=2-1,to=2-2] 
      \end{cd}
      Since the above square is a cofiber sequence,
      it is also a fiber sequence. 
      This shows that \emph{looping} $\Omega : C \to C$,
      given by pullback against zeros, gives an inverse for $\Sigma$
      and hence shows that $\Sigma$ is an equivalence.
      Taking homotopy categories, we obtain an equivalence 
      $[1] : hC \map{\sim}{} hC$, which we use as the shift functor
      for the triangulated structure.

      \item We call a diagram \[
        X \map{f}{} Y \map{g}{} Z \map{h}{} X[1] 
      \]
      in $hC$ an exact triangle (in the triangulated categorical sense) 
      when it comes from a diagram of the following form in $C$ : 
      \begin{cd}
        X & Y & 0 \\
        0 & Z & {X[1]}
        \arrow[from=1-1, to=1-2]
        \arrow[from=1-2, to=2-2]
        \arrow[from=2-2, to=2-3]
        \arrow[from=1-1, to=2-1]
        \arrow[from=2-1, to=2-2]
        \arrow[from=1-2, to=1-3]
        \arrow[from=1-3, to=2-3]
        \arrow["\lrcorner"{anchor=center, pos=0.125, rotate=180}, 
          draw=none, from=2-2, to=1-1]
        \arrow["\lrcorner"{anchor=center, pos=0.125, rotate=180}, 
          draw=none, from=2-3, to=1-2]
      \end{cd}
      i.e. two exact triangles (in the stable infinity categorical sense).
      \item For $X, Y$ objects of $C$,
      we have 
      \begin{align*}
        C(X,Y) &\simeq C(\Sigma \Omega X , Y) \simeq \Omega C(\Omega X , Y) \\
        &\simeq C(\Sigma^2 \Omega^2 X , Y) \simeq \Omega^2 C(\Omega^2 X , Y)
      \end{align*}
      Upon taking $\pi_0$ , we obtain 
      \[
        hC(X,Y) := \pi_0 C(X,Y) \simeq \pi_1 C(\Om X , Y) 
        \simeq \pi_2 (\Om^2 X , Y)
      \]
      where the last isomorphism is a group morphism.
      For $\pi_2$ of any ``space'' \footnote{
        In the quasi-category model of infinity categories,
        $C(X,Y)$ is a Kan complex,
        which one can take homotopy groups of.
      } the obvious group structure given by is abelian,
      this gives $hC(X,Y)$ an abelian group structure,
      making $hC$ into an additive category.
    \end{itemize}
  \end{prop}

  \item We are now ready for compactly generated $\infty$-categories.
  We will only make use of the case of 
  \emph{stable} compactly generated $\infty$-categories
  since many definitions then admit alternative characterisations
  which can be checked at the level of triangulated categories.
  
  For $\ka$ a regular cardinal\footnote{
    A regular cardinal $\ka$ is a cardinality that is 
    ``sufficiently large'' in the sense that
    the 1-category $\SET_{<\ka}$ of sets with cardinality strictly less than
    $\ka$ has all colimits of size strictly less than $\ka$.
    The cardinality of $\N$ is an example, 
    since a finite colimit of finite sets is still finite.
  }, 
  we denote with $(\infty,1)\dash\CAT^\mathrm{ex}_{\ka\text{-c.g.}}$
  the subcategory of $(\infty,1)\dash\CAT^\mathrm{ex}$ whose objects are
  \emph{$\ka$-compactly generated stable $\infty$-categories}
  and morphisms are colimit preserving functors.

  To explain, 
  the starting point is the theory of \emph{inductive cocompletions}\footnote{
    In the literature, this is called ind-completion.
    This is a bit of a misnomer because 
    intuitively we are adding filtered \emph{colimits}, not limits.
  }.
  Here is the main result concerning ind-completions in the stable case.
  \begin{prop}[Ind-completions of Stable $\infty$-Categories]
  
    Let $C$ be a small $\infty$-category and $\ka$ a regular cardinal.
    Then the Yoneda embedding $C \to \PSH\,C$ factors through a full subcategory
    $\IND_\ka(C)$ with the following properties : 
    \begin{itemize}
      \item (Lurie HTT 5.3.5.3) 
      $\IND_\ka(C)$ has all small $\ka$-filtered colimits
      and the inclusion $\IND_\ka(C) \subs \PSH\, C$ preserves them
      \item (Lurie HTT 5.3.5.4) 
      An object $X$ in $\PSH\,C$ is in $\IND_\ka(C)$ iff
      it is a $\ka$-filtered colimit of representables iff
      $X : C\op \to \SPC$ preserves $\ka$-small limits.
      \item (Lurie HA 1.1.3.6) If $C$ is stable then so is $\IND_\ka(C)$.
      \item (Lurie HTT 5.3.5.10) For any $\infty$-category $D$ 
      admitting small $\ka$-filtered colimits,
      we have the following equivalence of functor $\infty$-categories : 
      \[
        \FUN_\ka(\IND_\ka(C) , D) \map{\sim}{} \FUN(C , D)
      \]
      where \begin{itemize}
        \item the left category denotes the full subcategory of 
        $\FUN(\IND_\ka(C),D)$ consisting of functors preserving 
        $\ka$-filtered colimits.
        \item the forward functor is given by restricting along the Yoneda embedding 
        $C \to \IND_\ka(C)$
        \item the inverse functor is given by left Kan extension.
      \end{itemize} 
      Assuming $D$ is stable, the above equivalence restricts to
      an equivalence between the following two full subcategories :
      \[
        \FUN_\ka^\EX(\IND_\ka(C) , D) \map{\sim}{} \FUN^\EX(C , D)
      \]
      where
      \begin{itemize}
        \item $\FUN^\EX(C,D)$ is the full subcategory of $\FUN(C,D)$
        consisting of exact functors.
        \item $\FUN_\ka^\EX(\IND_\ka(C),D)$ is the full subcategory of
        $\FUN_\ka(\IND_\ka(C),D)$ consisting of exact functors
        preserving $\ka$-filtered colimits.\footnote{
          Left kan extending an exact functor $C \to D$
          to $\IND_\ka(C) \to D$ stays exact essentially because

        }
      \end{itemize}


    \end{itemize}
  \end{prop}
  
  For a regular cardinal $\ka$ and an $\infty$-category $C$,
  we say $C$ is $\ka$-compactly generated when
  it has all small colimits and there exists a small $\infty$-category
  $C^0$ with an equivalence $\IND_\ka(C^0) \map{\sim}{} C$.\footnote{
    This is unraveled from Lurie HTT 5.5.7.1, 5.5.0.18, and 5.4.2.1.
    In particular, the second condition is usually called 
    $\ka$-accessibility, but we have no need for such terminology.
  }
  For the case of $\ka =$ cardinality of $\N$,
  we simply say \emph{compactly generated}.

  For functors out of $\IND_\ka(C)$,,
  fully faithfulness and equivalence can be detected at the level of $C$.
  \begin{prop}[Functors out of Compactly Generated Categories]
    
    Let $C^0$ be a small $\infty$-category, $\ka$ a regular cardinal,
    $C = \IND_\ka(C^0)$ and
    $D$ an $\infty$-category admitting $\ka$-filtered colimits.
    Let $D^\ka$ be the full subcategory of $D$ consisting of 
    $\ka$-compact objects.
    Let $F : C \to D$ be a functor preserving $\ka$-filtered colimits
    and $F_0 : C^0 \to D$ its restriction along 
    the Yoneda embedding $C^0 \to C$.
    \begin{itemize}
      \item If $F_0$ is fully faithful and its essential image lands in $D^\ka$,
      then $F$ is fully faithful.
      \item $F$ is an equivalence iff the following are true :
      \begin{itemize}
        \item $F_0$ fully faithful
        \item $F_0$ factors through $D^\ka$
        \item all objects of $D$ are $\ka$-filtered colimits of diagrams
        in $D$ with objects in the image of $F_0$.
      \end{itemize}
      In particular,
      for any full subcategory $\tilde{C} \subs C^\ka$ which
      generates $C$ under $\ka$-filtered colimits,
      we have $\IND_\ka(\tilde{C}) \map{sim}{} C$.
    \end{itemize}
    (Lurie 5.3.5.11)

    Furthermore, 
    the Yoneda embedding $C^0 \to C$ factors through $C^\ka$.
    The fully faithful functor $C^0 \to C^\ka$ is not in general an equivalence,
    however it does exhibit $C^\ka$ as the \emph{idempotent completion}
    of $C^0$.
    (Lurie 5.4.2.4.)
    So if $C^0$ is idempotent complete, 
    then we recover the $\ka$-compact objects of $C$ as precisely 
    (the essential image of) $C^0$.

  \end{prop}

  Readers need not worry about what the $\infty$-categorical definition of 
  ``idempotent complete'' means,
  since when we specialise to \emph{stable $\infty$-categories},
  we will be able to detect this at the level of
  triangulated categories.

\item Ultimately, we would like to be able to create functors 
out of our favourite $\infty$-categories : 
derived $\infty$-categories of quasi-coherent sheaves $\QCOH(X)$
on a scheme $X$.

These $\infty$-categories tend to be ``generated'' by a much smaller 
(sometimes finite!) subcategory. 
The precise sense of ``generated'' is that of \emph{inductive cocompletions}.
The following is the main result.






\item $(\infty,1)\dash\CAT^\mathrm{Pr}_L$ denotes the
subcategory of $(\infty,1)\dash\CAT$ whose objects are
\emph{presentable $\infty$-categories} and
morphisms are colimit-preserving functors.

The point of presentable $\infty$-categories is that
we can use the adjoint functor theorem without worry.
\begin{prop}[Lurie HTT 5.5.2.9]
  
  Let $F : C \to D$ be a functor between presentable $\infty$-categories.
  \begin{itemize}
    \item $F$ admits a right adjoint iff $F$ preserves all small colimits.
    \item $F$ admits a left adjoint iff $F$ preserves all small limits
    and is accessible.
  \end{itemize}
\end{prop}



\item 
  $(\infty,1)\dash\CAT^\mathrm{ex}_\mathrm{cts}$ 
  denotes the subcategory of $1\dash\CAT^\mathrm{ex}$
  consisting of cocomplete stable infinity categories and 
  exact, colimit-preserving\footnote{
    GR confusingly refers to colimit-preservating as ``continuous''.
    This conflicts with the category theory literature where
    ``continuous'' usually refers to limit-preserving and 
    ``cocontinuous'' refers to colimit-preserving.
  } functors.
  For stable infinity categories,
  cocompleteness is equivalent to existence of arbitrary coproduct.
  (Intuition from 1-categories : colimits exist iff coequalizers and 
  arbitrary coproducts exist. Having cokernels gives coequalizers,
  so colimits exist iff arbitrary coproducts do.)
  Also, exact functors between cocomplete stable infinity categories are
  colimits preserving iff they preserve arbitrary coproduct.
  
  \item $1\dash\CAT^\mathrm{ex}_\mathrm{cts}$ 
  has symmetric monoidal structure $\otimes$
  via the \emph{Lurie tensor product}. 

  We won't really need to know anything about the tensor product
  other than its universal property. 

  (IP : Explain how practically speaking, 
  it suffices to know the univeral property
  because we will work with compactly generated dg-categories,
  meaning computation will come down to
  compact generators and their homs.)

  (IP : Give impression of symmetric monoidal infinity categories via
  Lawvere theory perspective.)

  \item An example of a cocomplete stable infinity category 
  is the infinity category of complexes of vector spaces over a field $k$.
  Fixing $k$, we will use $\VEC$ to denote it.
  
  The precise definition of $\VEC$ depends on the model of infinity categories
  one chooses to use.
  All we really need to know is that
  objects of $\VEC$ are complexes of vector spaces
  and for two objects $V, W$ in $\VEC$,
  the points of 
  
  \item $k$ field of characteristic zero.
  
  \textbf{
  Unanswered Q : is this only used because
  the theory of commutative dg-algebras compromises in positive characteristic?
  }

  \item $(\VEC, \otimes)$ 
  is the stable symmetric monoidal infinity category of 
  complexes of $k$-vector spaces.
  Its homotopy category $h(\VEC)$ is 1-categorical localisation of 
  the category of complexes of $k$-vectors spaces at quasi-isomorphisms,
  and has the usual $t$-structure.
  The heart of $\VEC$ is the usual abelian category of $k$-vector spaces.
  We use cohomological degree,
  where negative cohomological degree refers to homological degree.

  Practically speaking, 
  computations tensor product are done by using the projective model structure 
  on the category of complexes of $k$-vectors spaces.

  \item $(\VEC,\otimes)$ can be seen as an commutative algebra object in 
  the symmetric monoidal infinity category $1\dash\CAT^{st,cocompl}_{cts}$.
  Then $\DGCAT_{cts}$ denotes the infinity category of 
  modules over $\VEC$ inside $1\dash\CAT^{st,cocompl}_{cts}$.

  (IP : Explain how practically speaking, 
  computations such as infinity categorical limits 
  can be done via the model category of 
  honest-to-god dg-catgeories over $k$.)

  (IP : Explain how $\VEC\dash\MOD(1\dash\CAT^{st,cocompl}_{cts})$
  is closed symmetric monoidal infinity category.)
\end{itemize}

\subsection{Quasi-Coherent Sheaves}

\begin{dfn}[Covariant $\QCOH$]

  There is a functor $\QCOH_* : \DAFF \to \DGCAT_{cts}$
  that assigns to each $A \in \DAFF$ the 
  \emph{derived category of $A$-modules}, denoted $A\dash\MOD$.

  For $A$ discrete (i.e. a commutative ring),
  there is the following description of $A\dash\MOD$
  under the quasi-category model of $\infty$-categories
  summarised in a single diagram : 
  \begin{cd}
    {N(\mathrm{Ch}\,A)} \\
    {N_{dg}(\mathrm{Ch}\,A)} & 
      {N_{dg}((\mathrm{Ch}\,A)_f)} & {A\text{-}\mathrm{Mod}} \\
    & {D(A)}
    \arrow["\subseteq"', from=1-1, to=2-1]
    \arrow["L", shift left=2, from=2-1, to=2-2]
    \arrow["\supseteq", shift left=2, from=2-2, to=2-1]
    \arrow["\bot"{description}, draw=none, from=2-1, to=2-2]
    \arrow["{W^{-1}}", shift left=3, from=1-1, to=2-2]
    \arrow["{=:}"{description}, draw=none, from=2-2, to=2-3]
    \arrow["h", from=2-2, to=3-2]
  \end{cd}
  Details : 
  \begin{itemize}
    \item $\CH\,A$ is the honest-to-god dg-category of chain complexes of
    honest-to-god $A$-modules
    and $D(A)$ is the category of complexes of injectives 
    \item $\CH\,A$ has a model structure such that 
    cofibrations are degree-wise injections and 
    weak equivalences are quasi-isomorphisms (Lurie HA 1.3.5.3).
    Although the class of fibrations are defined abstractly as
    those satisfying right lifting with respect to acyclic cofibrations,
    it turns out that any fibrant complex must be degree-wise injective,
    and partially conversely,
    any bounded above complex of injectives is fibrant 
    (Lurie HA 1.3.5.6).
    $(\CH\,A)_f$ denotes the full subcategory of fibrant complexes.
    \item $N$ denotes the nerve functor which converts
    1-categories to simplicial sets, which have the property of being
    $\infty$-categories.
    $N_{dg}$ denotes the dg-nerve functor which achieves the same thing for
    honest-to-god dg-catgeory categories. 
    (See Kerodon 2.5.3 for a construction.)
    \item $h$ is the truncation of an infinity category to a 1-category
    by taking its homotopy catgeory.
    It is the left adjoint to $N$.
    (See Kerodon 1.2.5 for a construction.)

    We have that the homotopy category of $A\dash\MOD$ gives the
    usual derived category of $A$-modules, as in classical algebraic geometry.
    \item $L$ is a left adjoint to the inclusion 
    $N_{dg}((\CH\,A)_f) \subs N_{dg}(\CH\,A)$.
    Intuitively, for every complex $M_\bullet$, 
    there exists a acyclic cofibration $M_\bullet \to I_\bullet$ 
    to fibrant $I_\bullet$
    and this is initial in the category of arrows from 
    $M_\bullet$ into $N_{dg}((\CH\,A)_f)$
    (Lurie HA 1.3.5.12).
    This means for each $M_\bullet$, 
    such a morphism $M_\bullet \to I_\bullet$ is unique up to equivalence
    and assembles to the desired functor $L$.
    Practically speaking, $L(M_\bullet) \simeq I_\bullet$.
    \item The composition $N(\CH\, A) \to N_{dg}((\CH\, A)_f)$
    exhibits the latter as the $\infty$-categorical localisation 
    of the former at quasi-isomorphisms (Lurie HA 1.3.5.15).
    This matches the standard treatment in classical algebraic geometry : 
    the localisation functor from $\CH\, A$ to $D(A)$
    takes a complex and resolves it by injecting it
    quasi-isomorphically into a complex of injectives.
  \end{itemize}

\end{dfn}

\begin{prop}[Universal Property of Presheaf Categories (Lurie HTT 5.1.5.6)]

  Let $S$ be a small $\infty$-category and $C$ an $\infty$-category
  with small colimits. 
  Let $\mathrm{Fun}^L(\PSH\, S , C)$ denote the
  full subcategory of $\mathrm{Fun}(\PSH\, S, C)$ consisting of
  functors preserving small colimits.
  Then restricting along the Yoneda embedding $S \to \PSH\, S$
  gives an equivalence of $\infty$-categories : 
  \[
    \mathrm{Fun}^L(\PSH\, S , C) \map{\sim}{} \mathrm{Fun}(S , C)  
  \]
  An inverse functor is given by left Kan extension.
  In particular,
  for $F \in \mathrm{Fun}^L(\PSH\, S , C)$ corresponding to 
  $F_0 \in \mathrm{Fun}(S , C)$,
  we have for every $X \in \PSH\, S$ that
  $F$ exhibits $F(X)$ as the colimit of the diagram 
  $S_{/ X} \to S \to C$. 

\end{prop}

\end{document}