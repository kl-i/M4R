\documentclass[./main.tex]{subfiles}
\begin{document}

Let $X, Y \in \PSTK$.
The theory of integral transforms for quasi-coherent sheaves says that
under suitable conditions on $X, Y$,
there is an equivalence in $\DGCAT_\CTS$ : 
\[
  \QCOH(Y \times X) \simeq \DGCAT_\CTS(\QCOH\,X , \QCOH\,Y)
\]
which sends a sheaf $\QQ$ on $Y \times X$ to
the functor $(p_Y)_*(p_X^*(\_) \otimes \QQ)$.
For intuition, there is a nice analogy with linear algebra  : 
\begin{align*}
  &k \text{ base field} 
    & \VEC \\
  & \text{1-category of $k$-modules } k\MOD
    & \DGCAT_\CTS = \VEC\MOD \\
  & \SET
    & \PSTK \\
  & k^\_ = \SET(\_,k) : \SET \to (k\MOD)^\OP
    & \QCOH^* : \PSTK \to (\DGCAT_\CTS)^\OP \\
  &s, t \text{ finite sets} 
    & X , Y \text{ aft derived schemes} \\
  & t \leftarrow t\times s \rightarrow s
    & X \leftarrow X \times Y \rightarrow Y \\
  & k^s \to k^{t \times s} , 
  [v_j]_{j \in s} \mapsto [v_j]_{i , j \in t \times s}
    & p_X^* \\
  & k^{t \times s} \to k^t , 
  [A_{i , j}]_{i , j \in t \times s} 
  \mapsto \sqbrkt{\sum_{j \in s} A_{i , j}}_{i \in t}
    & (p_Y)_* \\
  &k^s \simeq k^{\oplus s}
    & \QCOH\,X \simeq \IND\,((\QCOH\,X)^c)\\
  &\text{pointwise multiplcation on } k^{t \times s} 
    & \otimes \text{ on } \QCOH(Y \times X) \\
  &k^{t \times s} \simeq k^s \otimes k^t
    &\QCOH(X \times Y) \simeq \QCOH\,X \otimes \QCOH\,Y \\
  &\<\_,\_\> : k^s \otimes k^s \to k
    &\<\_ , \_\> : \QCOH\,X \otimes \QCOH\,X \to \VEC \\
  &\id{} \otimes \<\_,\_\> : k^t \otimes k^s \otimes k^s \to k^t
    &\id{} \otimes \<\_,\_\> : 
    \QCOH\,Y \otimes \QCOH\,X \otimes \QCOH\,X \to \QCOH\,Y \\
  &k^s \otimes k^t \simeq \HOM_k(k^s , k^t)
    &\QCOH\,X \otimes \QCOH\,Y \simeq \DGCAT_\CTS(\QCOH\,X , \QCOH\,Y) \\
  &k^{t \times s} \simeq \HOM_k(k^s , k^t)
    &\QCOH(Y \times X) \simeq \DGCAT_\CTS(\QCOH\,X , \QCOH\,Y) \\
  &A \mapsto \brkt{v \mapsto \sqbrkt{\sum_{j \in s} A_{i,j} v_j}_{i \in t}}
    & \QQ \mapsto (p_2)_*(p_1^*(\_) \otimes \QQ)
\end{align*}
Following the analogy, the ansatz for the integral transform equivalence
is the following composition : 
\begin{cd}
  {\QCOH(Y\times X)} & {\QCOH\,X \otimes \QCOH\,Y} & {\DGCAT_\CTS(\QCOH\,X , \QCOH\,Y)}
	\arrow["\sim"',"{(1)}", from=1-2, to=1-1]
	\arrow["\sim","{(2)}"', from=1-2, to=1-3]
\end{cd}
where (2) should come from $\QCOH\,X$ being ``self-dual'' in a suitable sense
and (1) comes from a computation on ``basis elements''.

Let us first address the topic of duality.
There are many definitions of dualisability in a 
symmetric monoidal $\infty$-category.
We will take the following to be equivalent for granted.
\begin{prop}[Dualisable Objects]
  \link{duality.dualisable}
  
  Let $C$ be a symmetric monoidal $\infty$-category.
  Let $e : x^\vee \otimes x \to 1_C$ where $1_C$ is the unit.
  Then the following are equivalent : 
  \begin{enumerate}
    \item (duality datum) there exists $c : 1_C \to x \otimes x^\vee$
    such that we have the following commuting triangles in $hC$ : 
    \begin{cd}
      x & {x \otimes x^\vee \otimes x} & x^\vee \\
      & x & {x^\vee \otimes x \otimes x^\vee} & x^\vee
      \arrow["\id{}"', from=1-1, to=2-2]
      \arrow["{c \otimes \id{}}", from=1-1, to=1-2]
      \arrow["{\id{} \otimes e}", from=1-2, to=2-2]
      \arrow["{\id{} \otimes c}"', from=1-3, to=2-3]
      \arrow["{e \otimes \id{}}"', from=2-3, to=2-4]
      \arrow["{\id{}}", from=1-3, to=2-4]
    \end{cd}
    \item (Adjunction) for all $a \in C$, 
    the morphism $\id{} \otimes e : a \otimes x^\vee \otimes x \to a$
    induces an equivalence 
    \begin{cd}
      {C(b , a \otimes x^\vee)} & {C(b \otimes x , a \otimes x^\vee \otimes x)} \\
      & {C(b \otimes x , a)}
      \arrow["{\_ \otimes x}", from=1-1, to=1-2]
      \arrow["{(\id{} \otimes e)\,\_}", from=1-2, to=2-2]
      \arrow["\sim"', from=1-1, to=2-2]
    \end{cd}
    \item (Internal Hom) 
    Under the extra condition that $C$ has internal hom $\underline{\HOM}$,
    for any object $a \in C$,
    the morphism $e \otimes \id{} : a \otimes x^\vee \otimes x \to a$
    induces an equivalence \[
      a \otimes x^\vee \to \underline{\HOM}(x , a)
    \]

  \end{enumerate}
  \cite[Prop 4.6.1.6]{Lurie-HA}
  An object $x$ is called \emph{dualisable} when 
  there exists another object $x^\vee$ together with
  a morphisms $e : x^\vee \otimes x \to 1_C$ satisfying any
  (and thus all) of the above.

\end{prop}

\begin{rmk}
  In the definition of dualisable objects,
  we assumed $C$ is symmetric monoidal.
  This implies given a dualisable pair $e : x^\vee \otimes x \to 1_C$,
  $e_\mathrm{transposed} : x \otimes x^\vee \to 1_C$ is also a dualisable pair.
  In particular, $(x^\vee)^\vee \simeq x$.
\end{rmk}

\begin{dfn}
  
  Let $C \in \DGCAT_\CTS$.
  Then $C$ is called \emph{dualisable} when
  it is dualisable as an object in $\DGCAT_\CTS = \VEC\MOD$.
\end{dfn}

\begin{rmk}
  
  Let's unpack the above a bit.
  To say $C \in \DGCAT_\CTS$ is dualisable means
  there exists $C^\vee \in \DGCAT_\CTS$ and 
  $e \in \DGCAT_\CTS(C^\vee \otimes C, \VEC)$
  such that for all $D \in \DGCAT_\CTS$,
  we have an equivalence
  \begin{cd}
    {\DGCAT_\CTS(E , D \otimes C^\vee)} & {\DGCAT_\CTS(E \otimes C , D \otimes C^\vee \otimes C)} \\
    & {\DGCAT_\CTS(E \otimes C , D)}
    \arrow["{\_ \otimes C}", from=1-1, to=1-2]
    \arrow["{(\id{} \otimes e)\,\_}", from=1-2, to=2-2]
    \arrow["\sim"', from=1-1, to=2-2]
  \end{cd}
  In particular, applying $D = \VEC$, 
  gives $C^\vee \simeq \DGCAT_\CTS(C , \VEC)$.
  On the other hand, applying $E = \VEC$ and a transposition yields
  $C^\vee \otimes D \simeq \DGCAT_\CTS(C , D)$.

  The following shows that compact generation of $C$ is sufficient for
  being dualisable and in fact, the dual $C^\vee$ has a explicit description.
\end{rmk}

\begin{prop}[Duality for Compactly Generated DG $\infty$-Categories]
  \link{duality.cg}
  
  Let $C \in \DGCAT_\CTS$ be compactly generated.
  Consider the following : 
  \begin{cd}
    {(C^c)^\OP \times C^c} 
      & {\IND((C^c)^\OP) \times C} 
        & {\IND((C^c)^\OP) \otimes C} \\
    & \VEC
    \arrow[from=1-1, to=1-2]
    \arrow["{\HOM_C}"', from=1-1, to=2-2]
    \arrow[dashed, from=1-2, to=2-2]
    \arrow[from=1-2, to=1-3]
    \arrow["{e}", dashed, from=1-3, to=2-2]
  \end{cd}
  The enriched hom bi-functor $\HOM_C$ on compact objects
  is $\VEC$-linear and exact in both variables 
  hence the
  middle dashed morphism 
  \linkto{dgcat.ind.up}{by left Kan extension in each variable}/
  This is $\VEC$-linear and continuous in both variables,
  hence the right dashed morphism in $\DGCAT_\CTS$ by
  the \linkto{dgcat.dgcat.tensor}{universal property of tensors}. 
  Then $e$ exhibits $\IND((C^c)^\OP)$ as the dual of $C$.
  \cite[Ch 1, 7.3.2]{GR1}
\end{prop}
\begin{proof}
  For any $E \in \DGCAT_\CTS$, we have the following equivalences
  \begin{align*}
    \DGCAT_\CTS(\DGCAT_\CTS(C , D) , E)
    &\simeq \DGCAT_\CTS(\DGCAT(C^c , D) , E) \\
    &\simeq \DGCAT_\CTS(E^\OP , \DGCAT(C^c , D)^\OP) \\
    &\simeq \DGCAT_\CTS(E^\OP , \DGCAT((C^c)^\OP , D^\OP)) \\
    &\simeq \DGCAT((C^c)^\OP , \DGCAT_\CTS(E^\OP , D^\OP)) \\
    &\simeq \DGCAT((C^c)^\OP , \DGCAT_\CTS(D , E)) \\
    &\simeq \DGCAT_\CTS(\IND((C^c)^\OP) , \DGCAT_\CTS(D , E)) \\
    &\simeq \DGCAT_\CTS(\IND((C^c)^\OP) \otimes D , E)
  \end{align*}
  where we have used the following equivalence without proof : 
  \[
    \DGCAT_\CTS(\_ , \star) \map{\sim}{} \DGCAT_\CTS(\star^\OP , \_^\OP)
  \]
  where the forward direction is obtained by taking right adjoints via the
  \linkto{dgcat.adjoint}{adjoint functor theorem} and the right side is
  well-defined because \linkto{dgcat.presentable.has_lim}{
    presentable $\infty$-categories have small limits
  }.
\end{proof}

\begin{prop}[$\QCOH\,A$ Compactly Generated (Derived Affine Case)]
  \link{duality.daff}

  Let $\SPEC\,A \in \DAFF$ and $C := \QCOH\,A$.
  Then $C \simeq \IND\,C(\om)$ where
  $C(\om)$ is the smallest stable full subcategory of $C$ containing $A$.
  In particular, $C$ is compactly generated.
  
  Furthermore, for $M \in C$, the following are equivalent : 
  \begin{itemize}
    \item $M$ is a retract of objects in $C(\om)$.
    \item $M$ is compact
    \item $M$ is dualisable
  \end{itemize}

\end{prop}
\begin{proof}

  Let $C := \QCOH\,A$.
  One can check that $A$ is a compact object of $\QCOH\,A$ by
  using the \linkto{dgcat.stable.cg}{1-categorical characterisation of
  compact objects} and the fact that
  homotopy coproducts in $h\QCOH\,A$ coincide with
  taking degree-wise coproducts.
  Similarly, one can check that $A$ is a generator
  at the level of the 1-categorical derived category $h\QCOH\,A$.
  \linkto{dgcat.stable.cg}{Hence} $C \simeq \IND\,C(\om)$.
  We also deduce that \linkto{dgcat.idem}{$C(\om) \to C^c$ exhibits $C^c$
  as the idempotent completion of $C(\om)$}.
  Therefore compact objects coincide with retracts of objects in $C(\om)$.

  We now show an $M \in C$ is compact if and only if it is dualisable.
  This is a slightly different proof to \cite[Lem 3.4]{BZFN}.

  Suppose $M$ is compact.
  Our ansatz for $M^\vee$ is $\underline{\HOM}(M,1)$,
  where $\underline{\HOM} := \underline{\HOM}_S$ is the internal hom of
  $\QCOH\,S$.
  For $e : M \otimes \underline{\HOM}(M,1) \to 1$,
  we choose it to correspond to the identity morphism 
  $\underline{\HOM}(M,1) \to \underline{\HOM}(M,1)$ under the adjunction
  $M\otimes \_ \dashv \underline{\HOM}(M , \_)$.
  \linkto{duality.dualisable}{It remains to show}
  that the natural transformation 
  $\_ \otimes \underline{\HOM}(M , 1) \to \underline{\HOM}(M , \_)$
  coming from 
  $\id{} \otimes e : \_ \otimes M \otimes \underline{\HOM}(M , 1) \to \_$
  is an equivalence.
  By assumption, $\underline{\HOM}(M , \_)$ preserves small coproducts.
  \linkto{dgcat.stable.colimits}
  {Hence}, both $\underline{\HOM}(M , \_)$ and 
  $\_ \otimes \underline{\HOM}(M , 1)$ preserves small colimits.
  Since $\QCOH\,S =: C \simeq \IND\,C(\om)$,
  \linkto{dgcat.ind.up}{it suffices} to show that
  $\_ \otimes \underline{\HOM}(M , 1) \to \underline{\HOM}(M , \_)$
  is an equivalence on $C(\om)$.
  Since $C(\om)$ is obtained from $1 = \OO(S)$ by 
  \link{dgcat.stable.cg}{iteratively adding it cofibers} and 
  both functors preserve small colimits, it suffices to show that
  $1 \otimes \underline{\HOM}(M , 1) \to \underline{\HOM}(M , 1)$
  is an equivalence.
  This is clear.

  Now assume $M$ is dualisable.
  Then $\underline{\HOM}(M , \_) \simeq \_ \otimes M^\vee$
  and hence preserves small coproducts.
  Therefore $M$ is compact.

\end{proof}

\begin{prop}[$\QCOH\,X$ Compactly Generated (Derived Scheme Case)]
  \link{duality.dsch}

  Let $X \in \DSCH_\mathrm{qc}$. 
  Then $\QCOH\,X \simeq \IND\,\PERF\,X$ where
  $\FF$ is in the full subcategory $\PERF\,X$ if and only if
  any of the following equivalent conditions are true : 
  \begin{itemize}
    \item $\FF$ is dualisable
    \item $\FF$ is compact
    \item for all $x : S \to X$ where $S$ derived affine,
    $\FF_x \in \QCOH\,S$ is compact.
  \end{itemize}
\end{prop}
\begin{proof}
  
  We omit the proof of $\QCOH\,X \simeq \IND\,\PERF\,X$ for brevity.
  See \cite[Prop 3.19]{BZFN} for a proof.
  For $\FF \in \QCOH\,X$,
  the fact that $\FF$ is dualisable if and only if
  it is fiberwise dualisable is proved in \cite[Prop 3.6]{BZFN}.
  The equivalence of dualisability and compactness
  is shown in \cite[Prop 3.9]{BZFN}.
\end{proof}

We now show that $\QCOH\,X$ has a \emph{self-duality} when 
$X$ is a quasi-compact derived scheme.

\begin{prop}[Self Duality of $\QCOH\,X$]
  \link{duality.naive}

  Let $X \in \DSCH_\mathrm{qc}$.
  Define \[
    \<\_,\_\> : 
    \QCOH\,X \otimes \QCOH\,X \map{\otimes}{} \QCOH\,X \map{(p_X)_*}{} \VEC
  \]
  Then $\<\_,\_\>$ exhibits $\QCOH\,X$ as its own dual.
  Let $\D : \QCOH\,X \simeq (\QCOH\,X)^\vee$ correspond to $\<\_,\_\>$.
  We refer to $\D$ as the \emph{naive dualisation functor}.

\end{prop}
\begin{proof}
  This proved in \cite[Ch 1 , 9.2]{GR1} for general 
  stable rigid monoidal $\infty$-categories.
  What follows is an application of the general proof to this example.
  
  We know that the dual of $\QCOH\,X \simeq \IND\,\PERF\,X$
  is $\IND((\PERF\,X)^\OP)$.
  We thus compute the naive dualisation functor as : 
  \begin{align*}
    \QCOH\,X \map{}{} 
    \DGCAT_\CTS(\QCOH\,X , \VEC) \mapfrom{\sim}{}
    (\QCOH\,X)^\vee \\
    \FF \mapsto 
    \<\_ , \FF \> \simeq \HOM_X(\D \FF , \_)
    \mapsfrom \D \FF
  \end{align*}
  It suffices to show $\D$ is an equivalence.
  Since $\QCOH\,X$ is the ind-completion of $\PERF\,X$,
  \linkto{dgcat.cg.out}{it suffices} to show that
  $\D$ maps $\PERF\,X$ fully faithfully into 
  $(\PERF\,X)^\OP$ inside $(\QCOH\,X)^\vee$.

  Let $\FF \in \PERF\,X$.
  Then \[
    \HOM_X(\D \FF , \_) \simeq 
    \<\_ , \FF\> \simeq
    \HOM_X(\OO_X , \_ \otimes \FF) \simeq
    \HOM_X(\FF^\vee , \_) \simeq
    \HOM_X(\underline{\HOM}_X(\FF , \OO_X) , \_)
  \]
  Therefore on $\PERF\,X$, 
  the dualisation functor 
  $\D$ lands in the image under Yoneda of $(\PERF\,X)^\OP$.
  Identifying $(\PERF\,X)^\OP$ as a full subcategory of $(\QCOH\,X)^\vee$,
  we see that $\D(\_) \simeq \_^\vee \simeq \underline{\HOM}_X(\_ , \OO_X)$
  which is an equivalence $\PERF\,X \simeq (\PERF\,X)^\OP$
  because this is simply taking dual objects.

\end{proof}

This sorts out the equivalence (2) for $X \in \DSCH_\mathrm{qc}$.
We now address the equivalence (1).

\begin{prop}[Integral Transform for Quasi-Coherent Sheaves]
   \link{fm.qcoh}
  
  Let $X , Y \in \PSTK$.
  Suppose $\QCOH\,X$ is dualisable in $\DGCAT_\CTS$.
  Then we have an equivalence
  \begin{cd}
    \boxtimes & {\QCOH\,X \otimes \QCOH\,Y} & {\QCOH(Y\times X)} \\
    & {\QCOH\,X \times \QCOH\,Y}
    \arrow["\sim", from=1-2, to=1-3]
    \arrow[from=2-2, to=1-2]
    \arrow["{p_X^*(\_) \otimes p_Y^*(\_)}"'{pos=1}, from=2-2, to=1-3]
    \arrow["{:}"{description}, draw=none, from=1-1, to=1-2]
  \end{cd}
  \cite[Ch 3, 3.1.7]{GR1}
\end{prop}
\begin{proof}
  % (2) We defer the reader to the reference.

  What follows is the proof of \cite[Ch 3 , 3.1.7]{GR1}
  completed with details from \cite{BZFN}.

  By the \linkto{dgcat.dgcat.tensor}
  {universal property of the tensor product in $\DGCAT_\CTS$},
  it suffices to give an abstract equivalence 
  $\QCOH\,X \otimes \QCOH\,Y \simeq \QCOH(Y \times X)$ 
  that restricts to $p_X^* \otimes p_Y^*$ on $\QCOH\,X \times \QCOH\,Y$.

  Let us show that it suffices to show
  for any derived affines $S, T$, we have that
  $\boxtimes : \QCOH\,S \otimes \QCOH\,T \to \QCOH(T \times S)$ is 
  an equivalence.
  Assume this. Then we have the following chain of equivalences : 
  \begin{align*}
    &\QCOH\,X \otimes \QCOH\,Y \\
    &\map{\sim}{} \QCOH\,X \otimes \brkt{\LIM_{T \in \DAFF / Y} \QCOH\,T}
      &\text{by $\QCOH^*$ left Kan extension} \\
    &\map{\sim}{} \LIM_{T \in \DAFF / Y} \QCOH\,X \otimes \QCOH\,T
      &\text{by $\QCOH\,X$ dualisable} \\
    &\map{\sim}{} \LIM_{T \in \DAFF / Y} 
    \brkt{\LIM_{S \in \DAFF / X} \QCOH\,S} \otimes \QCOH\,T
      &\text{by $\QCOH^*$ left Kan extension} \\
    &\map{\sim}{} \LIM_{T \in \DAFF / Y} 
    \LIM_{S \in \DAFF / X} \QCOH\,S \otimes \QCOH\,T 
      &\text{by $\QCOH\,T$ dualisable} \\
    &\map{\sim}{} \LIM_{(T , S) \in \DAFF / Y \times \DAFF / X}
    \QCOH(S \times T)
      &\text{by assumption and limits commute with limits} \\
    &\mapfrom{\sim}{} \LIM_{R \in \DAFF / Y \times X} \QCOH\,R
      &\text{by $\DAFF / Y \times \DAFF / X \to \DAFF / Y \times X$ cofinal}\\
    &\mapfrom{\sim}{} \QCOH(Y \times X)
      &\text{by $\QCOH^*$ left Kan extension}
  \end{align*}
  The fact that restricting along $\otimes : \QCOH\,X \times \QCOH\,Y \to 
  \QCOH\,X \otimes \QCOH\,Y$ produces $p_X^*(\_) \otimes p_Y^*(\_)$
  comes from its definition, which amounts to the following
  commuting diagram : 
  \begin{cd}
    {\QCOH\,X \times \QCOH\,Y} & {\QCOH(Y\times X)} \\
    {\LIM_{(T , S) \in \DAFF/Y \times \DAFF / X} \QCOH\,S \times \QCOH\,T} 
      & {\LIM_{T \times S \in \DAFF / Y \times X} \QCOH\,(T \times S)}
    \arrow["{p_X^*(\_) \otimes p_Y^*(\_)}", from=1-1, to=1-2]
    \arrow["\sim"', from=1-1, to=2-1]
    \arrow["\sim", from=1-2, to=2-2]
    \arrow[from=2-1, to=2-2, "{\LIM\,p_S^*(\_)\otimes p_T^*(\_)}"'{yshift = -3}]
  \end{cd}
  We thus reduce to the special case of derived affines.
  We need to compute with the tensor product of dg-$\infty$-categories.
  \begin{lem}[GR1 Ch1 7.4.2]
    Let $C , D \in \DGCAT_\CTS$ with compactly generating sets of objects
    $C_0 \subs C$ and $D_0 \subs D$.
    Then the set $\set{c_0 \otimes d_0 \st c_0 \in C_0, d_0 \in D_0}$ 
    compactly generates $C \otimes D$.
    Furthermore, 
    for $c_0, d_0 \in C_0, D_0$ and $c , d \in C , D$,
    we have an equivalence \[
      \HOM_C(c_0 , c) \otimes \HOM_D(d_0 , d) \simeq
      \HOM_{C \otimes D}(c_0 \otimes d_0 , c \otimes d)  
    \]
    \cite[Ch 1 , 7.4.2]{GR1}
    \begin{proof1}
      $C \simeq \IND\,C_0$ \linkto{duality.cg}{implies}
      $C^\vee := \IND(C_0^\OP)$ is a dual for $C$.
      Then we have \[
        C \otimes D \simeq \DGCAT_\CTS(C^\vee , D)
        \simeq \DGCAT(C_0^\OP , \DGCAT(D_0^\OP , \VEC)) 
        \simeq \FUN^\mathrm{Bi-Ex}_\VEC(C_0^\OP \times D_0^\OP , \VEC)
      \]
      where the latter is the dg-$\infty$-category of
      functors $C_0^\OP \times D_0^\OP$ which are $\VEC$-linear and exact
      in each variable.
      Under these equivalences, 
      the functor $\boxtimes : C \times D \to C \otimes D$ corresponds to
      \begin{align*}
        C \times D &\map{}{} 
        \DGCAT(C_0^\OP , \VEC) \times \DGCAT(D_0^\OP , \VEC) 
        \,\,\,\,\text{enriched Yoneda}\\
        &\map{\sim}{}
        \FUN^\mathrm{Bi-Ex}_\VEC(C_0^\OP \times D_0^\OP , \VEC \times \VEC)\\
        &\map{}{}
        \FUN^\mathrm{Bi-Ex}_\VEC(C_0^\OP \times D_0^\OP , \VEC)
        \,\,\,\,\text{compose with }\otimes
      \end{align*}
      Let $h$ the above composition.
      Let $\HOM$ denote the enriched hom of 
      $\FUN^\mathrm{Bi-Ex}_\VEC(C_0^\OP \times D_0^\OP , \VEC)$.
      Now applying the tensor-hom adjunction in $\VEC$ with
      two applications of enriched Yoneda's lemma shows that
      for any $F \in 
      \FUN^\mathrm{Bi-Ex}_\VEC(C_0^\OP \times D_0^\OP , \VEC)$,
      and $(c_0, d_0) \in C_0 \times D_0$, we have \[
        \HOM(h(c_0 , d_0) , F) \map{\sim}{} F(c_0 , d_0) \text{ in }\VEC
      \]
      Since $h$ corresponds to $\boxtimes$,
      this shows that $\set{c_0 \otimes d_0 \st c_0 \in C_0 , d_0 \in D_0}$
      compactly generates $C \otimes D$.

      Finally, for formula,
      $c \otimes d$ and $c_0 \otimes d_0$ transported across
      the equivalences yield $h(c , d)$ and $h(c_0 , d_0)$.
      So we have \[
        \HOM_{C \otimes D}(c_0 \otimes d_0 , c \otimes d)
        \simeq \HOM(h(c_0 , d_0) , h(c , d))
        \simeq \HOM_C(c_0 , c) \otimes \HOM_D(d_0 , d)
      \]
      as desired.

    \end{proof1}
  \end{lem}
  Finally, we are ready.
  \begin{lem}[Integral Transforms for Quasi-Coherent Sheaves on Derived Affines]
    \link{fm.qcoh.daff}
    For $S , T \in \DAFF$, 
    we have the equivalence $\boxtimes : 
    \QCOH\,S \otimes \QCOH\,T \map{\sim}{} \QCOH(S \times T)$.
    \begin{proof1}
      We follow the approach of \cite[Prop 4.6]{BZFN}.
      By the \linkto{dgcat.stable.cg}{
        theory of compactly generated stable $\infty$-categories
      },
      it suffices to show that $\QCOH\,S \otimes \QCOH\,T$
      has a small compactly generating full subcategory that
      gets sent fully faithfully to a compactly generating full subcategory
      of $\QCOH(S \otimes T)$.
      Of course, we choose to show that
      \begin{enumerate}
        \item $\set{M \boxtimes N \in \QCOH(S \times T) \st 
        M \in (\QCOH\,S)^c , N \in (\QCOH\,T)^c}$ compactly generates
        $\QCOH(S \times T)$
        \item for all $M, M_1 \in (\QCOH\,S)^c$ and $N , N_1 \in (\QCOH\,T)^c$,
        we have \[
          \HOM_S(M , M_1) \otimes \HOM_T(N , N_1) \map{\sim}{}
          \HOM_{S \times T}(M \boxtimes N , M_1 \boxtimes N_1)  
        \]
      \end{enumerate}
      For (1), we use that fact that 
      for $M \in \QCOH\,S$ where $S$ is derived affine, 
      \linkto{duality.daff}{$M$ is compact if and only if it is dualisable}.
      Therefore, if $M, N \in \QCOH\,S , \QCOH\,T$ are compact,
      then they are dualisable.
      Since for any morphism $f$ of derived affines, 
      $f^*$ is \linkto{qcoh.symm_mon}{symmetric monoidal},
      we obtain that $p_S^*(M), p_T^*(N)$ are both dualisable in
      $\QCOH(S \times T)$.
      It is not hard to show that the tensor product of two dualisable objects
      is again dualisable, so we obtain $M \boxtimes N$ is dualisable,
      and hence compact in $\QCOH(S \times T)$.

      Now we show that $\set{M \boxtimes N \st M, N \text{ compact } \in 
      \QCOH\,S , \QCOH\,T}$ generates $\QCOH(S \times T)$.
      What follows is essentially a series of definitions.
      Let $P \in \QCOH(S \times Y)$ and assume that
      \[
        \pi_0 \HOM_{S\times T}((\QCOH\,S)^c \boxtimes (\QCOH\,T)^c , P) \simeq 0
      \]
      Then for any $N \in (\QCOH\,T)^c$, we have 
      \[
        0 \simeq \pi_0\HOM_{S \times T}((\QCOH\,S)^c \boxtimes N , P)
        \simeq \pi_0\HOM_S((\QCOH\,S)^c , 
          (p_S)_*\underline{\HOM}_{S \times T}(p_T^* N , P))
      \]
      Since $(\QCOH\,S)^c$ compactly generates $\QCOH\,S$,
      we obtain $0 \simeq (p_S)_*\underline{\HOM}_{S \times T}(p_T^* N , P)$ 
      for any $N \in (\QCOH\,T)^c$.

      We are now in the world of algebra.
      For $N \in (\QCOH\,T)^c$,
      $0 \simeq (p_S)_*\underline{\HOM}_{S \times T}(p_T^* N , P)$ says
      that the $\OO(S \times T)$-module 
      $\underline{\HOM}_{S \times T}(p_T^* N , P)$ is zero
      after applying the forgetful functor $\OO(S\times T)\MOD \to \OO(S)\MOD$.
      Thus \begin{align*}
        0 &\simeq \underline{\HOM}_{S \times T}(p_T^* N , P) \\
        &\Rightarrow
        0 \simeq \HOM_{S \times T}(\OO(S \times T) , 
          \underline{\HOM}_{S \times T}(p_T^* N , P))
          \simeq \HOM_{S \times T}(p_T^*N , P)
          \simeq \HOM_{T}(N , (p_T)_* P)
      \end{align*}
      This holds for all $N \in (\QCOH\,T)^c$.
      So by compact generation of $\QCOH\,T$,
      we see that $0 \simeq (p_T)_* P$.
      This just says $0 \simeq P$ as an $\OO(T)$-module.
      This clearly implies $0 \simeq P$ as an $\OO(S \times T)$-module,
      finishing the proof of (1).
      
      (2) \begin{align*}
        &\,\,\,\,\HOM_{S \times T}(M \boxtimes N , M_1 \boxtimes N_1) \\
        \simeq &\,\,\,\, \Ga(S \times T , 
        (p_S^*(M) \otimes p_T^*(N))^\vee \otimes p_S^*(M_1) \otimes p_T^*(N_1))
        &\text{ tensor of dualisables is dualisable }
        \\
        \simeq &\,\,\,\, \Ga(S \times T , 
        p_S^*(M^\vee) \otimes p_T^*(N^\vee) \otimes p_S^*(M_1) \otimes p_T^*(N_1))
        &\text{ $p^*$ preserve dualisables }
        \\
        \simeq &\,\,\,\, \Ga\brkt{S \times T , 
        p_S^*(M^\vee \otimes M_1) \otimes p_T^*(N^\vee \otimes N_1)}
        &\text{ $p^*$ symmetric monoid }
        \\
        \simeq &\,\,\,\, \Ga\brkt{S \times T , 
        p_S^*(\underline{\HOM}_S(M , M_1) 
        \otimes p_T^*(\underline{\HOM}_T(N,N_1)))}
        &\text{ internal hom and dualisable }
        \\
        \simeq &\,\,\,\, \Ga\brkt{T , (p_T)_*\brkt{
          p_S^*(\underline{\HOM}_S(M , M_1) 
        \otimes p_T^*(\underline{\HOM}_T(N,N_1)))
        }}
        &\text{ ${p_T}_*$ definition }
        \\
        \simeq &\,\,\,\, \Ga\brkt{
          T , (p_T)_* p_S^* (\underline{\HOM}_S(M , M_1))
        \otimes \underline{\HOM}_T(N,N_1)
        }
        &\text{ projection formula for $S \times T \to T$ }
        \\
        \simeq &\,\,\,\, \HOM_S(M , M_1) \otimes \HOM_T(N , N_1)
        &\text{ ??? }
      \end{align*}
      The special case of projection formula for $S \times T \to T$
      can be deduced as follows.
      We wish to show that for all $M \in \QCOH(S \times T)$, 
      we have an equivalence : 
      \[
        ((p_T)_* M) \otimes \_ \map{\sim}{} (p_T)_*\brkt{M \otimes p_T^* \_}
      \]
      as endo-functors of $\QCOH\,T$.
      Let $C := \QCOH\,T$. 
      We have seen that $C \simeq \IND\,C(\om)$ where
      $C(\om)$ is the smallest stable full subcategory of $C$ containing
      $\OO(T)$.
      \linkto{dgcat.ind.up}{It suffices}
      that the above two functors are continuous and
      that they agree on $C(\om)$.
      Since $\otimes$ is continuous in each variable,
      $((p_T)_* M) \otimes \_$ is continuous.
      Continuity of $(p_T)_*\brkt{M \otimes p_T^* \_}$
      follows from continuity of $p_T^*, M \otimes\_ \,, (p_T)_*$
      where the last one uses our specific situation $S \times T \to T$
      of derived affines.
      Now, to show the two functors agree on $C(\om)$,
      \linkto{dgcat.stable.cg}{recall} that $C(\om)$ is obtained from
      $\OO(T)$ by first taking all shifts, then iteratively 
      adding in cofibers.
      Since the two functors preserve shifts and cofibers,
      it suffices that they agree on $\OO(T)$.
      This is now clear.
    \end{proof1}
  \end{lem}

\end{proof}

\end{document}