\documentclass[./main.tex]{subfiles}
\begin{document}
  
The main tool is the theory of descent as formulated in
\cite[Prop 4.7.5.2]{Lurie-HA}.

\begin{prop}[Descent]
  \link{induction.desc}
  
  Let $C^\bullet : \Delta \to (\infty,1)\dash\CAT$. 
  Suppose $C^\bullet$ has the following property : 
  \begin{itemize}
    \item for all $\al : [m] \to [n]$ in $\De$, 
    then the following commutative square is \emph{left-adjointable} :
     \begin{cd}
      {C^m} & {C^{m+1}} & \rightsquigarrow & {C^m} & {C^{m+1}} \\
      {C^n} & {C^{n+1}} && {C^n} & {C^{n+1}}
      \arrow[from=1-1, to=2-1]
      \arrow[from=1-2, to=2-2]
      \arrow["{d_0}", from=1-1, to=1-2]
      \arrow["{d_0}"', from=2-1, to=2-2]
      \arrow[from=1-4, to=2-4]
      \arrow[from=1-5, to=2-5]
      \arrow["{\de_0}", from=2-5, to=2-4]
      \arrow["{\de_0}"', from=1-5, to=1-4]
     \end{cd}
     meaning the coface morphisms $d^0 : C^k \to C^{k+1}$ admit 
     left adjoints $F(k) : C^{k+1} \to C^k$ and
     the commutative square on the left hand side
     induces the commutative square on the right by passing to left adjoints.
  \end{itemize}

  Let $C$ be the underlying $\infty$-category of the limit $\LIM C^\bullet$.
  Then \begin{itemize}
    \item the functor $G : C \to C^0$ admits a left adjoint $F$.
    \item the following commutative square is left-adjointable : 
    \begin{cd}
      C & {C^0} & \rightsquigarrow & C & {C^0} \\
      {C^0} & {C^1} && {C^0} & {C^1}
      \arrow["G"', from=1-1, to=2-1]
      \arrow["G", from=1-1, to=1-2]
      \arrow["{d^0}"', from=2-1, to=2-2]
      \arrow["{d^1}", from=1-2, to=2-2]
      \arrow["G"', from=1-4, to=2-4]
      \arrow["{d^1}", from=1-5, to=2-5]
      \arrow["F"', from=1-5, to=1-4]
      \arrow["{F(0)}", from=2-5, to=2-4]
    \end{cd}
    \item the adjunction $F \dashv G : C^0 \rightleftarrows C$ is monadic.
  \end{itemize}
  \cite[Prop 4.7.5.2]{Lurie-HA}
\end{prop}

Under the simplifying assumption of 
$X \in \PSTK_\LAFT$ being classically formally smooth so that
\[
  \CRYS^R\,X \map{\sim}{} \LIM \INDCOH\brkt{\check{C}(X / X_\DR)}
\]
we see that to obtain the induction functor for right crystals, 
we want the projections 
$p_1, p_2 : X \times_{X_\DR} X \rightrightarrows X$ to give
$p_1^! , p_2^!$ which admit left adjoints and satisfies base change.
This is indeed possible and is what \cite[Ch 3, 2.1]{GR2} achieves.

\begin{prop}[Base Change for $\INDCOH$ on Ind-Schematic, Ind-Proper Morphisms]
  
  Suppose we have the following cartesian square in $\PSTK_\LAFT$ : 
  \begin{cd}
    W & Y \\
    X & Z
    \arrow["v"{description}, from=1-1, to=2-1]
    \arrow["g"{description}, from=1-2, to=2-2]
    \arrow["f"{description}, from=2-1, to=2-2]
    \arrow["u"{description}, from=1-1, to=1-2]
    \arrow["\lrcorner"{anchor=center, pos=0.125}, draw=none, from=1-1, to=2-2]
  \end{cd}
  where $g$ is \emph{ind-schematic} and \emph{ind-proper}.
  Then $g^!, v^!$ respectively admit left adjoints $g_* , v_*$ such that
  we have an equivalence \[
    v_* u^! \map{\sim}{} f^! g_*
  \]
  arising by adjunction from the equivalence $f_* v_* \simeq g_* u_*$.
\end{prop}
\begin{proof}
  Omitted since this would require developing
  the theory of ind-schemes for derived algebraic geometry.
  See \cite[Ch 3, 2.1]{GR2} for the proof.
\end{proof}

Due to the constraint in space,
we also choose to omit the verification that
the projections $p_1, p_2 : X \times_{X_\DR} X \rightrightarrows X$ are 
ind-schematic and ind-proper.
We defer the reader to the definitions at 
\cite[Ch 2, 1.6.5]{GR2} and \cite[Ch 2, 1.6.11]{GR2}.
We thus obtain induction for right crystals.

\begin{prop}[Induction for Right Crystals (Crys 3.3.2)]
  \link{induction.right}
  
  Let $X \in \PSTK_\LAFT$ such that
  \[
    \CRYS^R\,X \map{\sim}{} \LIM \INDCOH\brkt{\check{C}(X / X_\DR)}
  \]
  (such as $X$ classically formally smooth).
  Consider $p : X \to X_\DR$ and 
  $p_s, p_t : X \times_{X_\DR} X \rightrightarrows X$.
  Then \begin{enumerate}
    \item the functor $\OBLV^R = p^! : \CRYS^R\,X \to \INDCOH\,X$
    has a left adjoint $\INDUCE^R$.
    \item we have an equivalence of functors \[
      \OBLV^R\,\INDUCE^R \simeq (p_t)_*\,p_s^!
    \]
    \item the adjunction $\INDUCE^R \dashv \OBLV^R$ is monadic.
    In particular, if $\INDCOH\,X$ is compactly generated,
    so is $\CRYS^R\,X$.\cite[Cor 3.3.3]{Crys}
  \end{enumerate}
  \cite[Prop 3.3.2]{Crys}
\end{prop}
\begin{proof}
  We've already discussed how (1) and (2) follow from
  \linkto{induction.desc}{Lurie's theory of descent}.

  For (3), note that since $\INDUCE^R \dashv \OBLV^R$ is monadic,
  $\OBLV^R$ must be conservative.
  This implies that the image of the set of objects of $(\INDCOH\,X)^c$
  compactly generates $\CRYS^R\,X$.
\end{proof}

The induction functor for left crystals 
$\INDUCE^L : \QCOH\,X \to \CRYS^L\,X$ is obtained by 
transferring across the equivalence 
$\Upsilon_{X_\DR} : \CRYS^L\,X \simeq \CRYS^R\,X$.

\end{document}