\documentclass[./main.tex]{subfiles}
\begin{document}
  
We are now ready for compactly generated $\infty$-categories.
We will only make use of the case of 
\emph{stable} compactly generated $\infty$-categories
since many definitions then admit alternative characterisations
which can be checked at the level of triangulated categories.

We first note that the theory of colimits
simplifies in the stable case.
\begin{prop}
  \link{dgcat.stable.colimits}
  
  \begin{itemize}
    \item For a stable $\infty$-category $C$, TFAE: 
    \begin{itemize}
      \item admitting small colimits
      \item admitting small filtered colimits
      \item admitting small coproducts
    \end{itemize}
    \item For a functor $F : C \to D$ between stable $\infty$-categories
    which admit small colimits, TFAE: 
    \begin{itemize}
      \item preserving small colimits
      \item preserving small filtered colimits
      \item preserving small coproducts
    \end{itemize}
    Any functor satisfying the above, GR calls \emph{continuous}.
  \end{itemize}
  \cite[Prop 1.4.4.1]{Lurie-HA}
\end{prop}

We now explain compact generation.
The starting point is the theory of \emph{inductive completions}\footnote{
  This is a bit of a misnomer because 
  intuitively we are adding filtered \emph{colimits}, not limits.
}.
Here are the main results concerning ind-completions in the stable case.
\begin{prop}[Ind-completions of Stable $\infty$-Categories]
  \link{dgcat.ind.up}

  Let $C$ be a small $\infty$-category and $\ka$ a regular cardinal.\footnote{
    A regular cardinal $\ka$ is a cardinality that is 
    ``sufficiently large'' in the sense that
    the 1-category $\SET_{<\ka}$ of sets with cardinality strictly less than
    $\ka$ has all colimits of size strictly less than $\ka$.
    The cardinality of $\N$ is an example, 
    since a finite colimit of finite sets is still finite.
  }
  Then the Yoneda embedding $C \to \PSH\,C$ factors through a full subcategory
  $\IND_\ka(C)$ with the following properties : 
  \begin{itemize}
    \item \cite[Prop 5.3.5.3]{Lurie-HTT} 
    $\IND_\ka(C)$ has all small $\ka$-filtered colimits
    and the inclusion $\IND_\ka(C) \subs \PSH\, C$ preserves them
    \item \cite[Prop 5.3.5.4]{Lurie-HTT}
    An object $X$ in $\PSH\,C$ is in $\IND_\ka(C)$ iff
    it is a $\ka$-filtered colimit of representables iff
    $X : C\op \to \SPC$ preserves $\ka$-small limits.
    \item \cite[Prop 1.1.3.6]{Lurie-HA} 
    If $C$ is stable then so is $\IND_\ka(C)$.
    \item \cite[Prop 5.3.5.10]{Lurie-HTT} For any $\infty$-category $D$ 
    admitting small $\ka$-filtered colimits,
    we have the following equivalence of functor $\infty$-categories : 
    \[
      \FUN_\ka(\IND_\ka(C) , D) \map{\sim}{} \FUN(C , D)
    \]
    where \begin{itemize}
      \item the left category denotes the full subcategory of 
      $\FUN(\IND_\ka(C),D)$ consisting of functors preserving 
      $\ka$-filtered colimits.\footnote{
        At \cite[Prop 5.3.4.5]{Lurie-HTT}, 
        these are called $\ka$-continuous functors.
        Taking the minimal case of $\ka = \abs{\N}$,
        it seems only reasonable to refer to 
        functors preserving filtered colimits as
        \emph{continuous} functors.
        This is a potential explanation of GR's choice of terminology
        for continuous functors.
      }
      \item the forward functor is given by restricting along the Yoneda embedding 
      $C \to \IND_\ka(C)$
      \item the inverse functor is given by left Kan extension.
    \end{itemize} 
    Assuming $C, D$ are stable and $\ka$ is the cardinality of $\N$, 
    the above equivalence restricts to
    an equivalence between the following two full subcategories :
    \[
      \FUN_\CTS^\EX(\IND_\ka(C) , D) \map{\sim}{} \FUN^\EX(C , D)
    \]
    where the left is the $\infty$-category of
    exact continuous functors from $\IND_\ka(C)$ to $D$.

  \end{itemize}

  When $\ka$ is the cardinality of $\N$,
  we write $\IND$ instead of $\IND_\ka$.
\end{prop}

For a regular cardinal $\ka$ and an $\infty$-category $C$,
we say $C$ is $\ka$-compactly generated when
it has all small colimits and there exists a small $\infty$-category
$C^0$ with an equivalence $\IND_\ka(C^0) \map{\sim}{} C$.\footnote{
  This is unraveled from 
  \cite[Def 5.5.7.1]{Lurie-HTT}
  \cite[Def 5.5.0.18]{Lurie-HTT}
  \cite[Def 5.4.2.1]{Lurie-HTT}.
  In particular, the second condition is usually called 
  $\ka$-accessibility, but we have no need for such terminology.
}
For the case of $\ka =$ cardinality of $\N$,
we simply say \emph{compactly generated}.
A \emph{presentable} $\infty$-category is one that is
$\ka$-compactly generated for some $\ka$.
We use $(\infty,1)\dash\CAT^\EX_\CTS$ to denote
the subcategory of $(\infty,1)\dash\CAT^\EX$ whose objects are
presented stable $\infty$-categories and morphisms are
exact functors preserving small coproducts.

One appeal of presentable stable $\infty$-categories is that
we have the adjoint functor theorem at our disposal.
\begin{prop}[Adjoint Functor Theorem for Presentable Stable $\infty$-Categories]
  \link{dgcat.adjoint}
  
  Let $F : C \to D$ be a functor between presentable $\infty$-categories.
  \begin{itemize}
    \item $F$ is a left adjoint iff it preserves small colimits.
    \item $F$ is a right adjoint iff it preserves small limits and
    there exists a regular cardinal $\ka$ such that
    $C$ has small $\ka$-filtered colimits and $F$ preserves them.
  \end{itemize}
  \cite[Prop 5.5.2.9]{Lurie-HTT}
\end{prop}
We also have a way of detecting when a right adjoint is continuous.
\begin{prop}[Continuity of Right Adjoint]
  \link{dgcat.radj.cts}
  
  Let $F \dashv G : C \rightleftarrows D$ be an adjunction between 
  compactly generated stable $\infty$-categories.
  Then $G$ is continuous if and only if $F$ preserve compact objects.
\end{prop}
\begin{proof}
  Direct application of \cite[Prop 5.5.7.2]{Lurie-HTT}.
  We saw the idea of the proof in \ref{why.indcoh}.
\end{proof}

For functors out of $\IND_\ka(C)$,
fully faithfulness and equivalence can be detected at the level of $C$.
\begin{prop}[Functors out of $\ka$-Compactly Generated Categories]
  \link{dgcat.cg.out}
  
  Let $C^0$ be a small $\infty$-category, $\ka$ a regular cardinal,
  $C = \IND_\ka(C^0)$ and
  $D$ an $\infty$-category admitting $\ka$-filtered colimits.
  Let $D^\ka$ be the full subcategory of $D$ consisting of 
  $\ka$-compact objects.
  Let $F : C \to D$ be a functor preserving $\ka$-filtered colimits
  and $F_0 : C^0 \to D$ its restriction along 
  the Yoneda embedding $C^0 \to C$.
  \begin{itemize}
    \item If $F_0$ is fully faithful and its essential image lands in $D^\ka$,
    then $F$ is fully faithful.
    \item $F$ is an equivalence iff the following are true :
    \begin{itemize}
      \item $F_0$ fully faithful
      \item $F_0$ factors through $D^\ka$
      \item all objects of $D$ are $\ka$-filtered colimits of diagrams
      in $D$ with objects in the image of $F_0$.
    \end{itemize}
    In particular,
    for any full subcategory $\tilde{C} \subs C^\ka$ which
    generates $C$ under $\ka$-filtered colimits,
    we have $\IND_\ka(\tilde{C}) \map{\sim}{} C$.
  \end{itemize}
  \cite[Prop 5.3.5.11]{Lurie-HTT}
\end{prop}

Furthermore, 
the Yoneda embedding $C^0 \to C$ factors through $C^\ka$.
The fully faithful functor $C^0 \to C^\ka$ is not in general an equivalence,
however it does exhibit $C^\ka$ as the \emph{idempotent completion}
of $C^0$. (See in next proposition.)
So if $C^0$ is idempotent complete, 
then we recover the $\ka$-compact objects of $C$ as precisely 
(the essential image of) $C^0$.

There is a complication with defining idempotents in an $\infty$-category $C$.
All we need to know is that 
they are diagrams $\mathrm{Idem} \to C$ from a certain ``shape''\footnote{
  meaning simplicial set, if we choose to use the quasi-categories
  model for $\infty$-categories. 
}
and $C$ is called idempotent complete when
all such diagrams admit a colimit.

The following propositions contain all we need about idempotent completions.
One should not worry about having to check idempotent completeness
$\infty$-categorically because we will only be in the stable setting,
in which we have a characterisation at the level of triangulated categories.
\begin{prop}[Idempotent Completions]
  \link{dgcat.idem}

  Let $C$ be an $\infty$-category.
  \begin{itemize}
    \item If $C$ has small colimits, then $C$ is idempotent complete.
    \item (\cite[Def 5.1.4.1, Prop 5.1.4.9, Lem 5.1.4.7]{Lurie-HTT})
    A functor $F : C \to D$ in $(\infty,1)\dash\CAT$ is said to
    \emph{exhibit $D$ as the idempotent completion of $C$} when 
    $F$ is fully faithful, $D$ is idempotent complete and every
    object of $D$ is a retract of some $F(x)$ with $x \in C$.

    Suppose $F : C \to D$ is such a functor.
    Then for any idempotent complete $\infty$-category $E$,
    we have an equivalence : 
    \[
      \FUN(D , E) \map{\sim}{} \FUN(C , E)
    \]
    given by restriction along $F$.
    An inverse functor is given by left Kan extension along $F$.
    \item \cite[Prop 5.4.2.4]{Lurie-HTT} 
    Let $C$ be small and $\ka$ a regular cardinal.
    Then $C \to (\IND\,C)^\ka$ exhibits the latter $\infty$-category as
    the idempotent completion of $C$.
    In particular, the latter is equivalent to a small $\infty$-category.
    \item \cite[Prop 1.2.4.6]{Lurie-HA} Let $C$ be a stable $\infty$-category.
    Then $C$ is idempotent complete iff $hC$ is as a 1-category,
    i.e. for every morphism $e : B \to B$ such that $e^2 = e$,
    there exists a retract of $s : A \rightleftarrows B : r$
    that exhibits $e = r s$.

    If the above is the case and $C$ is small,
    then for any regular cardinal $\ka$ we have
    $C \map{\sim}{} (\IND_\ka(C))^\ka$.
  \end{itemize}
\end{prop}

\textbf{\emph{But how do I actually get my hands on a presentable
stable $\infty$-category?}}
Worry not! All the computable examples
are in the compactly generated case.
As it turns out, this can also be checked at the level of 
the triangulated categories.
\begin{prop}[Compact Generation of Stable Infinity Categories]
  \link{dgcat.stable.cg}
  
  Let $C$ be a stable $\infty$-category.
  We say an object $X$ \emph{generates $C$} when
  for all objects $Y$ in $C$, 
  $hC (X,Y) = 0$ implies $Y \simeq 0$.
  Additionally, we say a set $I$ of objects in $C$ 
  \emph{compactly generates $C$} when $I$ consists of compact objects and
  $X := \bigoplus_{X_i \in I} X_i$ generates $C$.

  Then the following are true : 
  \begin{enumerate}
    \item Suppose we have that :
    \begin{itemize}
      \item $C$ has small coproducts
      \item $hC$ is locally small
      \item There exists a compact object $X$ which generates $C$.
    \end{itemize}
    Define the following sequence of full subcategories of $C$ : 
    \begin{align*}
      C(0) &:= 
        \text{full subcategory of $C$ spanned by $\set{X[n]}_{n \in\Z}$} \\
      C(k + 1) &:= 
        \text{full subcategory of $C$ spanned by 
        cofibers of morphisms in $C(k)$} \\
      C(\om) &:= \bigcup_{n \in \N} C(n) 
    \end{align*}
    Then \begin{enumerate}
      \item $C(\om)$ is the smallest stable full subcategory of $C$
      containing $X$ and is equivalent to a small $\infty$-category.
      \item $\IND\,C(\om) \map{\sim}{} C$.
      In particular, $C$ is compactly generated.
    \end{enumerate}
    \item \cite[Prop 1.4.4.1]{Lurie-HA} For an object $X$ in $C$,
    $X$ is compact if and only if for every morphism 
    $f : X \to \coprod_{i \in I} Y_i$ in $C$,
    there exists a finite subset $I_0 \subs I$ such that
    in $hC$, $f$ factors through 
    $\coprod_{i \in I_0} Y_i \to \coprod_{i \in I} Y_i$.

    \item Suppose $F \in \FUN^\EX_\CTS(C , D)$ where
    both $C, D$ satisfies the conditions in (1) with
    compact generators $X \in C , F(X) \in D$ respectively.
    Suppose the induced morphism $C(X,X) \to D(F(X) , F(X))$ is 
    an equivalence.
    Then $F$ is an equivalence.\footnote{
      The idea that ``for an equivalence it suffices that
      compact generators and their $\mathrm{Ext}$'s match''
      is old,
      but I could not find a formalisation of this idea
      in the stable $\infty$-categorical set up.
      The proof I wrote here is likely to be faulty so
      I would be grateful if someone can check the proof.
    }
  \end{enumerate}
\end{prop}
\begin{proof}
  The following proof of (1) is adapted from the proof of 
  \cite[Prop 1.4.4.2]{Lurie-HA}
  which is about general presentable $\infty$-categories,
  rather than the compactly generated special case.

  We first show $C(\om)$ is equivalent to a small $\infty$-category.
  This part is the same as in \cite[Prop 1.4.4.2]{Lurie-HA}.
  It is a set-theoretic issue that's not very interesting,
  so we will use the following result without proof.
  \begin{lem}

    Let $C$ be an $\infty$-category and $\ka$ a regular cardinal 
    which is uncountable.
    Then there exists a $\ka$-small $\infty$-category $D \map{\sim}{} C$
    if and only if $hC$ is $\ka$-small and $C$ is locally $\ka$-small,
    i.e. for every morphism $f : X \to Y$ in $C$,
    $\pi_0 C(X,Y)$ and $\pi_{i > 0}(C(X,Y) , f)$ are $\ka$-small.
    \begin{proof1}
      See \cite[Prop 5.4.1.2]{Lurie-HTT}
    \end{proof1}
  \end{lem}
  Since $h C(\omega)$ is small, in order to show $C(\om)$ is equivalent to
  a small $\infty$-category,
  it remains to show that $C$ is locally small. 
  We have $\pi_0 (X , Y) = hC(X , Y)$ is small by assumption.
  For the higher homotopy groups, note that 
  $C(X,Y) \simeq \Om C(\Om X , Y)$ therefore
  the we can WLOG assume $f = 0$ for computation of the higher homotopy groups.
  Then $\pi_{i > 0}(C(X,Y), 0) \simeq h C(X[i] , Y)$ which is small again.

  Now we may WLOG assume $C(\om)$ is small.
  By construction, $C(\om)$ is closed under translations and cofibers.
  It follows from the stability of $C$ that $C(\om)$ is also closed under
  fibers and hence a stable full subcategory.
  We thus have (a) by construction.

  To prove (b),
  by the \linkto{dgcat.ind.up}{universal property of ind-completions},
  the inclusion $C(\om) \to C$ factors into 
  \[
    C(\om) \map{\text{Yoneda}}{} \IND\,C(\om) \map{j}{} C
  \]
  where $j$ is the left Kan extension of $C(\om) \to C$.
  The inclusion $C(\om) \subs C$ is fully faithful and 
  every object in $C(\om)$ is compact,
  \linkto{dgcat.cg.out}{therefore $j$ is fully faithful}.
  
  It remains to show the essential image of $j$ is all of $C$.
  We will achieve this by explicitly computing an inverse.
  We saw above that $C$ is locally small.
  So by \linkto{dgcat.psh.up}
  {the universal property of presheaf $\infty$-categories}
  applied to the inclusion $i : C(\om) \to C$, we have an adjunction
  \begin{cd}
    {\IND\,C(\omega)} \\
    {\PSH\, C(\omega)} & C
    \arrow[shift left=3, shorten <=5pt, from=1-1, to=2-2, "{j}"]
    \arrow["\subseteq"', from=1-1, to=2-1]
    \arrow["{i_!}", shift left=2, from=2-1, to=2-2]
    \arrow["{i^*}", shift left=2, from=2-2, to=2-1]
    \arrow["\bot"{description}, draw=none, from=2-1, to=2-2]
  \end{cd}
  where $i_!$ is the left Kan extension of $i$ along 
  $C(\om) \to \PSH\,C(\om)$.
  The above diagram commutes up to isomorphism 
  \linkto{dgcat.ind.up}{because} both $j$ and $i_!$ restrict to give
  $i : C(\om) \to C$.
  Now, the fact that $i$ preserves finite colimits 
  \linkto{dgcat.ind.up}{implies} that $i^* C \subs \IND\,C(\om)$.
  Thus, we have an adjunction 
  $j \dashv i^* : \IND\,C(\om) \rightleftarrows C$. 
  It suffices to show that for all $Y$ in $C$,
  $j(i^* Y) \map{\sim}{} Y$.
  Let $K \to j(i^* Y) \to Y$ be a fiber sequence.
  It suffices to show $K \simeq 0$.
  Since $X$ is a generator of $C$, 
  this is equivalent to showing \[
    0 \simeq hC(X , K) \simeq \pi_0 (\IND\,C(\om))(X , i^* K)
  \]
  where the latter isomorphism comes from the adjunction $j \dashv i^*$.
  This follows by applying the right adjoint $i^*$ to the fiber sequence :

  \begin{cd}
    K & {j(i^*Y)} & \rightsquigarrow & {i^*K} & {i^*(j(i^*Y))} \\
    0 & Y && 0 & {i^*Y}
    \arrow[from=1-1, to=1-2]
    \arrow[from=1-2, to=2-2]
    \arrow[from=1-1, to=2-1]
    \arrow[from=2-1, to=2-2]
    \arrow["\sim", from=1-5, to=2-5]
    \arrow[from=1-4, to=1-5]
    \arrow[from=1-4, to=2-4]
    \arrow[from=2-4, to=2-5]
    \arrow["\lrcorner"{anchor=center, pos=0.125}, draw=none, from=1-4, to=2-5]
    \arrow["\lrcorner"{anchor=center, pos=0.125}, draw=none, from=1-1, to=2-2]
  \end{cd}
  The equivalence $i^* j i^* Y \to i^* Y$ comes from 
  the adjunction $j \dashv i^*$.
  Therefore $i^* K \simeq 0$ and hence we have
  $\pi_0 (\IND\,C(\om))(X , i^* K) \simeq 0$ as desired. 

  (2) We refer the reader to the reference.

  (3)
  By compact generation of $C$ and $D$,
  $F$ is an equivalence \linkto{dgcat.cg.out}{iff}
  $F$ is fully faithful on the full subcategory $C^c$ of compact objects
  of $C$ and $F C^c \subs D^c$.
  By the \linkto{dgcat.idem}{theory of idempotent completions},
  $C(\om) \to C^c$ exhibits the latter as the idempotent completion of
  the former, the smallest stable full subcategory of $C$ containing $X$.
  We have the same situation with $D(\om) \to D^c$.
  It thus suffices to show that $F : C(\om) \to D$ factors through
  $D(\om)$ and gives an equivalence $F : C(\om) \map{\sim}{} D(\om)$.
  We check this inductively.
  For the base case, we have $F : C(0) \map{\sim}{} D(0)$
  because the homs in $C(0), D(0)$ are given by suspensions and loopings
  of respectively $C(X,X) , D(F(X) , F(X))$, which we assumed are equivalent
  under $F$.
  For $k + 1$, \linkto{dgcat.stable}{exactness of $F$} and
  the inductive hypothesis that $F : C(k) \simeq D(k)$,
  we obtain that the essential image of $C(k+1)$ under $F$ is $D(k+1)$.
  Since homs between cofibers of morphisms in $C(k)$ are determined by
  fibers and cofibers of homs in $C(k)$,
  and $F$ is exact,
  we obtain that $F : C(k+1) \to D(k + 1)$ is fully faithful.
  Therefore, $F$ induces an equivalence $C(k+1) \simeq D(k+1)$.

\end{proof}

Finally, there is one hard fact we shall use without motivation.
\begin{prop}
  \link{dgcat.presentable.has_lim}
  
  Let $C \in (\infty,1)\dash\CAT^\EX_\CTS$.
  Then $C$ has all small limits.
  \cite[Prop 5.5.2.4]{Lurie-HTT}
\end{prop} 

\end{document}