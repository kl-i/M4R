\documentclass[./main.tex]{subfiles}
\begin{document}

\begin{rmk}[Why use derived-ness?]
	
	A simple reason is that base change theorems break 
	without flatness assumptions.
	Consider the following example.
	Let $A := k[t]/(t)^2$ where $k$ is a field and let
	$\SPEC\,k \to \SPEC\,A$ be the closed embedding of $t = 0$.
	Then the \emph{classical} fiber product gives the following.
	\begin{cd}
		{\SPEC\,k} & {\SPEC\,k} & \rightsquigarrow & {\QCOH\,k} & {\QCOH\,k} \\
		{\SPEC\,k} & {\SPEC\,A} && {\QCOH\,k} & {\QCOH\,A}
		\arrow["i"', from=2-1, to=2-2]
		\arrow["{\id{}}"', from=1-1, to=2-1]
		\arrow["{\id{}}", from=1-1, to=1-2]
		\arrow["i", from=1-2, to=2-2]
		\arrow["\lrcorner"{anchor=center, pos=0.125}, draw=none, from=1-1, to=2-2]
		\arrow["{\id{}}"', from=1-4, to=2-4]
		\arrow["{\id{}}"', from=1-5, to=1-4]
		\arrow["{Li^*}", from=2-5, to=2-4]
		\arrow["{R i_*}", from=1-5, to=2-5]
		\arrow["\not\simeq"{description}, Rightarrow, from=1-4, to=2-5]
	\end{cd}
	Indeed the diagram on the right hand side does not commute up to isomorphism
	since we have : 
	\begin{cd}
		&&& \vdots & \vdots \\
		&&& {A \otimes_A k} & k \\
		{Li^*(Ri_* \,k)} & {Li^*(k)} & {k \otimes^L_A k} & {A \otimes_A k} & k & k
		\arrow["\simeq"{description}, draw=none, from=3-1, to=3-2]
		\arrow["\simeq"{description}, draw=none, from=3-2, to=3-3]
		\arrow["{t \otimes 1}"', from=2-4, to=3-4]
		\arrow["{t \otimes 1}"', from=1-4, to=2-4]
		\arrow["\simeq"{description}, draw=none, from=3-3, to=3-4]
		\arrow["0"', from=1-5, to=2-5]
		\arrow["0"', from=2-5, to=3-5]
		\arrow["\simeq"{description}, draw=none, from=3-4, to=3-5]
		\arrow["\not\simeq"{description}, draw=none, from=3-5, to=3-6]
	\end{cd}
  Going derived fixes this because we have the isomorphism 
	\[
    \_ \otimes^L_A k \simeq \_ \otimes^L_k (k \otimes^L_A k)		
	\]

\end{rmk}
  
IP : explain the setup of prestacks, i.e. the following diagram.

GR1, Ch 2, 1.7.6.
\begin{cd}
	& {\mathrm{dAff}} & {\mathrm{Aff}} & {\mathrm{Aff}^\mathrm{red}} \\
	{\mathrm{Stk}_\mathrm{ét}} & {\mathrm{PStk}} & {\mathrm{PStk}_\mathrm{cl}} & {\mathrm{PStk}_\mathrm{red}} \\
	{} & {\mathrm{PStk}_\mathrm{laft}} && {\mathrm{PSh}\,\mathrm{dAff}^{<\infty}_\mathrm{ft}} && {} \\
	&& {\mathrm{PSh}\,\mathrm{dAff}^{<\infty}}
	\arrow["\sim"{description}, draw=none, from=3-2, to=3-4]
	\arrow["\LAN"', shift right=2, from=2-3, to=2-2]
	\arrow["\LAN"', shift right=2, from=2-4, to=2-3]
	\arrow["L"', shift right=2, from=2-2, to=2-1]
	\arrow["\subseteq"', shift right=2, from=2-1, to=2-2]
	\arrow["\bot"{description}, draw=none, from=2-1, to=2-2]
	\arrow["\subseteq"', from=3-2, to=2-2]
	\arrow[shift left=2, from=3-4, to=3-2]
	\arrow["{\mathrm{res}}", shift left=1, from=3-2, to=3-4]
	\arrow["{\mathrm{Ran}}"{description}, from=4-3, to=3-2]
	\arrow["{\mathrm{Lan}}"{description}, from=3-4, to=4-3]
	\arrow[from=1-2, to=2-2]
	\arrow["\supseteq"{description}, from=1-3, to=1-2]
	\arrow[from=1-3, to=2-3]
	\arrow[from=1-4, to=2-4]
	\arrow["\supseteq"{description}, from=1-4, to=1-3]
	\arrow["{\mathrm{cl}}"', shift right=2, from=2-2, to=2-3]
	\arrow["{\mathrm{red}}"', shift right=2, from=2-3, to=2-4]
	\arrow["\bot"{description}, draw=none, from=2-2, to=2-3]
	\arrow["\bot"{description}, draw=none, from=2-3, to=2-4]
\end{cd}
\[
  \DSCH_\AFT := \DSCH_\mathrm{qc} \cap \STK_\text{ét} \cap \PSTK_\LAFT  
\]
\begin{itemize}

	\item \link{dsch.lan}
	
	The left Kan extension functors $\PSTK_\CL \to \PSTK$
  are obtained from the \linkto{dgcat.psh.up}
	{universal property of presheaf $\infty$-categories}
	to inclusion of full subcategories $\AFF \subs \DAFF$.

	Note that since small colimits in presheaf $\infty$-categories are computed
	pointwise, $\mathrm{res}$ preserves small colimits.
	We obtain that $\mathrm{res}\,\LAN \map{\sim}{} \id{}$
	since this is the case on derived affines.
	Thus, $\LAN$ is fully faithful.
	In other words, we can safely think of classical prestacks
	as special cases of prestacks.

	The same reasoning applies for $\PSTK_\RED \to \PSTK_\CL$
	and the composite $\PSTK_\RED \to \PSTK$.
	So we can think of reduced prestacks as special cases of
	classical prestacks, and also of prestacks.

	We henceforth view $\PSTK_\RED \to \PSTK_\CL \to \PSTK$
	as inclusions of full subcategories.

	\item For $X \in \PSTK$, we say $X$ has an underlying scheme when
	the underlying classical prestack is a scheme.

	\item (GR1 Ch2 1.2.3, 1.2.7)
	$\DAFF^{<\infty} := \bigcup_{0 \leq n} \DAFF^{\leq n}$ where
  $\DAFF^{\leq n}$ is the full subcategory of $\DAFF$ consisting of
	$\SPEC\,A$ such that $\pi_{i > n} A \simeq H^{i < -n} A \simeq 0$.
	Derived affine schemes in $\DAFF^{<\infty}$ are called
	\emph{eventually coconnective}.\footnote{
		The ``coconnectivity'' refers to \emph{cohomological grading}.
		So eventually coconnective means the same thing as
		\emph{``homologically eventually connective''}.
	}

	Since $\DAFF^{<\infty} \subs \DAFF$ is a full subcategory,
	we obtain a \linkto{dgcat.psh.triple}{triple of adjoints} 
  $\LAN \dashv \mathrm{res} \dashv \RAN : 
	\PSTK \rightleftarrows \PSH\,\DAFF^{<\infty}$.
	Then essential image of $\RAN$ is of importance due to the following
	characterisation.
	\begin{prop}[GR1 Ch2 1.4.7]
		
		Let $X \in \PSTK$.
		Then $X$ is in the essential image of 
		$\RAN : \PSH\,\DAFF^{<\infty} \to \PSTK$ if and only if
		for all $\SPEC\,A \in \DAFF$,
		we have $X(A) \map{\sim}{} \LIM_{0 \leq n} X(\tau^{\leq n} A)$.
	\end{prop}

	\item $\DAFF_\AFT$ is defined as the full subcategory of $\DAFF$
  consisting of derived affines of \emph{almost finite type}.
	This means $\SPEC\,A$ where $H^0 A$ is finite type over the base field $k$
	and $H^{i > 0} A$ are finite generated over $H^0 A$.

	$\DAFF^{<\infty}_\FT := \DAFF^{<\infty} \cap \DAFF_\AFT$.

	\begin{prop}[Characterisation of Prestacks of Locally Almost Finite Type]
		\link{dsch.laft.char}
		
		Restriction along $\DAFF^{<\infty}_\FT \subs \DAFF$
		gives an equivalence \begin{cd}
			{\PSTK_\LAFT} && {\PSH\,\DAFF^{<\infty}_\FT} \\
			& {\PSH\,\DAFF^{<\infty}}
			\arrow["{\mathrm{res}}", shift left=2, from=1-1, to=1-3]
			\arrow["\sim"{description}, draw=none, from=1-1, to=1-3]
			\arrow[shift left=2, from=1-3, to=1-1]
			\arrow["\RAN", shorten >=3pt, from=2-2, to=1-1]
			\arrow["\LAN", shorten <=4pt, from=1-3, to=2-2]
		\end{cd}
		where an inverse is given by 
		first \emph{left} Kan extending along 
		$\DAFF^{<\infty}_\FT \subs \DAFF^{<\infty}$
		then \emph{right} Kan extending along $\DAFF^{<\infty} \subs \DAFF$.
	\end{prop}

	\linkto{dsch.laft.fin_lim}{}
	$\PSTK_\LAFT \subs \PSTK$ preserves finite limits.
	(GR1 Ch2 1.7.10)

	$\DAFF_\AFT = \DAFF \cap \PSTK_\LAFT$ (GR1 Ch2 1.7.2)

\end{itemize}

\end{document}