\documentclass[./main.tex]{subfiles}
\begin{document}

Now that we have explained
the kind of category of sheaves,
we describe the spaces the sheaves will be on.
The big picture to follow is the following : 
% \begin{cd}
% 	& {\mathrm{dAff}} & {\mathrm{Aff}} & {\mathrm{Aff}^\mathrm{red}} \\
% 	{\mathrm{Stk}_\mathrm{ét}} & {\mathrm{PStk}} & {\mathrm{PStk}_\mathrm{cl}} & {\mathrm{PStk}_\mathrm{red}} \\
% 	{} & {\mathrm{PStk}_\mathrm{laft}} && {\mathrm{PSh}\,\mathrm{dAff}^{<\infty}_\mathrm{ft}} && {} \\
% 	&& {\mathrm{PSh}\,\mathrm{dAff}^{<\infty}}
% 	\arrow["\sim"{description}, draw=none, from=3-2, to=3-4]
% 	\arrow["\LAN"', shift right=2, from=2-3, to=2-2]
% 	\arrow["\LAN"', shift right=2, from=2-4, to=2-3]
% 	\arrow["L"', shift right=2, from=2-2, to=2-1]
% 	\arrow["\subseteq"', shift right=2, from=2-1, to=2-2]
% 	\arrow["\bot"{description}, draw=none, from=2-1, to=2-2]
% 	\arrow["\subseteq"', from=3-2, to=2-2]
% 	\arrow[shift left=2, from=3-4, to=3-2]
% 	\arrow["{\mathrm{res}}", shift left=1, from=3-2, to=3-4]
% 	\arrow["{\mathrm{Ran}}"{description}, from=4-3, to=3-2]
% 	\arrow["{\mathrm{Lan}}"{description}, from=3-4, to=4-3]
% 	\arrow[from=1-2, to=2-2]
% 	\arrow["\supseteq"{description}, from=1-3, to=1-2]
% 	\arrow[from=1-3, to=2-3]
% 	\arrow[from=1-4, to=2-4]
% 	\arrow["\supseteq"{description}, from=1-4, to=1-3]
% 	\arrow["{\mathrm{cl}}"', shift right=2, from=2-2, to=2-3]
% 	\arrow["{\mathrm{red}}"', shift right=2, from=2-3, to=2-4]
% 	\arrow["\bot"{description}, draw=none, from=2-2, to=2-3]
% 	\arrow["\bot"{description}, draw=none, from=2-3, to=2-4]
% \end{cd}
\begin{cd}
	{\AFF^\RED} & \AFF & \DAFF & {\PSH\,\DAFF^{\leq n}} & {\PSH\,\DAFF^{\leq n}_\FT} & {} \\
	{\PSTK_\RED} & {\PSTK_\CL} & \PSTK & {\PSH\,\DAFF^{<\infty}} & {\PSH\,\DAFF^{<\infty}_\FT} \\
	&& {\STK_\text{ét}} & {\PSTK_\LAFT}
	\arrow["\LAN"', shift right=3, from=1-4, to=2-4]
	\arrow["\LAN"', shift right=3, from=1-5, to=2-5]
	\arrow["\LAN", shift left=3, from=2-1, to=2-2]
	\arrow["\LAN", shift left=3, from=2-2, to=2-3]
	\arrow["\RED", shift left=3, from=2-2, to=2-1]
	\arrow["\CL", shift left=3, from=2-3, to=2-2]
	\arrow["\bot"{description}, draw=none, from=2-1, to=2-2]
	\arrow["\bot"{description}, draw=none, from=2-2, to=2-3]
	\arrow["L"', shift right=3, from=2-3, to=3-3]
	\arrow["\subseteq"', shift right=3, from=3-3, to=2-3]
	\arrow["\dashv"{description}, draw=none, from=3-3, to=2-3]
	\arrow["\subseteq"{description}, shift left=1, from=3-4, to=2-3]
	\arrow["\RAN", shift left=3, from=2-4, to=3-4]
	\arrow["\LAN", shift left=3, from=2-5, to=2-4]
	\arrow[shift left=3, from=2-4, to=2-5]
	\arrow["\top"{description}, draw=none, from=2-4, to=2-5]
	\arrow[shift left=3, from=3-4, to=2-4]
	\arrow["\dashv"{description}, draw=none, from=3-4, to=2-4]
	\arrow["\sim"{description}, shift right=2, from=3-4, to=2-5]
	\arrow[shift right=3, from=2-3, to=2-4]
	\arrow[shift right=3, from=2-4, to=2-3]
	\arrow["\bot"{description}, draw=none, from=2-3, to=2-4]
	\arrow["\LAN"', shift right=3, from=1-5, to=1-4]
	\arrow["{\leq n}"', shift right=3, from=2-4, to=1-4]
	\arrow["\dashv"{description}, draw=none, from=2-4, to=1-4]
	\arrow[shift right=3, from=1-4, to=1-5]
	\arrow["\bot"{description}, draw=none, from=1-4, to=1-5]
	\arrow[shift right=3, from=2-5, to=1-5]
	\arrow["\dashv"{description}, draw=none, from=1-5, to=2-5]
	\arrow["\subseteq", from=1-1, to=1-2]
	\arrow["\subseteq", from=1-2, to=1-3]
	\arrow[from=1-1, to=2-1]
	\arrow[from=1-2, to=2-2]
	\arrow[from=1-3, to=2-3]
\end{cd}
\[
  \DSCH_\AFT := \DSCH_\mathrm{qc} \cap \STK_\text{ét} \cap \PSTK_\LAFT  
\]
Let us explain.
\begin{itemize}

	\item $\DAFF$ refers to the opposite of the infinity category
	of commutative algebra objects in $\VEC$.

	Practically speaking, $\DAFF^\OP$ can be realised as 
	the localisation of commutative dg-algebras concentrated in non-positive degree 
	with respect to a suitable model structure. 
  (Alternatively, localisation of simplicial commutative rings w.r.t. 
  suitable model structure. However, you would then need to show
  is equivalent to commutative algebra objects in $\VEC$.)
	We use cohomological grading so homological degree refer to 
	negative cohomological degree.

	$\AFF$ is the full subcategory of $\DAFF$ of classical affine schemes,
	and $\AFF^\RED$ the reduced classical affine schemes,
	which we will simply refer to as \emph{reduced} for brevity.

	We use $\SPEC$ to denote the Yoneda embedding of
	any of $\AFF^\RED , \AFF , \DAFF$ into their presheaf $\infty$-categories.

	\item \link{dsch.lan}
	
	The left Kan extension functors $\PSTK_\CL \to \PSTK$
  are obtained from the \linkto{dgcat.psh.up}
	{universal property of presheaf $\infty$-categories}
	to inclusion of full subcategories $\AFF \subs \DAFF$.

	Note that since small colimits in presheaf $\infty$-categories are computed
	pointwise, $\mathrm{res}$ preserves small colimits.
	We obtain that $\mathrm{res}\,\LAN \map{\sim}{} \id{}$
	since this is the case on derived affines.
	Thus, $\LAN$ is fully faithful.
	In other words, we can safely think of classical prestacks
	as special cases of prestacks.

	The same reasoning applies for $\PSTK_\RED \to \PSTK_\CL$
	and the composite $\PSTK_\RED \to \PSTK$.
	So we can think of reduced prestacks as special cases of
	classical prestacks, and also of prestacks.

	We henceforth view $\PSTK_\RED \to \PSTK_\CL \to \PSTK$
	as inclusions of full subcategories.

	\item For $X \in \PSTK$, we say $X$ has an underlying scheme when
	the underlying classical prestack is a scheme.

	\item \cite[Ch 2, 1.2.3, 1.2.7]{GR1}
	$\DAFF^{<\infty} := \bigcup_{0 \leq n} \DAFF^{\leq n}$ where
  $\DAFF^{\leq n}$ is the full subcategory of $\DAFF$ consisting of
	$\SPEC\,A$ such that $\pi_{i > n} A \simeq H^{i < -n} A \simeq 0$.
	Derived affine schemes in $\DAFF^{<\infty}$ are called
	\emph{eventually coconnective}.\footnote{
		The ``coconnectivity'' refers to \emph{cohomological grading}.
		So eventually coconnective means the same thing as
		\emph{``homologically eventually connective''}.
	}

	Since $\DAFF^{<\infty} \subs \DAFF$ is a full subcategory,
	we obtain a \linkto{dgcat.psh.triple}{triple of adjoints} 
  $\LAN \dashv \mathrm{res} \dashv \RAN : 
	\PSTK \rightleftarrows \PSH\,\DAFF^{<\infty}$.
	Then essential image of $\RAN$ is of importance due to the following
	characterisation.
	\begin{prop}
		\link{dsch.pstk.conv}
		
		Let $X \in \PSTK$.
		Then $X$ is in the essential image of 
		$\RAN : \PSH\,\DAFF^{<\infty} \to \PSTK$ if and only if
		for all $\SPEC\,A \in \DAFF$,
		we have $X(A) \map{\sim}{} \LIM_{0 \leq n} X(\tau^{\leq n} A)$.
		We call such prestacks \emph{convergent}.
		\cite[Ch2, 1.4.7]{GR1}
	\end{prop}

	\item $\DAFF_\AFT$ is defined as the full subcategory of $\DAFF$
  consisting of derived affines of \emph{almost finite type}.
	This means $\SPEC\,A$ where $H^0 A$ is finite type over the base field $k$
	and $H^{i > 0} A$ are finite generated over $H^0 A$.\cite[Ch 2, 1.7.1]{GR1}
  We also define 
  \[ \DAFF^{\leq n}_\FT := \DAFF^{\leq n} \cap \DAFF_\AFT \]
	With 
	\linkto{dgcat.psh.up}{the same argument as in previous points},
	we obtain an adjunction $\LAN \dashv \RES : 
	\PSH\,\DAFF^{\leq n}_\FT \rightleftarrows \PSH\,\DAFF^{\leq n}$
	where $\LAN$ is fully faithful so 
	we can see the former as a full subcategory of the latter.
	The objects of this subategory has the following charactersation : 
	\begin{prop}
		\link{dsch.pstk.preserve_fil_colim}
		
		Let $X \in \PSH\,\DAFF^{\leq n}$.
		Then $X$ lies in $\PSH\,\DAFF^{\leq n}_\FT$ if and only if
		as a functor $(\DAFF^{\leq n})^\OP \to \SPC$,
		it preserves small filtered colimits.
		\cite[Ch 2, 1.6.4]{GR1}
	\end{prop}
  Finally, the building blocks of derived algebraic geometry
	which will allow us to do computations
	\[ \DAFF^{<\infty}_\FT := \DAFF^{<\infty} \cap \DAFF_\AFT 
	= \bigcup_{0 \leq n} \DAFF^{\leq n}_\FT \]
	We have a characterisation of the prestacks determined by
	how $S \in \DAFF^{<\infty}_\FT$ map into them.
	These are the \emph{laft prestacks}.
	\begin{prop}[Characterisation of Prestacks of Locally Almost Finite Type]
		\link{dsch.laft.char}
		
		Restriction along $\DAFF^{<\infty}_\FT \subs \DAFF$
		gives an equivalence \begin{cd}
			{\PSTK_\LAFT} && {\PSH\,\DAFF^{<\infty}_\FT} \\
			& {\PSH\,\DAFF^{<\infty}}
			\arrow["{\mathrm{res}}", shift left=2, from=1-1, to=1-3]
			\arrow["\sim"{description}, draw=none, from=1-1, to=1-3]
			\arrow[shift left=2, from=1-3, to=1-1]
			\arrow["\RAN", shorten >=3pt, from=2-2, to=1-1]
			\arrow["\LAN", shorten <=4pt, from=1-3, to=2-2]
		\end{cd}
		where \begin{itemize}
			\item $\PSTK_\LAFT$ is the full subcategory of $\PSTK$ consisting of
      $X$ such that \begin{enumerate}
				\item $X$ is \linkto{dsch.pstk.conv}{convergent}.
        \item $X$ takes filtered limits in $\DAFF^{\leq n}$ to 
				filtered colimits in $\SPC$,
				\linkto{dsch.pstk.preserve_fil_colim}{equivalently}
				$X^{\leq n}$ lies in $\PSH\,\DAFF^{\leq n}_\FT$.
			\end{enumerate}
			\item an inverse is given by 
			first \emph{left} Kan extending along 
			$\DAFF^{<\infty}_\FT \subs \DAFF^{<\infty}$
			then \emph{right} Kan extending along $\DAFF^{<\infty} \subs \DAFF$.
		\end{itemize}
		\cite[Ch 2, 1.7.6]{GR1}
	\end{prop}
	This is the class of prestacks for which
	$\INDCOH$ will be developed.

  We also have that
	$\PSTK_\LAFT \subs \PSTK$ preserves finite limits,
	which will be useful for us later.
	\link{dsch.laft.fin_lim}
	\cite[Ch 2 , 1.7.10]{GR1}

  One can also show that
	$\DAFF_\AFT = \DAFF \cap \PSTK_\LAFT$.
	\cite[Ch 2 , 1.7.2]{GR1}

	\item In the adjunction 
	$L \dashv \text{ forget } : \PSTK \rightleftarrows \STK_\text{ét}$,
	we have $\STK_\text{ét}$ denoting the full subcategory of $\PSTK$
	consisting of prestacks satisfying étale descent.
	$L$ is the sheafification functor.
	We refer the reader to \cite[Ch 2, 2.3]{GR1}
	for details of the definition.

	\item $\DSCH$ denotes the full subcategory of $\PSTK$ consisting of
	\emph{derived schemes}. 
	See \cite[Ch 2, 3.1.1]{GR1} for the definition.
	A thing to note is that the derived schemes in \cite{GR1}
	have diagonals that are closed embeddings and affine by definition.

	$\DSCH_\mathrm{qc}$ denotes the full subcategory of $\DSCH$
	consisting of derived schemes which admit a finite Zariski cover by
	derived affines.

	Finally, $\DSCH_\AFT$ denotes the $\infty$-category of
	\emph{derived schemes of almost finite type}.
	\cite[Ch 2, 3.5]{GR1}
	The theory of $\INDCOH$ will be built from these spaces.
	
\end{itemize}

\end{document}