\documentclass[./main.tex]{subfiles}
\begin{document}
  
$(\infty,1)\dash\CAT^\mathrm{ex}$ denotes subcategory of $(\infty,1)\dash\CAT$
consisting of \emph{stable infinity categories} and \emph{exact functors}.  
It contains all small limits and
the ``inclusion'' $(\infty,1)\dash\CAT^\EX \to (\infty,1)\dash\CAT$ preserves 
small limits \cite[Prop 1.1.4.4]{Lurie-HA}.

Stable infinity categories are basically
triangulated categories where exact triangles are determined by
an infinity-categorical universal property.
Here is the definition.
\begin{dfn}
  \link{dgcat.stable}
  
  Let $C$ be an infinity category. 
  We say $C$ has a \emph{zero object} when
  it has an object that is both initial and final. 
  \cite[Def 1.1.1.1]{Lurie-HA}

  Now assume $C$ have a zero object.
  Then a \emph{triangle} in $C$ is defined as a diagram in $C$ of the form : 
  \begin{cd}
    X & Y \\
    0 & Z
    \arrow[from=1-1,to=1-2]
    \arrow[from=1-1,to=2-1]
    \arrow[from=1-2,to=2-2]
    \arrow[from=2-1,to=2-2]
  \end{cd}
  A triangle is called a \emph{fiber sequence} when it is a cartesian
  and a \emph{cofiber sequence} when it is cocartesian.
  \cite[Def 1.1.1.4]{Lurie-HA}
  In the first case,
  we say \emph{$Y \to Z$ admits a fiber} and refer to $X$ as the fiber,
  and in the other case
  we say \emph{$X \to Y$ admits a cofiber} and refer to $Z$ as the cofiber.

  $C$ is called \emph{stable} when the following are true : 
  \begin{itemize}
    \item every morphism has both a fiber and a cofiber.
    \item A triangle is fiber sequence iff it is a cofiber sequence.
    Such triangles are called \emph{exact triangles}.
  \end{itemize}
  \cite[Prop 1.1.1.9]{Lurie-HA}

  An exact functor $F : C \to D$ between stable infinity categories
  is one which satisfy any of the following equivalent conditions : 
  \cite[Prop 1.1.4.1]{Lurie-HA}
  \begin{itemize}
    \item $F$ preserves exact triangles
    \item $F$ preserves finite limits
    \item $F$ preserves finite colimits.
  \end{itemize}
  
  For stable $\infty$-categories $C, D$
  the full subcategory $\FUN^\EX(C,D)$ of $\FUN(C,D)$ consisting of
  exact functors is also stable.\footnote{
    \cite[Ch 1, 5.1.4]{GR1}claims this.
    \cite[Prop 1.1.3.1]{Lurie-HA} shows that $\FUN(K,C)$ is stable for
    any $K$ and stable $C$.
    The result follows given that
    finite (co)limit-preserving functors
    are closed under finite (co)limits.
  }
\end{dfn}
To help build intuition of ``stable infinity categories as 
fixed triangulated categories'',
we record here the important parts of the procedure
of extracting a triangulated category from a stable infinity category.
\begin{prop}
  
  Let $C$ be a stable infinity category.
  Then the following defines a triangulated structure on 
  the 1-category $hC$ : 
  \begin{itemize}
    \item Define the \emph{suspension functor} $\Sigma : C \to C$ by
    pushout against zeros : 
    \begin{cd}
      X & 0 \\
      0 & \Sigma X
      \arrow[from=1-1,to=1-2]
      \arrow[from=1-1,to=2-1]
      \arrow[from=1-2,to=2-2]
      \arrow[from=2-1,to=2-2] 
    \end{cd}
    Since the above square is a cofiber sequence,
    it is also a fiber sequence. 
    This shows that \emph{looping} $\Omega : C \to C$,
    given by pullback against zeros, gives an inverse for $\Sigma$
    and hence shows that $\Sigma$ is an equivalence.
    Taking homotopy categories, we obtain an equivalence 
    $[1] : hC \map{\sim}{} hC$, which we use as the shift functor
    for the triangulated structure.

    \item We call a diagram \[
      X \map{f}{} Y \map{g}{} Z \map{h}{} X[1] 
    \]
    in $hC$ an exact triangle (in the triangulated categorical sense) 
    when it comes from a diagram of the following form in $C$ : 
    \begin{cd}
      X & Y & 0 \\
      0 & Z & {X[1]}
      \arrow[from=1-1, to=1-2]
      \arrow[from=1-2, to=2-2]
      \arrow[from=2-2, to=2-3]
      \arrow[from=1-1, to=2-1]
      \arrow[from=2-1, to=2-2]
      \arrow[from=1-2, to=1-3]
      \arrow[from=1-3, to=2-3]
      \arrow["\lrcorner"{anchor=center, pos=0.125, rotate=180}, 
        draw=none, from=2-2, to=1-1]
      \arrow["\lrcorner"{anchor=center, pos=0.125, rotate=180}, 
        draw=none, from=2-3, to=1-2]
    \end{cd}
    i.e. two exact triangles (in the stable infinity categorical sense).
    \item For $X, Y$ objects of $C$,
    we have 
    \begin{align*}
      C(X,Y) &\simeq C(\Sigma \Omega X , Y) \simeq \Omega C(\Omega X , Y) \\
      &\simeq C(\Sigma^2 \Omega^2 X , Y) \simeq \Omega^2 C(\Omega^2 X , Y)
    \end{align*}
    Upon taking $\pi_0$ , we obtain 
    \[
      hC(X,Y) := \pi_0 C(X,Y) \simeq \pi_1 C(\Om X , Y) 
      \simeq \pi_2 (\Om^2 X , Y)
    \]
    where the last isomorphism is a group morphism.
    For $\pi_2$ of any ``space'' \footnote{
      In the quasi-category model of infinity categories,
      $C(X,Y)$ is a Kan complex,
      which one can take homotopy groups of.
    } the obvious group structure given by is abelian,
    this gives $hC(X,Y)$ an abelian group structure,
    making $hC$ into an additive category.
  \end{itemize}
  \cite[Thm 1.1.2.14]{Lurie-HA}

  For $X, Y$ objects in $C$,
  we define the abelian group $\EXT^n_C(X,Y) := hC(X , Y[n])$.
  \cite[Notation 1.1.2.17]{Lurie-HA}
\end{prop}

A t-structure on a stable $\infty$-category is simply a
t-structure on its homotopy category.
\cite[Def 1.2.1.4]{Lurie-HA}
% (IP : t-structures, truncation as reflective localisation,
% $(D^-(A))^\heartsuit \simeq A$ Lurie 1.3.2.19)

\begin{dfn}
  
  Let $F : C \to D$ be a functor in $(\infty,1)\dash\CAT^\EX$
  where $C$ and $D$ are equipped with t-structures.
  Then $F$ is called \emph{right t-exact} when
  $F C_{0 \leq} \subs D_{0 \leq}$.
  It is called \emph{left t-exact} when $F C_{\leq 0} \subs D_{\leq 0}$.
  We say $F$ is \emph{t-exact} when it is both left and right t-exact. 

  \cite[Def 1.3.3.1]{Lurie-HA}
\end{dfn}

\end{document}