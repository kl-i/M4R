\documentclass[./main.tex]{subfiles}
\begin{document}

\begin{dfn}[de Rham space]
  
By \linkto{dgcat.psh.triple}{the theory of left and right Kan extensions},
we have the following adjunctions :
\begin{cd}
  {\mathrm{Aff}^\mathrm{red}} & {\mathrm{dAff}} & \rightsquigarrow & {\mathrm{PStk}_\mathrm{red}} && {\mathrm{PStk}}
	\arrow["\subseteq", from=1-1, to=1-2]
	\arrow["{\mathrm{res}}"{description}, from=1-6, to=1-4]
	\arrow["{\mathrm{Ran}}"', shift right=5, from=1-4, to=1-6]
	\arrow["{\mathrm{Lan}}", shift left=5, from=1-4, to=1-6]
	\arrow["\bot"{description}, shift left=3, draw=none, from=1-4, to=1-6]
	\arrow["\bot"{description}, shift right=3, draw=none, from=1-4, to=1-6]
\end{cd}

The functor which takes de Rham space is defined to be 
$\DR := \mathrm{Ran}\,\mathrm{res}$.

The unit of the adjunction $\mathrm{res} \dashv \RAN$ is a
natural transformation $\id{\PSTK} \to \DR$.
For $X \in \PSTK$, we write $p_\DR : X \to X_\DR$
for the component of the natural transformation.

\cite[Section 1.1]{Crys}
\end{dfn}

\begin{rmk}
  Let $X \in \PSTK$ and $\SPEC\, A \in \DAFF$.
  Then points of $X_\DR$ has the following description :
\[
  \PSTK(\SPEC\,A , X_\DR) 
  \simeq \PSTK_\mathrm{red}(\SPEC\,A_\mathrm{red} , X_\mathrm{red})
  \simeq \PSTK(\SPEC\,A_\RED , X)
\]
  The first equivalence is by definition of $\RAN$ and
  the second is by the fully faithful embedding $\PSTK_\RED \to \PSTK$.
  In particular, the morphism $p_\DR : X \to X_\DR$ gives
  at the level of points : 
  \[
    \PSTK(\SPEC\,A , X) \to \PSTK(\SPEC\,A_\RED , X)  
  \]
\end{rmk}

\begin{rmk}[Relation to Grothendieck's infinitesimal site]
  
  Let $X \in \PSTK$.
  For simplicity, we assume the underlying classical prestack
  of $X$ is zero-truncated, meaning for every $\SPEC\, A \in \AFF$
  we have $X(A)$ is a set. 
  (E.g. if $X$ has an underlying scheme.)
  We will describe the underlying classical prestack of $X_\DR$,
  i.e. the over category $\AFF / X_\DR$.

  First note that since $X^\CL$ is zero-truncated,
  so is $X_\DR^\CL$.
  This means for $\SPEC\, A \in \AFF$,
  $X_\DR(A)$ is a set so we do not have to worry about
  any homotopical phenomena.
  Then given two classical points $x : \SPEC\, A \to X_\DR$ and
  $y : \SPEC\, B \to X_\DR$,
  a morphism $f : y \to x$ over $X_\DR$ is precisely 
  a morphism $f : \SPEC\, B \to \SPEC\, A$ such that 
  $x f = y$ at the reduced level.
  \begin{cd}
    {\mathrm{Spec}\,B} && {\leftrightsquigarrow} & {\mathrm{Spec}\,B} & {\mathrm{Spec}\,B^\RED} \\
    {\mathrm{Spec}\,A} & {X_\mathrm{dR}} && {\mathrm{Spec}\,A} & {\mathrm{Spec}\,A^\RED} & X
    \arrow["f"', from=1-1, to=2-1]
    \arrow["y", from=1-1, to=2-2]
    \arrow["x"', from=2-1, to=2-2]
    \arrow["f"', from=1-4, to=2-4]
    \arrow[from=1-5, to=1-4]
    \arrow[from=2-5, to=2-4]
    \arrow["{f^\RED}"', from=1-5, to=2-5]
    \arrow["{y^\RED}", from=1-5, to=2-6]
    \arrow["{x^\RED}"', from=2-5, to=2-6]
  \end{cd}
  Now assuming $X$ has a underlying scheme,
  Grothendieck's infinitesimal site of $X$ is precisely
  the full subcategory of $\AFF / X_\DR$
  with objects $x : \SPEC\, A \to X_\DR$ such that $\SPEC\, A^\RED \to X$
  is an open immersion.
  \cite[Def 60.9.1]{stacks}

\end{rmk}

\begin{rmk}[What about $\mathrm{Lan}\,\mathrm{res}$]
  
  From the discussion on \linkto{dsch.lan}{prestacks},
  the left Kan extension functor $\LAN : \PSTK_\RED \to \PSTK$
  simply takes a reduced prestack and views it as a general prestack.
  Thus $\LAN\,\mathrm{res}$ is the functor that takes 
  the underlying reduced prestack of a general prestack.
  
\end{rmk}

\begin{rmk}
  
  We actually have more than just an inclusion $\AFF^\RED \to \DAFF$ 
  but an adjunction : 
  \begin{cd}
    {\AFF^\RED} && \DAFF
    \arrow["\subs", shift left=3, from=1-1, to=1-3]
    \arrow["{\SPEC\, (\pi_0 A)_\RED \mapsfrom \SPEC\,A}", shift left=3, from=1-3, to=1-1]
    \arrow["\bot"{description}, draw=none, from=1-1, to=1-3]
  \end{cd}
  From this, we can also deduce that $\DR$ preserves 
  both small colimits and small limits which is useful.
\end{rmk}

\begin{prop}
  
  The functor $\DR$ taking de Rham spaces preserves 
  small colimits and small limits.
\end{prop}
\begin{proof}

  We proceed by a slightly different proof to \cite[Lem 1.1.4]{Crys}.
  Basically, this is an exercise in 
  \linkto{dgcat.psh.triple}{the theory of presheaf $\infty$-categories}.
  To show $\DR = \RAN\,\RES$ preserves small colimits and small limits, 
  it suffices that $\RAN$ is a left adjoint.
  By applying \linkto{dgcat.psh.triple}{the theory of left and right 
  Kan extensions of presheaves} to the adjunction 
  $\AFF^\RED \rightleftarrows \DAFF$, we obtain \emph{two} adjoint triples : 
  \begin{cd}
    &&& {\PSTK_\RED} && \PSTK \\
    {\AFF^\RED} & \DAFF & \rightsquigarrow \\
    {} & {} & {} & {\PSTK_\RED} & {} & \PSTK
    \arrow["\LAN"', shift right=5, from=3-6, to=3-4]
    \arrow["{\mathrm{res}}"{description}, from=3-4, to=3-6]
    \arrow["\RAN", shift left=5, from=3-6, to=3-4]
    \arrow["\bot"{description}, shift left=3, draw=none, from=3-4, to=3-6]
    \arrow["\bot"{description}, shift right=3, draw=none, from=3-4, to=3-6]
    \arrow["\subs", shift left=3, from=2-1, to=2-2]
    \arrow["\bot"{description}, draw=none, from=2-1, to=2-2]
    \arrow["\RED", shift left=3, from=2-2, to=2-1]
    \arrow["\LAN", shift left=5, from=1-4, to=1-6]
    \arrow["{\mathrm{res}}"{description}, from=1-6, to=1-4]
    \arrow["\RAN"', shift right=5, from=1-4, to=1-6]
    \arrow["\bot"{description}, shift left=3, draw=none, from=1-4, to=1-6]
    \arrow["\bot"{description}, shift right=3, draw=none, from=1-4, to=1-6]
  \end{cd}
  The point is that we have a commuting square : 
  \begin{cd}
    {\AFF^\RED} & \DAFF \\
    {\PSTK_\RED} & \PSTK
    \arrow[from=1-1, to=2-1]
    \arrow["{\mathrm{res}}", from=2-2, to=2-1]
    \arrow["{\RED}"',from=1-2, to=1-1]
    \arrow[from=1-2, to=2-2]
  \end{cd}
  So $\mathrm{res}$ in the top adjunction is isomorphic to 
  $\LAN$ in the bottom adjunction,
  hence by uniqueness of adjoint functors 
  we obtain a quadruple of adjoints : 
  \begin{cd}
    {\PSTK_\RED} && \PSTK
    \arrow["\LAN", shift left=9, from=1-1, to=1-3]
    \arrow["{\mathrm{res}}"{description}, shift right = 3, from=1-3, to=1-1]
    \arrow["\RAN"{description}, shift right=3, from=1-1, to=1-3]
    \arrow["\bot"{description}, shift left=6, draw=none, from=1-1, to=1-3]
    \arrow["\bot"{description}, shift right=6, draw=none, from=1-1, to=1-3]
    \arrow[shift left=9, from=1-3, to=1-1]
    \arrow["\bot"{description}, draw=none, from=1-3, to=1-1]
  \end{cd}
  This proves $\RAN$ is a left adjoint and hence preserves small colimits.

\end{proof}
  
\begin{dfn}[Left and Right Crystals]
  
  Let $X \in \PSTK_\LAFT$.
  Then we define the \emph{$\infty$-category of left crystals on $X$} to be
  \[
    \CRYS^L(X) := \QCOH\,X_\DR  
  \]
  \cite[Def 2.1.1]{Crys}
  The \emph{$\infty$-category of right crystals on $X$} is defined to be
  \[
    \CRYS^R(X) := \INDCOH\,X_\DR  
  \]
  \cite[Def 2.3.2]{Crys}
\end{dfn}

Technically speaking, for the definition of right crystals to make sense,
we need to show that $X_\DR$ is also locally almost finite type.
Indeed, we have the following.

\begin{prop}[$\PSTK_\LAFT$ is closed under $\DR$]
  \link{crys.dR.laft}

  Let $X \in \PSTK_\LAFT$.
  Then $X_\DR \in \PSTK_\LAFT$.
  \cite[prop 1.3.3]{Crys}
\end{prop}
\begin{proof}
  
  We first show the convergence of $X_\DR$.
  This follows from the following diagram which commutes up to isomorphism : 
  \begin{cd}
    {X(A_\RED)} & {\varprojlim_n X(A_\RED)} \\
    {X_\DR(A)} & {\varprojlim_n X_\DR(\tau^{\leq n} A)} 
    \arrow["{=}"', from=1-1, to=2-1]
    \arrow[from=2-1, to=2-2]
    \arrow["\sim", from=1-2, to=2-2]
    \arrow["\sim", from=1-1, to=1-2]
  \end{cd}
  It remains to show $X_\DR$ takes filtered limits in $\DAFF^{\leq n}$ to
  filtered colimits in $\SPC$.
  By assumption, $X$ does this so it suffices to show that
  $\RED : \DAFF^{\leq n} \to \AFF^\RED$ preserves filtered limits.
  This is a composition $\DAFF^{\leq n} \to \AFF \to \AFF^\RED$
  where the first part takes $H^0$ and the second part takes reduction.
  Both functors preserve filtered limits.
\end{proof}

\begin{rmk}
  
  A specific situation worth noting is when 
  the morphism $X \to X_\DR$ exhibits $X_\DR$ as the 
  geometric realisation of its Cech nerve $\check{C}(X/X_\DR)$,
  i.e. when $X \to X_\DR$ is an effective epimorphism.\footnote{
    For a definition of Cech nerves,
    see \cite[Prop 6.1.2.11]{Lurie-HTT}.
  }
\end{rmk}

\begin{prop}
  \link{crys.cfsm}

  Let $X \in \PSTK_\LAFT$.
  Then the following are equivalent :
  \begin{enumerate}
    \item For all $S \in \DAFF$, 
    the map of sets $\pi_0 X(S) \to \pi_0 X(S_\RED)$ is surjective.
    \item $X \to X_\DR$ is an effective epimorphism in $\PSTK$.
  \end{enumerate}
  We say $X$ is \emph{classically formally smooth}
  when any (and thus all) of the above are satisfied.
  In this case, we have 
  \begin{cd}
    {\mathrm{Crys}^L\,X} & {\varprojlim \mathrm{QCoh}(\check{C}(X/X_\mathrm{dR}))} \\
    {\mathrm{Crys}^R\,X} & {\varprojlim \mathrm{IndCoh}(\check{C}(X/X_\mathrm{dR}))}
    \arrow["\sim", from=1-1, to=1-2]
    \arrow["\sim", from=2-1, to=2-2]
  \end{cd}
  i.e. ``crystals are the same as sheaves equipped with equivariance
  with respect to the infinitesimal groupoid''.\footnote{
    Crys 3.1.3 actually proves this for general 
    dg ind-scheme $X$ of locally almost finite type without
    assumptions of classically formally smooth.
    However, we focused on this because it is all that's needed for
    the equivalence crystals and D-modules.
  }
\end{prop}
\begin{proof}
  
  Let us remark that \cite[Lem 1.2.4]{Crys} only proves (1) implies (2),
  but we think the two are actually equivalent.
  For the equivalence of $(1)$ and $(2)$,
  we use that colimits in $\PSTK = \PSH\,\DAFF$ are computed pointwise,
  the lemma is equivalent to claiming that
  $X(S) \to X(S_\RED)$ is effectively epic if and only if
  it is surjective on $\pi_0$.
  This is a non-trivial fact about $\infty$-groupoids which we will assume.
  \begin{lem}
    
    Let $q : X \to Y$ be a morphism of $\infty$-groupoids.
    Then $q$ is an effective epimorphism if and only if 
    it is surjective on $\pi_0$.
    \begin{proof1}
      We defer the interested reader to
      \cite[Prop 7.2.1.15]{Lurie-HTT}.

      % Let us assume $X, Y$ are sets.
      % Then the colimit of the Cech nerve $\check{C}(q)$ is
      % the same as that of a smaller portion of the simplicial object : 
      % \begin{cd}
      %   {X \times_Y X} & X
      %   \arrow[shift left=1, from=1-1, to=1-2]
      %   \arrow[shift right=1, from=1-1, to=1-2]
      % \end{cd}
      % Then $X \to Y$ exhibits $Y$ as the colimit of the about diagram
      % if and only if ``mapping out of $Y$ is the same as mapping out of $X$
      % such that fibers are mapped to the same element'',
      % in other words ``$Y$ is a quotient of $X$''.
      % This is plainly equivalent to $q$ being surjective.
      % 
      % !!! this is false. Eg. Bool -> * gives Bool x Bool => Bool.
      %     
    \end{proof1}
  \end{lem}

  Now assume $X$ is classically formally smooth.
  Since $\QCOH^* : \PSTK \to (\DGCAT_\CTS)^\OP$ is the left Kan extension of
  its restriction to $\DAFF$, it preserves small colimits
  and in particular geometric realisations.
  We thus obtain the equivalence 
  $\CRYS^L\,X \map{\sim}{} \LIM \QCOH(\check{C}(X / X_\DR))$.

  We want to use the same argument for right crystals,
  however the subtlety is that $\INDCOH^!$ is 
  \emph{not} defined on all of $\PSTK$ but rather just $\PSTK_\LAFT$.\footnote{
    This is \cite[Lem 2.3.11]{Crys}. However, 
    the proof in the reference simply claims that the argument is
    the same as for left crystals.
    We provide some unclear details.
  }
  It suffices to show that the Cech nerve of $X \to X_\DR$ lies in $\PSTK_\LAFT$
  and that $X \to X_\DR$ is an effective epimorphism \emph{in $\PSTK_\LAFT$}.
  For the first part,
  note that the morphism $X \to X_\DR$ is in $\PSTK_\LAFT$ because 
  \linkto{crys.dR.laft}{$dR$ preserves laft-ness}.
  Since \linkto{dsch.laft.fin_lim}{$\PSTK_\LAFT$ is closed under finite limits},
  we obtain that the Cech nerve of $X \to X_\DR$ in $\PSTK_\LAFT$
  agrees with the one in $\PSTK$.
  For the second part, we use \linkto{dsch.laft.char}
  {$\mathrm{res} : \PSTK_\LAFT \map{\sim}{} \PSH\,\DAFF^{<\infty}_\FT$}.
  Since colimits are computed pointwise, by the lemma above
  it suffices to show that $X(A) \to X_\DR(A)$ is a surjection on $\pi_0$
  for all $\SPEC\,A \in \DAFF^{< \infty}_\FT$.
  This is true by assumption.

\end{proof}

Here is a sufficient condition to be classically formally smooth,
whose proof we won't go into.

\begin{prop}[Comparison of classically formally smooth and smoothness]
  \link{crys.sm_implies_cfsm}
  
  Let $X$ be a laft smooth classical scheme of finite type over $k$.
  Then $X$ is classically formally smooth when considered as a prestack.
\end{prop}
\begin{proof}
  Omitted.
  It is shown in \cite[Prop 8.4.2]{DGINDSCH} that 
  smooth classical schemes of finite type over $k$
  are \emph{formally smooth} when considered as prestacks.\footnote{
  Gaitsgory--Rozenblyum's usage of the term ``formally smooth'' does \emph{not}
  immediately match the usual usage.
  The interested reader may find the definition in the reference provided.
}
With the additional assumption of laft-ness,
this implies classically formally smooth.
\end{proof}

\end{document}