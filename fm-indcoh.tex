\documentclass[./main.tex]{subfiles}
\begin{document}
  
The integral transform theorem for $\INDCOH$ follows the same strategy
as that of $\QCOH$.
\begin{cd}
  {\INDCOH(Y\times X)} & {\INDCOH\,X \otimes \INDCOH\,Y} 
  & {\DGCAT_\CTS(\INDCOH\,X , \INDCOH\,Y)}
	\arrow["\sim"',"{(1)}", from=1-2, to=1-1]
	\arrow["\sim","{(2)}"', from=1-2, to=1-3]
\end{cd}
As in the argument for $\QCOH$, 
(2) comes from a self-duality of $\INDCOH\,X$.
But let us address (1) first.

\begin{prop}[Integral Transform for Ind-Coherent Sheaves]
  \link{fm.indcoh}

  Let $X, Y \in \PSTK_\LAFT$.
  Assume $\INDCOH\,X$ is dualisable.
  Let us also assume that the base field $k$ is perfect.
  Then \[
    \boxtimes : \INDCOH\,X \otimes \INDCOH\,Y \map{\sim}{}
    \INDCOH(X \times Y)
  \]
\end{prop}
\begin{proof}

  As in the first part of the proof of
  \linkto{fm.qcoh}{integral transforms for quasi-coherent sheaves},
  we reduce to the case of $X, Y \in \DAFF^{<\infty}_\FT$.
  The result is in fact true for all $X, Y \in \DSCH_\AFT$.

  \begin{lem}[]
    \link{fm.indcoh.aft}

    Let $X , Y \in \DSCH_\AFT$.
    Then \[
      \boxtimes : \INDCOH\,X \otimes \INDCOH\,Y \map{\sim}{}
      \INDCOH(X \times Y)
    \] 
    is fully faithful.
    If the base field $k$ is perfect,
    then $\boxtimes$ is an equivalence. 

    \begin{proof1}
      We refer the reader to \cite[Ch 4 , 6.3.4]{GR1}.
    \end{proof1}
  \end{lem}
\end{proof}

It remains to describe self-duality for $\INDCOH\,X$ when $X \in \DSCH_\AFT$.
Note that by definition, 
$\INDCOH\,X$ is compactly generated and hence dualisable.
What's interesting is that the self-duality recovers 
the classical Serre duality for coherent sheaves.
First, we need to give $\INDCOH\,X$ a symmetric monoidal structure.

\begin{prop}

  Let $X , Y \in \DSCH_\AFT$. 
  Then there exists a unique functor
  \[
    \boxtimes \in \DGCAT_\CTS\brkt{
      \INDCOH\,X \otimes \INDCOH\,Y , \INDCOH(Y \times X)
    }
  \]
  that preserves compact objects and makes the following diagram commute : 
  \begin{cd}
    {\INDCOH\,X \otimes \INDCOH\,Y} & {\INDCOH(Y \times X)} \\
    {\QCOH\,X \otimes \QCOH\,Y} & {\QCOH(Y\times X)}
    \arrow["{\Psi_X \otimes \Psi_Y}"', from=1-1, to=2-1]
    \arrow["\boxtimes"', from=2-1, to=2-2]
    \arrow["{\Psi_{Y \times X}}", from=1-2, to=2-2]
    \arrow["\boxtimes", from=1-1, to=1-2]
  \end{cd}
\end{prop}
\begin{proof}
  Technical. See \cite[Ch 4 , 6.3.2]{GR1} for details.
\end{proof}

In \cite[Ch 5 , 4.1]{GR1}, functoriality of $\boxtimes$ is shown.
\begin{prop}
 
  There is a symmetric monoidal structure on 
  $\INDCOH^! : \DSCH_\AFT^\OP \to \DGCAT_\CTS$ 
  such that for $X , Y \in \DSCH_\AFT$,
  the morphism \[
    \INDCOH\,X \otimes \INDCOH\,Y \map{}{} \INDCOH(Y \times X)
  \]
  from the symmetric monoidal structure of $\INDCOH^!$
  recovers $\boxtimes$. 
\end{prop}
\begin{proof}
  Omitted. See \cite[Ch 5 , 4.1]{GR1}.
\end{proof}

\begin{rmk}
  
  Assuming the above result,
  we can give each $\INDCOH\,X$ a symmetric monoidal structure
  by the following trick from Gaitsgory--Rozenblyum.
  A symmetric monoidal dg-$\infty$-category is by their definition
  a commutative algebra object in $\DGCAT_\CTS$.
  The idea is that the source of commutative algebra structure on 
  $\INDCOH\,X$ actually comes a commutative algebra structure on
  \emph{$X$ itself}.
  This means the following.
  
\end{rmk}

\begin{lem}[Symmetric Monoidal Structure on $\PSTK$]
  
  There is a symmetric monoidal structure $\otimes$ on $\PSTK^\OP$
  that gives for each $X , Y \in \PSTK^\OP$, \[
    X \otimes Y \simeq X \times Y \in \PSTK^\OP
  \]

  Furthermore, we have a functor
  $\PSTK^\OP \to \CALG(\PSTK^\OP)$
  which for every $X \in \PSTK^\OP$,
  gives it a commutative algebra structure such that
  the multiplication
  \[
    X \times X \to X  
  \]
  is given by (the opposite of) the diagonal $\De^\OP$.
  \begin{proof1}
    The proof is not so important for our purposes.
    We refer the reader to \cite[Prop 2.4.1.5]{Lurie-HA}
    where the result is proved for any $\infty$-category
    admitting all finite products.
  \end{proof1}
\end{lem}

\begin{dfn}
  
  The symmetric monoidal functor $\INDCOH^! : \DSCH_\AFT^\OP \to \DGCAT_\CTS$
  sends commutative algebra objects to commutative algebra objects.
  So we have the following :
  \begin{cd}
    {\DSCH_\AFT^\OP} & {\DGCAT_\CTS} \\
    {\CALG(\DSCH_\AFT^\OP)} & {\CALG(\DGCAT_\CTS)}
    \arrow["{\INDCOH^!}", from=1-1, to=1-2]
    \arrow["{\text{use diagonal}}"', shift right=2, from=1-1, to=2-1]
    \arrow["{\INDCOH^!}"', from=2-1, to=2-2]
    \arrow["{\text{``forget''}}"', from=2-2, to=1-2]
    \arrow["{\text{``forget''}}"', shift right=2, from=2-1, to=1-1]
  \end{cd}
  Hence we obtain a factoring of $\INDCOH^!$ through the forgetful functor 
  $\CALG(\DGCAT_\CTS) \to \DGCAT_\CTS$.

  For $X \in \DSCH_\AFT$,
  we use $\OTIMES$ to denote the symmetric monoidal operation
  on $\INDCOH\,X$.
  
  \cite[Ch 5 , 4.1.3]{GR1}
\end{dfn}

\begin{rmk}
  
  Tracing through the above functor
  $\INDCOH^! : \DSCH_\AFT^\OP \to \CALG(\DSCH_\AFT^\OP) \to \CALG(\DGCAT_\CTS)$,
  we see that for $X \in \DSCH_\AFT$,
  the symmetric monoidal operation for $\INDCOH\,X$ is the following : 
  \begin{cd}
    {X \times X} & \rightsquigarrow & {\INDCOH\,X \otimes \INDCOH\,X} & {\INDCOH(X \times X)} \\
    X &&& {\INDCOH\,X}
    \arrow["\boxtimes", from=1-3, to=1-4]
    \arrow["{\De_X^!}", from=1-4, to=2-4]
    \arrow["\OTIMES"', from=1-3, to=2-4]
    \arrow[from=1-1, to=2-1]
  \end{cd}
\end{rmk}

\begin{prop}[``Serre Duality'']
  \link{duality.serre}
  
  Let $X \in \DSCH_\AFT$ and the base field $k$ be perfect.
  Define \[
    \<\_,\_\> : 
    \INDCOH\,X \otimes \INDCOH\,X \map{\OTIMES}{} 
    \INDCOH\,X \map{(p_X)_*}{} \VEC
  \]
  Then $\<\_,\_\>$ exhibits $\INDCOH\,X$ as its own dual.
  Let $\D_\SERRE : \INDCOH\,X \simeq (\INDCOH\,X)^\vee$
  correspond to $\<\_,\_\>$. 
  We refer to $\D_\SERRE$ as the \emph{Serre duality functor}.
\end{prop}
\begin{proof}
  Let us first say that the way in which \cite{GR1} obtains
  $\D_\SERRE$ is highly abstract.
  This is to ensure the duality has higher categorical functoriality.
  However, at \cite[Ch 5 , Rmk 4.2.8]{GR1},
  it is mentioned that one does not need all the machinery of 
  \emph{correspondences} to deduce Serre duality for each
  individual $X \in \DSCH_\AFT$ and a method is briefly explained.
  What follows is an expansion of the remark.
  
  Let $e := \<\_ , \_\>$.
  Following \cite[Ch 5 , Rmk 4.2.8]{GR1}, 
  we will explicitly give a morphism 
  $c : \VEC \to \INDCOH\,X \otimes \INDCOH\,X$ in $\DGCAT_\CTS$
  making $(c , e)$ into duality datum.
  The idea of \cite{GR1} is that
  the duality datum $(e , c)$ should come from
  a duality datum on the underlying space $X$ with itself.
  This is the philosophy of \emph{correspondences}.
  The key behind $X$ being self-dual is essentially the following :
  \begin{cd}
    {Y \times X} & R & \leftrightsquigarrow & Y \\
    & Z && R & {X \times Z}
    \arrow[from=1-2, to=1-1]
    \arrow[from=1-2, to=2-2]
    \arrow[from=2-4, to=1-4]
    \arrow[from=2-4, to=2-5]
  \end{cd}
  Applying the above to the identity correspondence 
  $X \mapfrom{}{} X \map{}{} X$ yields two correspondences :
  \begin{cd}
    {X\times X} & X & {\SPEC\,k} \\
    {} & {\SPEC\,k} & X & {X \times X}
    \arrow[from=1-2, to=1-1]
    \arrow[from=1-2, to=2-2]
    \arrow[from=2-3, to=1-3]
    \arrow[from=2-3, to=2-4]
  \end{cd}
  Using the fact \linkto{fm.indcoh.fm}
  {$\boxtimes$ is an equivalence for $\INDCOH$},
  the left correspondence gives $e$ and the right correspondence leads us to
  define $c := (\De_X)_* p_X^!$.
  We need to show the following triangles commute in $\DGCAT_\CTS$ :
  \begin{cd}
    {\INDCOH\,X} & {(\INDCOH\,X)^{\otimes 3}} & {\INDCOH\,X} \\
    {} & {\INDCOH\,X} & {(\INDCOH\,X)^{\otimes 3}} & {\INDCOH\,X}
    \arrow["{\id{} \otimes e}", from=1-2, to=2-2]
    \arrow["{\id{} \otimes c}"', from=1-3, to=2-3]
    \arrow["{e \otimes \id{}}"', from=2-3, to=2-4]
    \arrow["{c \otimes \id{}}", from=1-1, to=1-2]
    \arrow["{\id{}}"', from=1-1, to=2-2]
    \arrow["{\id{}}", from=1-3, to=2-4]
  \end{cd}
  Again, the source of the above commuting triangles
  should be at the level of spaces.
  Indeed, the functors in the above diagram come from
  the following correspondences : 
  \begin{cd}
    X & {X \times X} & X & X \\
    & {X \times X \times X} & {X \times X} & {X\times X} & {X \times X \times X} \\
    && X & X & {X \times X} & X
    \arrow["{(p_X,\id{})}"', from=1-2, to=1-1]
    \arrow["{(\De_X , \id{})}"', from=1-2, to=2-2]
    \arrow["{(\id{} , \De_X)}", from=2-3, to=2-2]
    \arrow["{(\id{} , p_X)}", from=2-3, to=3-3]
    \arrow["{\De_X}", from=1-3, to=1-2]
    \arrow["{\De_X}"', from=1-3, to=2-3]
    \arrow["\lrcorner"{anchor=center, pos=0.125, rotate=-90}, draw=none, from=1-3, to=2-2]
    \arrow["{(\id{} , p_X)}", from=2-4, to=1-4]
    \arrow["{(\id{} , \De_X)}", from=2-4, to=2-5]
    \arrow["{(\De_X , \id{})}"', from=3-5, to=2-5]
    \arrow["{(p_X , \id{})}"', from=3-5, to=3-6]
    \arrow["{\De_X}"', from=3-4, to=2-4]
    \arrow["{\De_X}", from=3-4, to=3-5]
    \arrow["\lrcorner"{anchor=center, pos=0.125, rotate=90}, draw=none, from=3-4, to=2-5]
  \end{cd}
  Thus, we are reduced to showing $\INDCOH$ base change for 
  the cartesian square :
  \begin{cd}
    {X \times X} & X \\
    {X \times X \times X} & {X \times X}
    \arrow["{(\De_X , \id{})}"', from=1-1, to=2-1]
    \arrow["{(\id{} , \De_X)}", from=2-2, to=2-1]
    \arrow["{\De_X}"', from=1-2, to=1-1]
    \arrow["{\De_X}", from=1-2, to=2-2]
    \arrow["\lrcorner"{anchor=center, pos=0.125, rotate=-90}, draw=none, from=1-2, to=2-1]
  \end{cd}
  where all morphisms are closed embeddings
  (since in \cite{GR1}'s definition of $\DSCH$ the diagonal morphism is
  assumed to be a closed embedding).
  This follows from \linkto{indcoh.proper_bc}{proper base change for $\INDCOH$}.

\end{proof}

\end{document}